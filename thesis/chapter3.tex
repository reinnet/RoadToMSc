
\chapter{کارهای مرتبط}

\section{مقدمه}
در این بخش تحقیقات مرتبط با استقرار سرویس و تخصیص منابع در معماری‌های \lr{NFV}  و \lr{SFC} را مورد بررسی قرار می‌دهیم.
ابتدا ابعاد مختلف مسائل تحقیقاتی و به خصوص مسئله تخصیص منابع را مورد بررسی قرار می‌دهیم.
سپس کارهایی که در زمینه استقرار سرویس و تخصیص منابع مدیریتی انجام شده‌اند را مرور می‌کنیم.

\section{ابعاد مختلف مسائل تحقیقاتی}
در این بخش با توجه به معماری های \lr{SFC} و \lr{NFV} ابعاد مختلف مسائل تحقیقاتی را مورد بررسی قرار می‌دهیم.
از آنجایی که موضوع این رساله بر مسئله تخصیص منابع تمرکز دارد بر این بخش تمرکز بیشتری خواهیم داشت.
در این تحقیق تخصیص منابع را به صورت اختصاص منابع شبکه، پردازشی، محاسباتی و ذخیره‌سازی
به ماشین‌‌های مجازی اجراکننده کارکردها و اختصاص پهنای باند به لینک‌های مجازی بین کارکردها بر روی شبکه ارتباطی زیرساخت تعریف می‌کنیم.
در این راستا باید مشخص شود که ماشین‌های مجازی اجرا کننده کارکردها که به عنوان نمونه ایجاد شده از کارکردها شناخته می شوند
بر روی چه سرورهایی ایجاد شوند.
این فرآیند را نگاشت کارکردها به نمونه‌ها می گوییم.
همچنین باید مشخص شود چه میزان پهنای باند از چه لینک‌هایی به به لینک‌های مجازی اختصاص یابد.
ممکن است پهنای باند یک لینک، بر روی چند لینک و یا چند مسیر در شبکه زیرساخت اختصاص پیدا کند.
به این فرایند نیز نگاشت لینک‌های مجازی گفته می‌شود.
با توجه به توضیحات گفته شده در ادامه ابعاد مختلف مسائل تحقیقاتی را شرح می دهیم.

\subsection{دیدگاه تعریف مساله}
معماری آینده اینترنت بر اساس مدل تجاری \lr{IaaS} است که
نقش \lr{ISP}‌ها به دو نقش فراهم کننده سرویس (\lr{SP}) و فراهم کننده زیرساخت (\lr{IP})، تبدیل می‌شود
فراهم کننده سرویس مسئول ارائه سرویس انتها به انتها به کاربران بر روی زیرساختی است که
از سمت \lr{IP} ارائه می‌شود و مسئولیت مدیریت منابع آن را برعهده دارد.
بر اساس این دو نقش، مسائل را از دو جنبه می توان دسته بندی کرد:

\begin{itemize}
    \item مسائلی که در آن صرفا بحث سرویس گرفتن از یک یا چند \lr{IP} مطرح است. از آنجایی که سرویس گیرنده خود تخصیص منابع را انجام نمی‌دهد، تمرکز این گونه مسئله‌ها بر روی مسائل قیمت گذاری در یک بازار \lr{NFV} خواهد بود.
    \item مسائلی که در آن‌ها تخصیص بهینه منابع به سرویس نیز مورد توجه است. در این حالت \lr{IP} از یک مرکز داده متمرکز یا چندین مرکز داده توزیع شده برای ارائه سرویس به کاربران استفاده می‌کند. در این بخش وابسته به سطح  انتزاع مسئله، \lr{NFVI-PoP} را می‌توان یک سرور یا یک مرکز داده در نظر گرفت. طبیعتا وابسته به سطح انتزاع، همبندی‌‌های مختلف شبکه ارتباطی را نیز می‌توان در نظر گرفت.
\end{itemize}

در هر یک از این دیدگاه‌ها می توان فرضیاتی را در نظر گرفت که منجر به مسائل متفاوتی می‌شود.
به عنوان مثال در دسته اول نحوه قیمت گذاری، همکاری و یا عدم همکاری \lr{IP}‌ها و کاربران را می‌توان مورد مطالعه قرار داد.
در مسائل تخصیص منابع هم اگر یک مرکز داده وجود داشته باشد،
درباره شبکه زیرساخت می توان انواع همبندی‌های \lr{Switch centeric} و یا \lr{Server centric} را در نظر گرفت
که تاثیر زیادی بر تعریف مسائل دارند.
زمانی که چندین مرکز داده توزیع شده از نظر جغرافیایی وجود داشته باشد،
مسائل مهمی از جمله نحوه کاهش ترافیک بین مراکز داده مطرح می‌شود.
همانگونه که بیان شد تمرکز این تحقیق بر دسته دوم مسائل یعنی تخصیص بهینه منابع به سرویس است.

\section{مرور پژوهش‌ها}

در \cite{Eramo2016}
نویسندگان قصد دارند با در نظر گرفتن محدودیت ظرفیت لینک‌ها و محدودیت پردازشی نودها
بیشترین تعداد زنجیره‌ی کارکرد را بپذیرند. برای این کار یک مساله‌ی \lr{ILP}
طراحی می‌کنند و ثابت می‌کنند که این مساله \lr{NP-Hard} می‌باشد.
با توجه به \lr{NP-Hard}
بودن مساله الگوریتم مکاشفه‌ای \lr{MASRN}
پیشنهاد می‌گردد.این الگوریتم یک الگوریتم حریصانه می‌باشد که براساس
منابع سرورها و بار لینک‌ها جایگذاری را انجام می‌هد.
در این مقاله وجود \lr{VNFM} برای زنجیره‌ها در نظر گرفته نشده است.

در \cite{AbuLebdeh2017}
نویسندگان استفاده از \lr{VNFM} را مدنظر قرار داده‌اند.
 در این مقاله فرض شده است که جایگذاری \lr{SFC}ها صورت گرفته است
و می‌خواهیم \lr{VNFM}ها را به صورت دوره‌ای با تغییر نگاشت زنجیره‌ها به گونه‌ای بازنگاشت کنیم
که با رعایت شدن نیازمندی‌های کارآیی، هزینه‌ی عملیاتی سیستم حداقل شود.
مساله مطرح شده به صورت \lr{ILP} مدلسازی می‌شود.
این مقاله هزینه‌ی عملیاتی سیستم را تحت چهار عنوان دسته‌بندی می‌کند:
هزینه‌ی مدیریت چرخه‌ی زندگی، هزینه‌ی منابع محاسباتی، هزینه‌ی مهاجرت و هزینه‌ی بازنگاشت.
در این مقاله فرض می‌شود که هر نمونه از \lr{VNFM}ها می‌تواند به تعداد مشخصی از نمونه‌های \lr{VNF}
سرویس‌دهی کند و این سرویس‌دهی به نوع نمونه وابسته نیست.
این مقاله محدودیت‌های پردازشی و ظرفیتی را مدنظر قرار می‌دهد.

در \cite{Ghaznavi2017}
نویسندگان سه مرحله برای عملیات جایگذاری زنجیره‌های کارکرد سرویس معرفی می‌کنند:
انتخاب،
جابگذاری و
مسیریابی.
در این مقاله فرض می‌شود برای هر نوع \lr{VNF}
چند مدل مختلف با مصرف منابع مختلف وجود دارند که می‌توان از آن‌ها نمونه ساخت، در این مرحله مشخص می‌شود
از کدام مدل نمونه‌سازی صورت می‌گیرد.
این مقاله جایگذاری یک \lr{SFC} را مدل‌سازی می‌کند،
در این مقاله فرض می‌شود جریان ورودی و خروجی از هر نمونه برابر بوده و در واقع
\lr{VNF} تغییری بر روی ترافیک ایجاد نمی‌کند.
در مدل‌سازی این مقاله که به صورت \lr{ILP} می‌باشد هدف کاهش هزینه در جایگذاری \lr{SFC} داده شده می‌باشد.
با در نظر گرفتن مدل‌های مختلف برای \lr{VNF}ها در این مقاله
در صورتی که نیاز به پردازش ترافیک زیادی باشد، چند نمونه از یک نوع \lr{VNF}
ساخته می‌شود و ترافیک بین آن‌ها تقسیم می‌شود.

در \cite{Yu2017}
نویسندگان برای اولین‌بار مساله‌ی \lr{Traffic Streering}
با در نظر گرفتن \lr{QoS} و \lr{Reliability}
فرمول‌بندی کرده‌اند.
این مقاله کاربرد \lr{NFV} را در شبکه‌های موبایل مدنظر قرار داده است.
در این مقاله مساله به صورت \lr{Link-Path}
مدل‌سازی شده است و فرض شده است که مسیرهای ممکن برای جایگذاری کلاس‌های ترافیکی از پیش تعیین شده‌اند.
در این مقاله منظور از کیفیت سرویس تاخیر و گذردهی کلاس‌های ترافیکی می‌باشد و 
برای فراهم آوردن قابلیت اطمینان فرض می‌شود که خرابی‌ها به صورت دلخواه بوده و در صورت خرابی‌
بخشی از پهنای باند از دست می‌رود.


در \cite{Huang2017}
نویسندگان مساله‌ی جایگذاری و مسیریابی زنجیره‌های کارکرد سرویس را به صورت توامان مدل‌سازی می‌کنند،
در این مساله نویسندگان تاثیر دو پارامتر \lr{Coordination Effect} و \lr{Traffic-Change Effect}
را نیز مدنظر قرار داده‌اند.
زمانی که چند \lr{VM} در پیاده‌سازی یک کارکرد شبکه استفاده می‌شوند
نیاز است که بین این ماشین‌های مجازی هماهنگی صورت بگیرد.
برای این هماهنگی ارتباطاتی صورت می‌گیرد که دارای سربار بوده و به این سربار
\lr{Coordination Effect} می‌گویند.
هر کارکرد شبکه می‌تواند روی ترافیک ورودی خود تاثیر گذاشته و نرخ آن را تغییر دهد
که این موضوع را با \lr{Traffic-Change Effect} بیان می‌کنند.

در \cite{Chen2017}
نویسندگان قصد دارند به صورت قطعی کیفیت سرویس را گارانتی نمایند.
این مقاله پیاده‌سازی \lr{NFV} را با استفاده از \lr{SDN} هدف قرار می‌دهد
و برای محاسبه‌ی تاخیر، تاخیر پیام‌های کنترلی \lr{SDN} و
تاخیر جابجایی بسته‌ها را در نظر می‌گیرد.
برای پیشنهاد یک راه‌حل قطعی از \lr{Network Calculus}
استفاده می‌شود که شرایط مرزی را بررسی می‌کند.
این شرایط مرزی برای پیام‌های کنترلی محاسبه شده
و از آن تاخیر مورد نظر در جابجایی بسته‌ها بدست می‌آید
که با استفاده از آن یک مساله‌ی بهینه‌سازی با هدف رعایت تاخیر بدست آمده حاصل می‌شود.

در \cite{Ma2017}
نویسندگان پیاده‌سازی \lr{NFV} با \lr{SDN}
را هدف قرار داده‌اند و جایگذاری \lr{middle box}ها
با هدف توزیع‌بار را فرمول‌بندی کرده‌اند.
در واقع \lr{middle box}ها
در این مقاله به صورت مجازی بوده و همان کارکردهای مجازی شبکه می‌باشند.
مدل‌سازی صورت گرفته به صورت \lr{node link} صورت پذیرفته است.
هدف مساله مسیریابی چند مسیره برای تقاضا به صورتی است که در آن
\lr{link load ratio} برای تمام لینک‌ها می‌نیمم شود.
این مقاله تغییر ترافیک توسط کارکردها را نیز مدنظر قرار داده است.

در \cite{Jang2017}
مساله‌ی جایگذاری زنجیره‌های کاکرد سرویس با دو هدف کاهش مصرف انرژی و افزایش نرخ جریان پذیرفته شده
مدل‌سازی می‌شود. این مدل‌سازی با توجه به معماری \lr{IETF SFC} صورت پذیرفته است.
در مدلسازی این مقاله جزئیات زیادی مورد توجه قرار گرفته است که این امر باعث پیچیده شدن
فرمول‌بندی شده است.

در \cite{Eramo2017}
نویسندگان ابتدا مساله‌ی جایگذاری و مسیریابی \lr{VNF}ها را
در اوج ترافیک حل می‌کنند. در ادامه آن‌ها فرض می‌کنند که ترافیک به صورت دوره‌ای-ثابت می‌باشد
به این معنا که ترافیک در تعداد متناهی بازه‌ی زمانی تعریف شده و تکرار می‌شود.
با این فرض در ادامه مقاله مساله‌ی دیگری مبنی بر مهاجرت نمونه‌ها با توجه به تغییر ترافیک را مطرح می‌کند.
در این مهاجرت‌ها مقاله از توان مصرفی در مهاجرت صرف نظر کرده و تلاش می‌کند جریمه‌ای که بابت قطعی سرویس پرداخت می‌شود
و توان مصرفی کل سیستم را بهینه کند.

در \cite{Pham2017}
نویسندگان مساله‌ی توزیع‌بار در \lr{NFV} را بررسی می‌کنند،
آن‌ها در این مساله ویژگی‌های پایه‌ای \lr{NFV} در کنار استفاده از
روش \lr{ECMP} مدنظر قرار می‌دهد.
در روش \lr{ECMP} بار بین مبدا و مقصد
به صورت یکسان بین تمام مسیرها تقسیم می‌گردد.
در این مساله تعدادی تقاضا در نظر گرفته می‌شود که کوتاهترین مسیرها بین مبدا و مقصد آن‌ها مشخص است
و در نهایت بار در این مسیرها توزیع شده و کارکردها شبکه‌ای نیز در این مسیرها مستقر می‌شوند.

\begin{table}[h]
    \caption{مقایسه مقالات پذیرش زنجیره‌های کارکرد سرویس}
    \label{table.1}
    \vspace{0.5cm}
    \begin{tabularx}{\textwidth}{ccccccccc}
        \toprule
        منبع &
        \multicolumn{4}{c}{منابع تخصیص یافته} &
        \multicolumn{2}{c}{محدودیت ظرفیت پردازشی نمونه} &
        \multicolumn{2}{c}{برخط یا برون خط} \\
        \midrule
        \lr{\#} &
        \lr{other} &
        \lr{MEM} &
        \lr{BW} &
        \lr{CPU} &
        دارد &
        ندارد &
        برخط &
        برون خط \\
        \midrule
        \cite{Eramo2016} &
        \lr{---} &
        \lr{---} &
        \checkmark&
        \checkmark&
        \lr{---}&
        \checkmark&
        \lr{---} &
        \checkmark\\
        \midrule
        \cite{Ghaznavi2017} &
        \lr{---} &
        \lr{---} &
        \checkmark&
        \checkmark&
        \checkmark&
        \lr{---} &
        \lr{---} &
        \checkmark\\
        \midrule
        \cite{Huang2017} &
        \lr{---} &
        \lr{---} &
        \checkmark&
        \checkmark&
        \checkmark&
        \lr{---} &
        \lr{---} &
        \checkmark\\
        \midrule
        پژوهش حاضر &
        \lr{---} &
        \checkmark&
        \checkmark&
        \checkmark&
        \checkmark&
        \lr{---} &
        \lr{---} &
        \checkmark\\
        \bottomrule
    \end{tabularx}
    \begin{tabularx}{\textwidth}{ccccccccc}
        \toprule
        منبع &
        \multicolumn{2}{c}{نگاشت کارکرد و لینک} &
        \multicolumn{2}{c}{انتساب کارکرد} &
        \multicolumn{2}{c}{اشتراک نمونه} &
        \multicolumn{2}{c}{تخصیص \lr{VNFM}} \\
        \midrule
        \lr{\#} &
        کارکرد &
        لینک &
        یک نمونه &
        چند نمونه &
        دارد &
        ندارد &
        دارد &
        ندارد \\
        \midrule
        \cite{Eramo2016} &
        \checkmark&
        \checkmark&
        \checkmark&
        \lr{---} &
        \lr{---} &
        \checkmark&
        \lr{---} &
        \checkmark\\
        \midrule
        \cite{Ghaznavi2017} &
        \checkmark&
        \checkmark&
        \lr{---} &
        \checkmark&
        \lr{---} &
        \checkmark&
        \lr{---} &
        \checkmark\\
        \midrule
        \cite{Huang2017} &
        \checkmark&
        \checkmark&
        \lr{---} &
        \checkmark&
        \lr{---} &
        \checkmark&
        \lr{---} &
        \checkmark\\
        \midrule
        پژوهش حاضر &
        \checkmark&
        \checkmark&
        \checkmark&
        \lr{---}&
        \lr{---}&
        \checkmark&
        \lr{---} &
        \checkmark\\
        \bottomrule
    \end{tabularx}
\end{table}

همانطور که در جدول \ref{table.1} دیده می‌شود،‌ مساله‌ی تخصیص منابع مدیریتی در پذیرش زنجیره‌های کارکرد
مورد بررسی قرار نگرفته است و در پژوهش حاضر قصد داریم این مورد را بررسی نماییم.

نزدیک‌ترین پژوهش به آنچه در این رساله انجام شده است، \cite{AbuLebdeh2017} می‌باشد.
این پژوهش قصد دارد با فرض مشخص بودن نگاشت زنجیره‌ها و \lr{NFVO} نگاشت \lr{VNFM} را انجام دهد.
این پژوهش فرض می‌کند که در طی بازه‌های زمانی مختلف نگاشت زنجیره‌ها تغییر می‌کند و این الگوریتم قصد دارد با کمترین هزینه‌‌ی عملیانی نگاشت دوباره \lr{VNMF}ها را انجام دهد.
در این نگاشت پارامترهایی مانند تاخیر، هزینه‌ی پهنای باند، هزینه‌ی منابع مدیریتی را مدنظر قرار می‌گیرند.
در بازنگاشت \lr{VNFM}ها لینک ارتباطی آن‌ها با \lr{VNFO} نیز مدنظر است.

پژوهش \cite{AbuLebdeh2017} برای \lr{VNFM}ها نوع متصور شده و
فرض می‌کند هر نوع از \lr{VNFM}ها می‌تواند از تعداد مشخصی نمونه \lr{VNF} پشتیبانی کند،
این فرض مشابه فرضی می‌باشد که در پژوهش حاضر نیز صورت گرفته است.

تفاوت‌های اصلی پژوهش حاضر با پژوهش \cite{AbuLebdeh2017} به شرح زیر می‌باشد:
\begin{itemize}
    \item پژوهش حاضر هزینه‌ی گواهی \lr{VNFM}ها را مدنظر قرار داده است.
    \item پژوهش حاضر یک مساله پذیرش برای زنجیره‌ها را حل می‌کند که در این پذیرش در کنار نگاشت زنجیره‌ها منابع مدیریتی آن‌ها را نیز نگاشت می‌کند.
    \item پژوهش حاضر \lr{VNFO} را در نظر نمی‌گیرد و مساله را برای یک دیتاسنتر توزیع نشده مطرح می‌کند.
    \item
    پژوهش حاضر ممکن است در جهت افزایش سود زنجیره‌ای را نپذیرد چرا که ممکن است نگاشت \lr{VNFM} برای آن سربار زیادی داشته باشد و
    این در حالی است که پژوهش \cite{AbuLebdeh2017} چنین فرضی نداشته و برای تمامی زنجیره‌ها منابع مدیریتی را نگاشت می‌کند.
\end{itemize}
در نهایت با توجه به متفاوت بودن تابع‌های هدف این دو پژوهش نمی‌توان آن‌ها را با یکدیگر مقایسه کرد.

\pagebreak
\section{جمع‌بندی}

در این فصل انواع مسائل مطرح در حوزه‌ی \lr{NFV} و \lr{SFC}
به صورت خلاصه مرور گشت. همانطور که پیشتر بیان شده بود مساله‌ی پیشنهادی یک مساله‌ی جایگذاری
یا نگاشت سرویس می‌باشد.
در ادامه پژوهش‌های این حوزه مرور شده و تفاوت‌های آن‌ها با پژوهش حاضر ذکر شد.
در آخر پژوهش حاضر سعی دارد برای اولین‌بار نگاشت سرویس‌ها را با در نظر گرفتن نیاز آن‌ها
به منابع مدیریتی مدنظر قرار دهد.