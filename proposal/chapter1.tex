\chapter{راهنمای استفاده از الگوی لاتک دانشگاه صنعتی امیرکبیر(پلی‌تکنیک تهران)}

\section{مقدمه}
حروف‌چینی پروژه کارشناسی، پایان‌نامه یا رساله یکی از موارد پرکاربرد استفاده از زی‌پرشین است. از طرفی، یک پروژه، پایان‌نامه یا رساله،  احتیاج به تنظیمات زیادی از نظر صفحه‌آرایی  دارد که ممکن است برای
یک کاربر مبتدی، مشکل باشد. به همین خاطر، برای راحتی کار کاربر، یک کلاس با نام 
\verb;AUTthesis;
 برای حروف‌چینی پروژه‌ها، پایان‌نامه‌ها و رساله‌های دانشگاه صنعتی امیرکبیر با استفاده از نرم‌افزار زی‌پرشین،  آماده شده است. این فایل به 
گونه‌ای طراحی شده است که کلیه خواسته‌های مورد نیاز  مدیریت تحصیلات تکمیلی دانشگاه صنعتی امیرکبیر را برآورده می‌کند و نیز، حروف‌چینی بسیاری
از قسمت‌های آن، به طور خودکار انجام می‌شود.

کلیه فایل‌های لازم برای حروف‌چینی با کلاس گفته شده، داخل پوشه‌ای به نام
\verb;AUTthesis;
  قرار داده شده است. توجه داشته باشید که برای استفاده از این کلاس باید فونت‌های
  \verb;Nazanin B;،
 \verb;PGaramond;
 و
  \verb;IranNastaliq;
    روی سیستم شما نصب شده باشد.
\section{این همه فایل؟!}\label{sec2}
از آنجایی که یک پایان‌نامه یا رساله، یک نوشته بلند محسوب می‌شود، لذا اگر همه تنظیمات و مطالب پایان‌نامه را داخل یک فایل قرار بدهیم، باعث شلوغی
و سردرگمی می‌شود. به همین خاطر، قسمت‌های مختلف پایان‌نامه یا رساله  داخل فایل‌های جداگانه قرار گرفته است. مثلاً تنظیمات پایه‌ای کلاس، داخل فایل
\verb;AUTthesis.cls;، 
تنظیمات قابل تغییر توسط کاربر، داخل 
\verb;commands.tex;،
قسمت مشخصات فارسی پایان‌نامه، داخل 
\verb;fa_title.tex;,
مطالب فصل اول، داخل 
\verb;chapter1;
و ... قرار داده شده است. نکته مهمی که در اینجا وجود دارد این است که از بین این  فایل‌ها، فقط فایل 
\verb;AUTthesis.tex;
قابل اجرا است. یعنی بعد از تغییر فایل‌های دیگر، برای دیدن نتیجه تغییرات، باید این فایل را اجرا کرد. بقیه فایل‌ها به این فایل، کمک می‌کنند تا بتوانیم خروجی کار را ببینیم. اگر به فایل 
\verb;AUTthesis.tex;
دقت کنید، متوجه می‌شوید که قسمت‌های مختلف پایان‌نامه، توسط دستورهایی مانند 
\verb;input;
و
\verb;include;
به فایل اصلی، یعنی 
\verb;AUTthesis.tex;
معرفی شده‌اند. بنابراین، فایلی که همیشه با آن سروکار داریم، فایل 
\verb;AUTthesis.tex;
است.
در این فایل، فرض شده است که پایان‌نامه یا رساله شما، از5 فصل و یک پیوست، تشکیل شده است. با این حال، اگر
  پایان‌نامه یا رساله شما، بیشتر از 5 فصل و یک پیوست است، باید خودتان فصل‌های بیشتر را به این فایل، اضافه کنید. این کار، بسیار ساده است. فرض کنید بخواهید یک فصل دیگر هم به پایان‌نامه، اضافه کنید. برای این کار، کافی است یک فایل با نام 
\verb;chapter6;
و با پسوند 
\verb;.tex;
بسازید و آن را داخل پوشه 
\verb;AUTthesis;
قرار دهید و سپس این فایل را با دستور 
\texttt{\textbackslash include\{chapter6\}}
داخل فایل
\verb;AUTthesis.tex;
و بعد از دستور
\texttt{\textbackslash include\{chapter6\}}
 قرار دهید.

\section{از کجا شروع کنم؟}
قبل از هر چیز، بدیهی است که باید یک توزیع تِک مناسب مانند 
\verb;Live TeX;
و یک ویرایش‌گر تِک مانند
\verb;Texmaker;
را روی سیستم خود نصب کنید.  نسخه بهینه شده 
\verb;Texmaker;
را می‌توانید  از سایت 
 \href{http://www.parsilatex.com}{پارسی‌لاتک}%
\LTRfootnote{\url{http://www.parsilatex.com}}
 و
\verb;Live TeX;
را هم می‌توانید از 
 \href{http://www.tug.org/texlive}{سایت رسمی آن}%
\LTRfootnote{\url{http://www.tug.org/texlive}}
 دانلود کنید.
 
در مرحله بعد، سعی کنید که  یک پشتیبان از پوشه 
\verb;AUTthesis;
 بگیرید و آن را در یک جایی از هارددیسک سیستم خود ذخیره کنید تا در صورت خراب کردن فایل‌هایی که در حال حاضر، با آن‌ها کار می‌کنید، همه چیز را از 
 دست ندهید.
 
 حال اگر نوشتن پایان‌نامه اولین تجربه شما از کار با لاتک است، توصیه می‌شود که یک‌بار به طور سرسری، کتاب «%
\href{http://www.tug.ctan.org/tex-archive/info/lshort/persian/lshort.pdf}{مقدمه‌ای نه چندان کوتاه بر
\lr{\LaTeXe}}\LTRfootnote{\url{http://www.tug.ctan.org/tex-archive/info/lshort/persian/lshort.pdf}}»
   ترجمه دکتر مهدی امیدعلی، عضو هیات علمی دانشگاه شاهد را مطالعه کنید. این کتاب، کتاب بسیار کاملی است که خیلی از نیازهای شما در ارتباط با حروف‌چینی را برطرف می‌کند.
 
 
بعد از موارد گفته شده، فایل 
\verb;AUTthesis.tex;
و
\verb;fa_title;
را باز کنید و مشخصات پایان‌نامه خود مثل نام، نام خانوادگی، عنوان پایان‌نامه و ... را جایگزین مشخصات موجود در فایل
\verb;fa_title;
 کنید. دقت داشته باشید که نیازی نیست 
نگران چینش این مشخصات در فایل پی‌دی‌اف خروجی باشید. فایل 
\verb;AUTthesis.cls;
همه این کارها را به طور خودکار برای شما انجام می‌دهد. در ضمن، موقع تغییر دادن دستورهای داخل فایل
\verb;fa_title;
 کاملاً دقت کنید. این دستورها، خیلی حساس هستند و ممکن است با یک تغییر کوچک، موقع اجرا، خطا بگیرید. برای دیدن خروجی کار، فایل 
\verb;fa_title;
 را 
\verb;Save;، 
(نه 
\verb;As Save;)
کنید و بعد به فایل 
\verb;AUTthesis.tex;
برگشته و آن را اجرا کنید. حال اگر می‌خواهید مشخصات انگلیسی پایان‌نامه را هم عوض کنید، فایل 
\verb;en_title;
را باز کنید و مشخصات داخل آن را تغییر دهید.%
\RTLfootnote{
برای نوشتن پروژه کارشناسی، نیازی به وارد کردن مشخصات انگلیسی پروژه نیست. بنابراین، این مشخصات، به طور خودکار،
نادیده گرفته می‌شود.
}
 در اینجا هم برای دیدن خروجی، باید این فایل را 
\verb;Save;
کرده و بعد به فایل 
\verb;AUTthesis.tex;
برگشته و آن را اجرا کرد.

برای راحتی بیشتر، 
فایل 
\verb;AUTthesis.cls;
طوری طراحی شده است که کافی است فقط  یک‌بار مشخصات پایان‌نامه  را وارد کنید. هر جای دیگر که لازم به درج این مشخصات باشد، این مشخصات به طور خودکار درج می‌شود. با این حال، اگر مایل بودید، می‌توانید تنظیمات موجود را تغییر دهید. توجه داشته باشید که اگر کاربر مبتدی هستید و یا با ساختار فایل‌های  
\verb;cls;
 آشنایی ندارید، به هیچ وجه به این فایل، یعنی فایل 
\verb;AUTthesis.cls;
دست نزنید.

نکته دیگری که باید به آن توجه کنید این است که در فایل 
\verb;AUTthesis.cls;،
سه گزینه به نام‌های
\verb;bsc;,
\verb;msc;
و
\verb;phd;
برای تایپ پروژه، پایان‌نامه و رساله،
طراحی شده است. بنابراین اگر قصد تایپ پروژه کارشناسی، پایان‌نامه یا رساله را دارید، 
 در فایل 
\verb;AUTthesis.tex;
باید به ترتیب از گزینه‌های
\verb;bsc;،
\verb;msc;
و
\verb;phd;
استفاده کنید. با انتخاب هر کدام از این گزینه‌ها، تنظیمات مربوط به آنها به طور خودکار، اعمل می‌شود.

\section{مطالب پایان‌نامه را چطور بنویسم؟}
\subsection{نوشتن فصل‌ها}
همان‌طور که در بخش 
\ref{sec2}
گفته شد، برای جلوگیری از شلوغی و سردرگمی کاربر در هنگام حروف‌چینی، قسمت‌های مختلف پایان‌نامه از جمله فصل‌ها، در فایل‌های جداگانه‌ای قرار داده شده‌اند. 
بنابراین، اگر می‌خواهید مثلاً مطالب فصل ۱ را تایپ کنید، باید فایل‌های 
\verb;AUTthesis.tex;
و
\verb;chapter1;
را باز کنید و محتویات داخل فایل 
\verb;chapter1;
را پاک کرده و مطالب خود را تایپ کنید. توجه کنید که همان‌طور که قبلاً هم گفته شد، تنها فایل قابل اجرا، فایل 
\verb;AUTthesis.tex;
است. لذا برای دیدن حاصل (خروجی) فایل خود، باید فایل  
\verb;chapter1;
را 
\verb;Save;
کرده و سپس فایل 
\verb;AUTthesis.tex;
را اجرا کنید. یک نکته بدیهی که در اینجا وجود دارد، این است که لازم نیست که فصل‌های پایان‌نامه را به ترتیب تایپ کنید. می‌توانید ابتدا مطالب فصل ۳ را تایپ کنید و سپس مطالب فصل ۱ را تایپ کنید.

نکته بسیار مهمی که در اینجا باید گفته شود این است که سیستم
\lr{\TeX},
محتویات یک فایل تِک را به ترتیب پردازش می‌کند. به عنوان مثال، اگه فایلی، دارای ۴ خط دستور باشد، ابتدا خط ۱، بعد خط ۲، بعد خط ۳ و در آخر، خط ۴ پردازش می‌شود. بنابراین، اگر مثلاً مشغول تایپ مطالب فصل ۳ هستید، بهتر است
که دو دستور
\verb~\chapter{مقدمه}

پیشتر در ارائه سرویس‌های شبکه، از سخت‌افزارهای اختصاصی که توسط سازندگان اختصاصی ارائه می‌شد و به آن‌ها
\lr{middle box}
گفته می‌شد استفاده می‌گشت.
تنوع و تعداد رو به افزایش سرویس‌های جدیدی که توسط کاربران تقاضا می‌گردد
باعث هزینه‌های زیاد برای خرید و نگهداری
\lr{middle box}‌ها
توسط اپراتورها شده است.
به تازگی فراهم آورندگان شبکه
شروع به حرکت به سوی مجازی‌سازی و نرم‌افزاری کردن بسترهای شبکه کرده‌اند،
به این ترتیب آن‌ها قادر خواهند بود
سرویس‌های نوآورانه‌ای به کاربران ارائه بدهند.
این روند به سرویس دهندگان اجازه می‌دهد که ارائه سرویس‌های دلخواه‌شان وابسته به سخت‌افزارهای اختصاصی نباشد و 
هزینه‌های راه‌اندازی و نگهداری فراهم آوردندگان سرویس را کاهش می‌دهد.
با نرم‌افزاری سازی کارکردها، وابستگی آن‌ها به سخت افزار اختصاصی کاهش یافته و به سرعت می‌توان آن‌ها را افزایش/کاهش مقیاس داد.
مجازی‌سازی کارکردهای شبکه و زنجیره‌سازی کارکرد سرویس‌ راهکاری‌هایی هستند که برای همین منظور پیشنهاد شده‌اند.

ایده‌ی اصلی مجازی‌سازی توابع شبکه جداسازی تجهیزات فیزیکی شبکه از کارکردهایی می‌باشد که
بر روی آن‌ها اجرا می‌شوند.
به این معنی که یک کارکرد شبکه مانند دیوار آتش می‌تواند بر روی سرورهای
\lr{HVS}\footnote{\lr{High Volume Server}}
به عنوان یک نرم‌افزار ساده مستقر شود.
با این روش یک سرویس می‌تواند با استفاده از کارکردهای مجازی شبکه‌ای، که می‌توانند به صورت نرم‌افزاری پیاده‌سازی شده
و روی یک یا تعدادی سرور استاندارد فیزیکی اجرا شوند، استقرار یابد.
کارکردهای مجازی شبکه‌ای می‌توانند در مکان‌های مختلف بازمکان‌یابی یا نمونه‌سازی شوند بدون آنکه
نیاز به خریداری و نصب تجهیز جدیدی باشد.
\cite{Mijumbi2016}

در ادامه به معرفی معماری \lr{NFV} پرداخته
و به چالش‌هایی که در \lr{MANO} وجود دارد می‌پردازیم.
در فصل دوم کارهای مرتبط مرور می‌شوند و در فصل سوم مساله تعریف شده بیان می‌گردد. در فصل چهارم
در مورد راه‌حل پیشنهادی برای مساله بحث خواهد شد.

\section{معماری \lr{NFV}}
با توجه به استاندارد \lr{ETSI} معماری \lr{NFV}
از سه عنصر کلیدی تشکیل شده است.
زیرساخت مجازی‌سازی کارکردهای شبکه،
کارکردهای مجازی شبکه‌ای و
\lr{NFV MANO}.
این اجزا در شکل \cref{fig.1} نمایش داده شده‌اند.

\begin{figure}[!h]
\center\includegraphics[scale=.5]{images/nfv-arch}
\caption{معماری مجازی‌سازی کارکردهای شبکه
}\label{fig.1}
\end{figure}

\subsection{زیرساخت مجازی‌سازی کارکردهای شبکه}
زیرساخت مجازی‌سازی کارکردهای شبکه ترکیبی از منابع نرم‌افزاری و سخت‌افزاری است
که محیطی برای نصب
کارکردهای مجازی شبکه فراهم می‌آورد.
منابع سخت‌افزاری شامل منابع محاسباتی،
ذخیره‌سازها و شبکه
(شامل لینک‌ها و گره‌ها)
هستند
که پردازش، ذخیره‌سازی و ارتباط را
برای کارکردهای مجازی شبکه فراهم می‌آورند.
منابع مجازی انتزاعی از منابع شبکه‌ای، پردازشی و ذخیر‌ه‌سازی هستند.
به وسیله انتزاع از طریق لایه‌ی مجازی‌سازی (بر پایه‌ی \lr{hypervisor})
منابع سخت افزاری در اختیار کارکردهای مجازی
قرار می‌گیرند که این منابع شامل منابع محاسباتی، شبکه‌ای و ذخیره‌سازی می‌باشند.

در مراکز داده‌ای ممکن است منابع پردازشی و ذخیره‌سازی تحت عنوان یک یا چند
ماشین مجازی نمایش داده شوند در حالی که شبکه‌های مجازی از لینک‌ها و گره‌های مجازی تشکیل می‌شوند.
شبکه‌های مجازی پیش از بحث مجازی‌سازی کارکردهای شبکه مدنظر بوده‌اند و روی آن‌ها کار شده است.
در واقع از شبکه‌های مجازی در مراکز داده‌ای جهت فراهم آوردن شبکه‌های مختلف و مجزا که به کاربران مختلفی تعلق دارند
استفاده شده است. راه‌حل‌های مختلفی برای پیاده‌سازی این شبکه‌ها وجود دارد. در بحث مجازی‌سازی کارکردهای شبکه‌، زیرساخت ارتباطی
مورد نیاز 
برای کارکردهای مجازی از طریق همین شبکه‌های مجازی فراهم آورده می‌شود.
یعنی مسائلی که پیشتر در بحث جایگذاری شبکه‌های مجازی مطرح بود
امروز جزئی از مسائل جایگذاری زنجیره‌های کارکرد سرویس می‌باشند.

\subsection{کارکردهای مجازی شبکه}
یک کارکرد شبکه، یک بلوک عملیاتی در زیرساخت شبکه است که عملکرد رفتاری و رابط‌های ارتباط با خارج خوش تعریف دارد.
مثال‌هایی از کارکردهای شبکه می‌تواند شامل
\lr{DHCP}
یا
\lr{firewall}
و ... باشد.
با این توضیحات کارکرد مجازی شبکه، پیاده‌سازی یک کارکرد شبکه است
که می‌تواند روی منابع مجازی شده اجرا شود.
از هر کارکرد شبکه می‌توان نمونه‌سازی کرده و چند نمونه را در شبکه مستقر ساخت. 
این نمونه‌ها می‌توانند برای سرویس‌دهی به زنجیره‌های مختلف استفاده شوند. از آنجایی که 
هر نمونه توان پردازشی محدودی دارد با افزایش تعداد نمونه‌ها می‌توان توان پردازشی یک کارکرد را نیز افزایش داد.

\subsection{\lr{NFV MANO}}
بر اساس چهارچوب پیشنهادی \lr{ETSI}
وظیفه‌ی \lr{NFV MANO} فراهم آوردن کارکردهای لازم
برای تدارک و فرآیند‌های مشابه مانند تنظیم کردن و ... کارکردهای مجازی شبکه است.
\lr{NFV MANO} شامل هماهنگ کننده و مدیریت کننده چرخه‌ی زندگی
منابع سخت‌افزاری و نرم‌افزاری که مجازی‌سازی زیرساخت را پشتیبانی می‌کنند، است.
هر زنجیره نیاز دارد که حداقل توسط یک \lr{VNFM} مدیریت شود
تا مثلا خطاهای آن را تحت نظر قرار دهد و در صورت نیاز در قسمت دیگری از شبکه استقرار یابد.
مساله‌ی جایگذاری زنجیره‌ها بسیار مورد مطالعه قرار گرفته است، اما در این بین توجه لازم به نیاز این زنجیره‌ها به یک
\lr{VNFM}
صورت نپذیرفته است.~
و
\verb~\chapter{مفاهیم پایه}

\section{مقدمه}

راه‌اندازی و استقرار سرویس در صنعت مخابرات به طور سنتی بر این اساس است که اپراتورهای شبکه سخت‌افزارهای اختصاصی فیزیکی و تجهیزات لازم برای هر کارکرد در سرویس را در زیرساخت خود مستقر کنند.
فراهم کردن نیازمندی‌هایی مانند پایداری و کیفیت بالا منجر به اتکای فراهم‌نندگان سرویس بر سخت‌افزارهای اختصاصی می‌شود. 
این در حالی است که نیازمندی کاربران به سرویس‌های متنوع و عموما با عمرکوتاه و نرخ بالای ترافیک افزایش یافته است.
بنابراین فراهم‌کنندگان سرویس‌ها باید مرتبا و به صورت پیوسته تجهیزات فیزیکی جدید را خریده، انبارداری کرده و مستقر کنند.
تمام این عملیات باعث افزایش هزینه‌های فراهم‌کنندگان سرویس می‌شود.
با افزایش تجهیزات، پیدا کردن فضای فیزیکی برای استقرار تجهیزات جدید به مرور دشوارتر می‌شود.
علاوه بر این باید افزایش هزینه و تاخیر ناشی از آموزش کارکنان برای کار با تجهیزات جدید را نیز در نظر گرفت.
بدتر این که هر چه نوآوری سرویس‌ها و فناوری شتاب بیشتری می‌گیرد، چرخه عمر سخت‌افزارها کوتاه‌تر می‌شود که مانع از ایجاد نوآوری در سرویس‌های شبکه می‌شود.

در روش سنتی استقرار سرویس شبکه، ترافیک کاربر باید از تعدادی کارکرد شبکه به ترتیب معینی عبور کند تا یک مسیر پردازش ترافیک ایجاد شود.
در حال حاضر این کارکردها به صورت سخت‌افزاری به یکدیگر متصل هستند و ترافیک با استفاده از جداول مسیریابی به سمت آن‌ها هدایت می‌شود.
چالش اصلی این روش در این است که استقرار و تغییر ترتیب کارکردها دشوار است.
به عنوان مثال، به مرور زمان با تغییر شرایط شبکه نیازمند تغییر همبندی و یا مکان کارکردها برای سرویس‌دهی بهتر به کاربران هستیم که نیاز به جا‌به‌جایی کارکردها و تغییر جداول مسیریابی دارد.
در روش سنتی این کار سخت و هزینه‌بر است که ممکن است خطاهای بسیاری در آن رخ دهد.
از جنبه‌ی دیگر، تغییر سریع سرویس‌های مورد نظر کاربران نیازمند تغییر سریع در ترتیب کارکردها است که در روش فعلی این تغییرات به سختی صورت گیرد.
بنابراین اپراتورهای شبکه نیاز به شبکه‌های قابل برنامه ریزی و ایجاد زنجیره سرویس کارکردها به صورت پویا پیدا کرده‌اند.

دو فناوری برای پاسخ‌گویی به این چالش‌ها مطرح شده است:

\begin{itemize}
    \item مجازی‌سازی کارکرد شبکه یا \lr{NFV}
    \item زنجیره‌سازی کارکردهای سرویس یا \lr{SFC}
\end{itemize}

با استفاده از مجازی‌سازی کارکردهای شبکه و اجرای آن‌ها بر روی سرورهای استاندارد با توان بالا،
امکان اجرای کارکردها بر روی سخت افزارهای عمومی فراهم شده است تا نیاز به تجهیزات سخت افزاری خاص منظوره کاهش یابد.
از طرف دیگر \lr{SFC} امکان تعریف زنجیره کارکردها را ارائه می‌کند که ایجاد
و انتخاب مسیرهای متفاوت برای پردازش ترافیک به صورت پویا و بدون ایجاد تغییر در زیرساخت فیزیکی را امکان‌پذیر می‌کند
با توجه به این فناوری‌ها، مسائل تحقیقاتی جدیدی مطرح شده‌اند که از مهم‌ترین آن‌ها می‌توان تخصیص منابع بهینه به سرویس درخواستی کاربر را نام برد.

از آنجایی که از مفاهیم این فناور‌ی‌ها برای طراحی و تعریف مساله در این رساله استفاده شده است، نیازمند آشنایی با مفاهیم ابتدایی و اصول اولیه آن‌ها خواهیم بود.

بنابراین در این فصل به صورت خلاصه اجزای این فناوری‌ها را مرور خواهیم کرد و کاربردها، چالش‌ها و مسائل تحقیقاتی که در هر یک از این معماری‌ها وجود دارد را مورد بررسی قرار خواهیم داد.

\section{مجازی‌سازی کارکرد شبکه}

مجازی‌سازی کارکرد شبکه اصل جداسازی کارکرد شبکه به وسیله انتزاع سخت‌افزاری مجازی از سخت افزاری است که بر روی آن اجرا می‌شود.
هدف مجازی‌سازی کارکرد شبکه تغییر روش اپراتورهای شبکه در طراحی شبکه
با تکامل مجازی سازی استاندارد فناوری اطلاعات به منظور تجمیع تجهیزات شبکه
در سرورهای استاندارد، سوییچ‌ها و ذخیره‌سازها با توان بالا است.
یک سرور استاندارد با توان بالا سروری است که توسط اجزای استاندارد شده \lr{IT}،
مانند معماری \lr{x86}، ساخته شده و
در تعداد بالایی، مانند میلیون،
فروخته می‌شود.
ویژگی اصلی این سرورها این است که اجزای آن‌ها به راحتی از فروشندگان مختلف قابل خریداری و
تعویض است.
این تجهیزات می‌توانند در مراکز داده، گره‌های شبکه، یا مکان کاربران انتهایی قرار بگیرند.
این روند در
شکل
\ref{fig.6}
نیز توصیف شده است.

\begin{figure}[!h]
\center\includegraphics[scale=.5]{images/nfv-concept}
\caption{رویکرد \lr{NFV}}\label{fig.6}
\end{figure}

با استفاده از \lr{NFV}، انواع کارکردهای شبکه مانند دیواره آتش و \lr{NAT}
را می‌توان به صورت یک برنامه نرم‌افزاری از فروشندگان مختلف تهیه کرد و
آن‌ها را بر روی سرورهای با توان بالا اجرا کرد که نیاز به نصب تجهیزات خاص منظوره و
جدید را برطرف می‌سازد.

مزایا و اهداف اساسی که \lr{NFV} برای تحقق و دست‍یابی به آن‍ها شکل گرفته است عبارتند از:

\begin{itemize}
    \item
    کاهش هزینه‌های تجهیزات و مصرف انرژی از طریق تجمیع کارکردها بر روی سرورها و در نتیجه کاهش تعداد تجهیزات
    \item
    کاهش نیاز به آموزش کارکنان، افزایش دسترسی پذیری به سخت افزار و کاهش زمان بازیابی از خرابی سخت افزار به علت استفاده از سخت افزارهای استاندارد و عمومی
    \item
    افزایش سرعت عرضه محصول به بازار با کوتاه‌کردن چرخه نوآوری و تولید. در واقع \lr{NFV} به اپراتورهای شبکه کمک می‍کند تا چرخه بلوغ محصول را به اندازه قابل توجهی کاهش دهند.
    \item
    امکان‌پذیر بودن تعریف سرویس مورد نظر بر اساس نوع مشتری یا محل جغرافیایی. مقیاس سرویس‌ها می‍تواند به سرعت، بر اساس نیاز، گسترش یا کاهش یابد.
    \item
    تشویق به ایجاد نوآوری و ارائه سرویس‌های جدید و دریافت جریان‌های درآمدی تازه با سرعت بالا و ریسک پایین.
    \item
    افزایش توانایی  مقابله با خرابی کارکردها، قابلیت به اشتراک گذاری منابع بین کارکردها و پشتیابی از چند مشتری
\end{itemize}

سازمان‌های استانداردگذاری متعددی در استانداردسازی فناوری \lr{NFV} دخیل هستند که شاخص‌ترین آن‌ها موسسه استانداردهای مخابراتي اروپا (\lr{ETSI}) است.
در اواخر سال ۲۰۱۲،
\lr{ETSI NFV ISG}
توسط هفت اپراتور جهانی شبکه به منظور ارتقا ایده مجازی‌سازی کارکرد شبکه تأسیس شد.
\lr{NFV ISG}
تبدیل به یک بستر صنعتی اصلی برای توسعه چارچوب معماری \lr{NFV} و نیازمندی‌های آن شده است و اکنون بیش از ۲۵۰ سازمان با آن همکاری می‌کنند.
اسناد معماری \lr{NFV} به صورت عمومی و رایگان توسط \lr{ETSI NFV ISG} منتشر می‌شود.
ما در این رساله برای توصیف معماری \lr{NFV} از اسناد ارائه شده این سازمان استفاده می‌کنیم.

\section{معماری \lr{NFV}}

در این بخش مؤلفه‌های تشکیل‌دهنده معماری \lr{NFV} شرح داده می‌شوند.
هر یک از اجزای معماری می‌توانند توسط تولیدکنندگان متفاوتی تأمین شوند و به وسیله واسط‌هایی که توسط معماری \lr{NFV}
توصیف شده‌اند با یکدیگر در ارتباط باشند.
بنابراین معماری \lr{NFV} توصیف شده توسط \lr{ETSI} راه‌حلی با قابلیت مشارکت و هماهنگی چندین تولیدکننده مختلف را دارد.
با توجه به استاندارد \lr{ETSI} معماری \lr{NFV}
از سه عنصر کلیدی تشکیل شده است.
زیرساخت مجازی‌سازی کارکردهای شبکه،
کارکردهای مجازی شبکه‌ای و
\lr{NFV MANO}.
این اجزا در شکل \ref{fig.1} نمایش داده شده‌اند.

\begin{figure}[!h]
\center\includegraphics[scale=.5]{images/nfv-arch}
\caption{معماری مجازی‌سازی کارکردهای شبکه
}\label{fig.1}
\end{figure}

\begin{itemize}
    \item
    \lr{NFVI}: شامل منابع سخت افزاری و نرم‌افزاری لازم برای اجرای \lr{VNF}‌ها
    \item
    \lr{Service}: شامل \lr{VNF}‌ها که کارکردهای شبکه را پیاده‌سازی کرده‌اند، \lr{EMS} برای مدیریت \lr{VNF}‌ها و \lr{OSS/BSS} برای ارتباط با سیستم‌های مدیریت سنتی
    \item
    \lr{MANO}: که وظیفه مدیریت و هماهنگی سرویس‌ها و تخصیص منابع را برعهده دارد و از سه بخش \lr{NFVO}، \lr{VIM} و \lr{VNFM} تشکیل شده است.
\end{itemize}

\subsection{زیرساخت مجازی‌سازی کارکردهای شبکه یا \lr{NFVI}}
زیرساخت مجازی‌سازی کارکردهای شبکه ترکیبی از منابع نرم‌افزاری و سخت‌افزاری است
که محیطی برای نصب
کارکردهای مجازی شبکه فراهم می‌آورد.
منابع سخت‌افزاری شامل منابع محاسباتی،
ذخیره‌سازها و شبکه
(شامل لینک‌ها و گره‌ها)
هستند
که پردازش، ذخیره‌سازی و ارتباط را
برای کارکردهای مجازی شبکه فراهم می‌آورند.
منابع مجازی انتزاعی از منابع شبکه‌ای، پردازشی و ذخیر‌ه‌سازی هستند.
به وسیله انتزاع از طریق لایه‌ی مجازی‌سازی (بر پایه‌ی \lr{hypervisor})
منابع سخت افزاری در اختیار کارکردهای مجازی
قرار می‌گیرند که این منابع شامل منابع محاسباتی، شبکه‌ای و ذخیره‌سازی می‌باشند.

در مراکز داده‌ای ممکن است منابع پردازشی و ذخیره‌سازی تحت عنوان یک یا چند
ماشین مجازی نمایش داده شوند در حالی که شبکه‌های مجازی از لینک‌ها و گره‌های مجازی تشکیل می‌شوند.
شبکه‌های مجازی پیش از بحث مجازی‌سازی کارکردهای شبکه مدنظر بوده‌اند و روی آن‌ها کار شده است.
در واقع از شبکه‌های مجازی در مراکز داده‌ای جهت فراهم آوردن شبکه‌های مختلف و مجزا که به کاربران مختلفی تعلق دارند
استفاده شده است. راه‌حل‌های مختلفی برای پیاده‌سازی این شبکه‌ها وجود دارد. در بحث مجازی‌سازی کارکردهای شبکه‌، زیرساخت ارتباطی
مورد نیاز 
برای کارکردهای مجازی از طریق همین شبکه‌های مجازی فراهم آورده می‌شود.
یعنی مسائلی که پیش‌تر در بحث جایگذاری شبکه‌های مجازی مطرح بود
امروز جزئی از مسائل جایگذاری زنجیره‌های کارکرد سرویس می‌باشند.

\subsection{کارکردهای مجازی شبکه}
یک کارکرد شبکه، یک بلوک عملیاتی در زیرساخت شبکه است که عملکرد رفتاری و رابط‌های ارتباط با خارج خوش تعریف دارد.
مثال‌هایی از کارکردهای شبکه می‌تواند شامل
\lr{DHCP}
یا
\lr{firewall}
و ... باشد.
با این توضیحات، کارکرد مجازی شبکه، پیاده‌سازی یک کارکرد شبکه است
که می‌تواند روی منابع مجازی شده اجرا شود.
از هر کارکرد شبکه می‌توان نمونه‌سازی کرده و چند نمونه را در شبکه مستقر ساخت. 
این نمونه‌ها می‌توانند برای سرویس‌دهی به زنجیره‌های مختلف استفاده شوند. از آنجایی که 
هر نمونه توان پردازشی محدودی دارد با افزایش تعداد نمونه‌ها می‌توان توان پردازشی یک کارکرد را نیز افزایش داد.

\subsection{\lr{EM}}
این مولفه کارکردهای \lr{FCAPS}\footnote{\lr{Fault,Config,Accounting,Performance,Security}} را برای \lr{VNF} ها انجام می‌دهد که شامل مدیریت خطا، پیکربندی، امنیت، حسابداری و کارایی برای کارکردی است که \lr{VNF} ارائه می‌دهد. این مولفه ممکن است آگاه از مجازی بودن کارکرد، باشد و با همکاری \lr{VNFM} عملکردهای خودش را انجام بدهد.

\subsection{\lr{OSS/BSS}}
این مولفه، ترکیبی از سایر بخش‌های عملکردهای اپراتور است که در چارچوب معماری \lr{NFV} ارائه شده از طرف \lr{ETSI} قرار نمی‌گیرند. به عنوان مثال می‌تواند شامل مدیریت سیستم‌های \lr{Legacy} باشد.

\subsection{\lr{NFV MANO}}
بر اساس چهارچوب پیشنهادی \lr{ETSI}
وظیفه‌ی \lr{NFV MANO} فراهم آوردن کارکردهای لازم
برای تدارک فرآیند‌های مشابه مانند تنظیم کردن و ... کارکردهای مجازی شبکه است.
\lr{NFV MANO} شامل هماهنگ‌کننده و مدیریت‌کننده چرخه‌ی زندگی
منابع سخت‌افزاری و نرم‌افزاری که مجازی‌سازی زیرساخت را پشتیبانی می‌کنند، است.
هر زنجیره نیاز دارد که حداقل توسط یک \lr{VNFM} مدیریت شود
تا مثلا خطاهای آن را تحت نظر قرار دهد و در صورت نیاز در قسمت دیگری از شبکه استقرار یابد.
مساله‌ی جایگذاری زنجیره‌ها بسیار مورد مطالعه قرار گرفته است، اما در این بین توجه لازم به نیاز این زنجیره‌ها به یک
\lr{VNFM}
صورت نپذیرفته است.

\lr{VNFO} بخشی از مولفه \lr{MANO} است که وظیفه تخصیص منابع به سرویس را برعهده دارد.
یکی از مهم ترین اجزای سرویس گراف \lr{VNF-FG} است که بیانگر \lr{VNF} های سرویس و ارتباطات بین آن‌ها است.
وظیفه اصلی مولفه \lr{NFVO} ایجاد نمونه از سرویس و مدیریت چرخه حیات آن است.
ایجاد نمونه از سرویس شامل ایجاد نمونه از \lr{VNF}‌های تشکیل‌دهنده آن و ایجاد ارتباط بین نمونه‌ها است.
سایر وظایف مولفه \lr{VNFO} به شرح زیر است:
\begin{itemize}
    \item مدیریت چرخه حیات سرویس شبکه
    \item مدیریت و هماهنگی منابع مورد نیاز \lr{NFVI} بین چندین \lr{VIM}
    \item مدیریت منابع و ایجاد نمونه از \lr{VNF}‌ها با هماهنگی \lr{VNFM}
    \item مدیریت منابع و نمونه‌سازی \lr{VNFM}
    \item مدیریت همبندی نمونه ساخته شده از سرویس شبکه مانند ایجاد، حذف و به روز رسانی \lr{VNF-FG}
    \item مدیریت قالب‌های استقرار سرویس شبکه و \lr{VNF‌}ها مانند اعتبارسنجی قالب‌ها
\end{itemize}
همچنین این مولفه مسئولیت مشخص کردن مکان فیزیکی نمونه های ایجاد شده از \lr{VNF}ها را برعهده دارد.

مولفه‌ی \lr{VNFM} مسئولیت مدیریت چرخه حیات نمونه‌های ایجاد شده از\lr{VNF}‌ها را برعهده دارد.
بنابراین فرض می‌شود هر نمونه ایجاد شده از هر \lr{VNF}، به یک \lr{VNFM} اختصاص یافته است.
مهم‌ترین وظایف این مولفه به شرح زیر است:
\begin{itemize}
    \item پیکربندی و نمونه‌سازی از \lr{VNF}‌ها
    \item گسترش و یا کاهش مقیاس‌پذیری افقی یا عمودی برای نمونه‌های ایجاد شده از \lr{VNF}ها
    \item مدیریت نمونه‌های ایجاد شده شامل  تغییرات، به روزرسانی برنامه‌ها و خاتمه دادن به نمونه‌ها
\end{itemize}
مولفه \lr{VNFM} با استفاده از \lr{VNFD}، از \lr{VNF} نمونه ایجاد می‌کند و
مدیریت چرخه حیات آن را انجام می‌دهد.
منابع پردازشی، محاسباتی و شبکه مطابق با توصیفات گفته شده در \lr{VNFD} به نمونه‌های آن اختصاص می‌یابند.

مولفه \lr{VIM} مسئولیت کنترل و مدیریت منابع محاسباتی، ذخیره‌سازی و شبکه‌ای،
معمولا در حوزه یک اپراتور، را برعهده دارد.
مهم ترین وظایف این مولفه عبارتند از:

\begin{itemize}
    \item
    هماهنگی تخصیص، ارتقا و آزادسازی منابع \lr{NFVI} شامل بهینه‌‌سازی استفاده از منابع و مدیریت انجمنی منابع مجازی و فیزیکی.
    بنابراین \lr{VIM} اطلاعات \lr{Inventory} تخصیص منابع مجازی به منابع فیزیکی را نگهداری می‌کند.
    \item
    پشتیبانی از مدیریت \lr{VNF-FG} به وسیله ایجاد و نگهداری لینک‌های مجازی،
    شبکه‌های مجازی زیرشبکه‌ها و پورت‌ها
    \item
    مدیریت اطلاعات \lr{Inventory} سخت افزارها و نرم افزارها و کشف قابلیت‌ها و ویژگی‌های آن‌ها
    \item
    مدیریت ظرفیت منابع مجازی مانند نسبت منابع مجازی به حقیقی
    \item
    مدیریت تصویرهای نرم‌افزاری، مانند تصاویر \lr{VNF}ها، که ممکن است توسط سایر مولفه‌های \lr{MANO} هم مورد استفاده قرار بگیرند.
    \item
    جمع‌آوری اطلاعات کارایی و خطا از منابع سخت‌افزاری و نرم‌افزاری
\end{itemize}

\section{\lr{VNFD}}

هر \lr{VNF} توسط توصیف‌گر مربوط به آن که نیازمندی‌های استقرار و رفتاری آن را مشخص می‌کند توصیف می‌شود.
مولفه \lr{VNFM} از \lr{VNFD} در فرایند نمونه‌سازی \lr{VNF}‌ها و مدیریت چرخه حیات آن‌ها استفاده می‌کند.
همچنین این اطلاعات توسط مولفه \lr{VNFO} برای ایجاد مدیریت و هماهنگی سرویس شبکه نیز استفاده می‌شود.
\lr{VNFD} شامل شاخص‌های کارایی است که می‌تواند توسط \lr{VNFM} نیز مورد استفاده قرار بگیرد.
در \lr{VNFD} ارتباطات داخلی و واسط‌ها نیز توصیف می‌شوند که
برای ایجاد لینک‌های مجازی بین مولفه‌های \lr{VNFC} و یا ارتباط بین \lr{VNF} با سایر \lr{VNF}‌ها مورد استفاده قرار می‌گیرد.
\lr{VNFD} همچنین شامل قالب‌های استقرار \lr{VNF} به همراه نیازمندی منابع برای هر قالب است.

\section{سرویس شبکه و اجزای آن}
یک سرویس شبکه را می‌توان به صورت یک گراف جلورانی از کارکردهای شبکه
\lr{(NF-FG)}
که به یکدیگر از طریق زیرساخت شبکه متصل هستند دید.
کارکردهای شبکه می‌تواند توسط یک یا چند اپراتور ارائه شده باشند.
نقاط انتهایی سرویس را می‌توان به صورت گره‌های گراف و
ارتباطات میان کارکردها را توسط لینک‌های گراف مدل سازی کرد
که لینک‌های گراف می‌توانند، یک طرفه یا دو طرفه، چند پخشی یا همه پخشی باشند.
مثالی از یک سرویس شبکه در شکل
\ref{fig.18}
نمایش داده شده است.
در این شکل، یک سرویس شبکه انتها به انتها از طریق نقاط انتهایی \lr{A} و \lr{B} ایجاد شده که شامل یک \lr{NF-FG} داخلی است.
این \lr{NF-FG} خود شامل سه کارکرد شبکه است که به یکدیگر متصل هستند.


\begin{figure}[h!]
\center\includegraphics[scale=.5]{images/network-service}
\caption{یک سرویس شبکه شامل یک گراف جلورانی}
\label{fig.18}
\end{figure}

در صورتی که در یک  \lr{NF-FG} حداقل یکی از این کارکردها \lr{VNF} باشد، به آن \lr{VNF-FG} گفته می شود.
در صورتی که فرض کنیم همه \lr{NF} های شکل
\ref{fig.18}
، \lr{VNF} هستند می‌توان آن را مطابق شکل
\ref{fig.19}
نمایش داد.
در این شکل \lr{NF2} خود توسط سه \lr{VNF} پیاده‌سازی شده است.

\begin{figure}[h!]
\center\includegraphics[scale=.5]{images/vnf-fg}
\caption{گراف \lr{VNF-FG} متناظر با شکل \ref{fig.18}}
\label{fig.19}
\end{figure}

مشخص است که گراف \lr{VNF-FG} صرفا ارتباطات بین \lr{VNF}‌ها را مشخص می‌کند ولی ترتیب عبور ترافیک از کارکردها را بیان نمی‌کند.
ترتیب عبور ترافیک از کارکردها توسط \lr{NFP} بیان می‌شود که یکی از اجزای \lr{VNF-FG} است و هر \lr{VNF-FG} باید حداقل یک \lr{NFP} داشته باشد.

\section{موارد کاربرد}
در این بخش موارد کاربرد مهم معماری \lr{NFV} را شرح می‌دهیم.
یکی از مهم‌ترین موارد کاربرد ذکر شده برای \lr{NFV}،  مجازی‌سازی تجهیزات \lr{CPE} است.
عموما تجهیزاتی در مکان کاربران برای اتصال به اینترنت نگهداری می‌شود که شامل دیواره آتش، \lr{NAT}، مسیریاب و سوییچ است.
در این حالت تنظیمات تجهیزات باید در مکان کاربر صورت بگیرد که هزینه بالایی دارد.
با استفاده از مجازی‌سازی این کارکردها و نگه‌داری آن در سمت \lr{ISP}، می‌توان هزینه تجهیزات و زمان نگهداری را کاهش داد.
این مورد کاربرد در شکل \ref{fig.20} نمایش داده شده است.

\begin{figure}[h!]
\center\includegraphics[scale=.5]{images/cpe}
\caption{مجازی‌سازی \lr{CPE}}
\label{fig.20}
\end{figure}

به عنوان مثالی دیگر از موارد کاربرد می‌توان مجازی‌سازی کارکردها در زیرساخت \lr{LTE} را در نظر گرفت.
در شکل \ref{fig.21} شرایط قبل و بعد از مجازی سازی
\lr{P-GW}، \lr{S-GW}، \lr{PCRF} و \lr{MMF}
نمایش داده شده است.
با مجازی‌سازی این کارکردها علاوه بر استفاده بهتر از منابع، می توان تعداد نمونه‌های آن ها را مطابق با تعداد کاربران بدون تغییر در زیرساخت افزایش و یا کاهش داد.

\begin{figure}[h!]
\center\includegraphics[scale=.5]{images/lte}
\caption{مجازی سازی زیرساخت \lr{LTE}}
\label{fig.21}
\end{figure}

\section{زنجیره‌سازی کارکرد سرویس}

زنجیره‌سازی کارکردها ایده جدیدی نیست.
در حال حاضر اپراتورها برای ارائه سرویس یک زنجیره از کارکردها ایجاد می‌کنند که ترافیک کاربر باید از کارکردها با یک ترتیب مشخص عبور کند.
اگرچه همانطور که بیان شد در صورت تغییر در ترتیب کارکردها و یا زنجیره‌ها و یا ایجاد سرویس‌های جدید، نیازمند تغییر مکان فیزیکی کارکردها خواهیم بود
که کاری سخت است و خالی از اشکال نیست.
معماری زنجیره‌سازی کارکردهای سرویس، اپراتورهای شبکه را قادر می‌سازد که سرویس‌های جدید را به صورت نرم افزاری و پویا و بدون اینکه در سطح سخت افزار تغییری ایجاد کنند، ارائه کنند.
در این راستا \lr{IETF} در اسناد متعددی به شرح معماری و اجزای آن پرداخته است.
در این بخش به شرح معماری زنجیره‌سازی کارکرد سرویس می‌پردازیم و بخش‌های اصلی آن را بیان می‌کنیم.

\subsection{اجزای معماری \lr{SFC}}

در این \lr{RFC} یک سرویس شبکه که توسط اپراتور ارائه می‌شود و
از طریق یک یا چند کارکرد سرویس تحویل می‌شود،
تعریف شده است.
یک کارکرد سرویس، رفتار خاصی (به غیر از جلورانی) با بسته را انجام می‌دهد
و می‌تواند در هر یک از لایه‌های مدل \lr{OSI} فعالیت کند.
به یک شبکه یا بخشی از آن که در آن \lr{SFC}  پیاده‌سازی شده است یک دامنه \lr{SFC} گفته می‌شود.
در یک دامنه \lr{SFC}، معماری \lr{SFC} مطابق با شکل \ref{fig.22} پیاده‌سازی می‌شود.
معماری \lr{SFC} توسط \lr{RFC 7665} تعریف شده است.

\begin{figure}[h!]
\center\includegraphics[scale=.5]{images/sfc}
\caption{معماری \lr{SFC}}
\label{fig.22}
\end{figure}

به صورت خلاصه اجزای اصلی این معماری عبارتند از:
\begin{itemize}
    \item زنجیره کارکرد سرویس: یا به صورت خلاصه زنجیره کارکرد یک مجموعه مرتب از کارکردهای سرویس انتزاعی و محدودیت‌های ترتیبی که باید به بسته‌ها، فریم‌ها و یا جریان‌های دریافتی به عنوان نتیجه دسته بندی اعمال شود.
    \item دسته‌بند: وظیفه دسته‌بندی و انتخاب زنجیره کارکرد برای ترافیک ورودی بر اساس قوانین از پیش تعیین شده را برعهده دارد.
    \item \lr{SFF}: وظیفه جلورانی و هدایت ترافیک در دامنه \lr{SFC} را برعهده دارد.
    \item کارکرد سرویس (\lr{SF}): یک کارکرد انتزاعی که مسئول رفتار خاصی با بسته به جز جلورانی است.
    \item \lr{SF Proxy}: وظیفه ارتباط با کارکردهای غیر آگاه از کپسول بندی \lr{SFC} را برعهده دارد.
    \item صفحه کنترل: وظیفه کنترل و نظارت بر زنجیره‌ها و ایجاد قوانین دسته‌بندی بر روی دسته‌بند را برعهده دارد.
\end{itemize}
همه این اعضا به صورت منطقی هستند و می‌توانند در شبکه به صورت فیزیکی و یا مجازی در یک یا چندین دستگاه فیزیکی به صورت مشترک با یکدیگر وجود داشته باشند.

\section{جمع‌بندی}
در این بخش معماری‌های \lr{NFV} و \lr{SFC} به صورت کامل شرح داده شد و اجزای آن‌ها بررسی شد.
همانگونه که بیان شد، معماری \lr{NFV} بر مجازی‌سازی کارکردها تمرکز دارد.
یک سرویس در معماری \lr{NFV} با استفاده از \lr{NSD} توصیف می‌شود که
شامل \lr{VNF-FG}، \lr{VNF}‌ها و لینک‌های توصیف کننده ارتباطات بین \lr{VNF}‌ها است.
معماری \lr{SFC} به ایجاد زنجیره پویا از کارکردها تمرکز دارد.
در معماری \lr{VNF} سرویس‌های مدیریتی و نظارتی مانند \lr{VNFM}، \lr{VNFO} و ... نیز تعریف می‌شوند که
وظیفه‌ی نظارت و مدیریت چرخه‌ی حیات در این معماری را دارا می‌باشد و تمرکز اصلی این پژوهش بر در نظر گرفتن اهمیت این سرویس‌ها می‌باشد.

یک زنجیره کارکرد توسط یک گراف \lr{SFC} توصیف می شود که
به صورت مجموعه مرتب از کارکردها که ترافیک باید با ترتیب مشخصی از آن‌ها عبور کند توصیف می‌‌شود.
این معماری تاکیدی بر مجازی‌سازی کارکردها ندارد.
همچنین برای مسیریابی ترافیک در این معماری نیز می توان از سرآیند \lr{NSH} استفاده کرد.
همانگونه که بیان شد در حقیقت این دو معماری مکمل یکدیگر هستند و
یک گراف \lr{SFC} را می‌توان توسط یک \lr{VNF-FG} معادل نمایش داد.~
را در فایل 
\verb~AUTthesis.tex~،
غیرفعال%
\RTLfootnote{
برای غیرفعال کردن یک دستور، کافی است پشت آن، یک علامت
\%
 بگذارید.
}
 کنید. زیرا در غیر این صورت، ابتدا مطالب فصل ۱ و ۲ پردازش شده (که به درد ما نمی‌خورد؛ چون ما می‌خواهیم خروجی فصل ۳ را ببینیم) و سپس مطالب فصل ۳ پردازش می‌شود و این کار باعث طولانی شدن زمان اجرا می‌شود. زیرا هر چقدر حجم فایل اجرا شده، بیشتر باشد، زمان بیشتری هم برای اجرای آن، صرف می‌شود.

\subsection{مراجع}
برای وارد کردن مراجع به فصل 2
مراجعه کنید.
\subsection{واژه‌نامه فارسی به انگلیسی و برعکس}
برای وارد کردن واژه‌نامه فارسی به انگلیسی و برعکس، بهتر است مانند روش بکار رفته در فایل‌های 
\verb;dicfa2en;
و
\verb;dicen2fa;
عمل کنید.

\section{اگر سوالی داشتم، از کی بپرسم؟}
برای پرسیدن سوال‌های خود در مورد حروف‌چینی با زی‌پرشین،  می‌توانید به
 \href{http://forum.parsilatex.com}{تالار گفتگوی پارسی‌لاتک}%
\LTRfootnote{\url{http://www.forum.parsilatex.com}}
مراجعه کنید. شما هم می‌توانید روزی به سوال‌های دیگران در این تالار، جواب بدهید.
