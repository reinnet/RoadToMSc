\chapter{مقدمه}


بیشتر سرویس‌های شبکه بر روی سخت افزارهای اختصاصی به نام
\lr{middle box}
ساخته می‌شوند.
تنوع و تعداد رو به افزایش سرویس‌های جدیدی که توسط کاربران تقاضا می‌گردد
باعث هزینه‌های زیاد برای خرید و نگهداری
\lr{middle box}‌ها
توسط اپراتورها شده است.
به تازگی فراهم آورندگان شبکه\footnote{\lr{Network Provider}}
شروع به حرکت به سوی مجازی‌سازی و نرم‌افزاری کردن بسترهای شکبه کرده‌اند،
به این ترتیب آن‌ها قادر خواهند بود
سرویس‌های نوآورانه‌ای به کاربران ارائه بدهند.

مجازی‌سازی توابع شبکه‌ راهکاری است که برای همین منظور پیشنهاد شده است.
مجازی‌سازی توابع شبکه‌ در واقع راه‌حل‌های مشخصی را برای چالش‌های جای‌گذاری،
زنجیره‌سازی و هماهنگی سرویس‌های شبکه فراهم می‌آورد.

ایده‌ی اصلی مجازی‌سازی توابع شبکه جداسازی تجهیزات فیزیکی شبکه از کارکردهایی می‌باشد که
بر روی آن‌ها اجرا می‌شوند.\cite{Mijumbi2016}