%%% !TeX document-id = {7f856805-4d41-4d13-829b-b08b1a8ea49a}
 %%%  کلاس AUTthesis، نسخه اردیبهشت 1396
%%%   دانشگاه صنعتی امیرکبیر                 http://www.aut.ac.ir
%%%  تالار گفتگوی پارسی‌لاتک،       http://forum.parsilatex.com
%%%   آپدیت شده در آبان 95
%%      پشتیبانی و راهنمایی          badali_farhad@yahoo.com
%%%
%%%   بازبینی و اصلاح شده در اردیبهشت ماه 1396
%%%  Tested via TeXstudio in TeXlive 2014,‌‌ 2015 & 2016.
%%%

%-----------------------------------------------------------------------------------------------------
%        روش اجرا.: 2 بار F1 ، 2 بار  F11(به منظور تولید مراجع) ، دوبار Ctrl+Alt+I (به منظور تولید نمایه) و دو بار F1 -------> مشاهده Pdf
%%%%%%%%%%%%%%%%%%%%%%%%%%%%%%%%%%%%%%%%%%%%%%%%%%%%%%
%   TeXstudio as your IDE
%%  برای compile در TeXstudio تنها کافی است منوی Options->Configure TeXstudio را زده و در پنجره Configure TeXstudio در بخش Build گزینه Default Compiler را به XeLaTeX تغییر دهید. سند شما به راحتی compile خواهد شد.
%   F1 & F5 : Build & view
%   F6      : Compile
%   F7      : View
%   --------------
%   شما می‌توانید این نرم‌افزار را از سایت زیر دریافت کنید. برای استفاده از این نرم‌افزار باید پیش از نصب آن TeXlive را نصب کرده باشید.
%   TeXstudio.org
%   این سند با نسخه‌های 2014، 2015 و 2016 نرم‌افزار TeXlive در محیط TeXstudio آزمایش شده است.
%   نکته: در هنگام compile با نرم‌افزار TeXstudio فایل اصلی حتما باید باز باشد ولی نیازی نیست که روی tab آن باشید.
%	همچنین در پوشه LookAtMe فایلهای png برای راهنمایی وجود دارند.
%	برای ساخت میانبر نیم فاصله نیز به فایلهای پوشه LookAtMe را نگاه کنید.
%   باآرزوی پیروزی برای شما
%   آریا
%%%%%%%%%%%%%%%%%%%%%%%%%%%%%%%%%%%%%%%%%%%%%%%%%%%%%%
%        اگر قصد نوشتن رساله دکتری را دارید، در خط زیر به جای msc،
%      کلمه phd را قرار دهید. کلیه تنظیمات لازم، به طور خودکار، اعمال می‌شود.
%%% !TEX TS-program = XeLaTeX
\documentclass[oneside,msc]{AUTthesis}
%       فایل commands.tex را حتماً به دقت مطالعه کنید؛ چون دستورات مربوط به فراخوانی بسته زی‌پرشین 
%       و دیگر بسته‌ها و ... در این فایل قرار دارد و بهتر است که با نحوه استفاده از آنها آشنا شوید. توجه شود برای نسخه نهایی پایان‌نامه حتماً hyperref را 
%        غیرفعال کنید.


\input{commands}

\begin{document}
\baselineskip=.75cm
%% -!TEX root = AUTthesis.tex
% در این فایل، عنوان پایان‌نامه، مشخصات خود، متن تقدیمی‌، ستایش، سپاس‌گزاری و چکیده پایان‌نامه را به فارسی، وارد کنید.
% توجه داشته باشید که جدول حاوی مشخصات پروژه/پایان‌نامه/رساله و همچنین، مشخصات داخل آن، به طور خودکار، درج می‌شود.
%%%%%%%%%%%%%%%%%%%%%%%%%%%%%%%%%%%%
% دانشکده، آموزشکده و یا پژوهشکده  خود را وارد کنید
\faculty{دانشکده مهندسی کامپیوتر و فناوری اطلاعات}
% گرایش و گروه آموزشی خود را وارد کنید
\department{گرایش شبکه‌های کامپیوتری}
% عنوان پایان‌نامه را وارد کنید
\fatitle{زنجیره‌سازی کارکردهای مجازی سرویس شبکه با در نظر گرفتن محدودیت منابع مدیریتی}
% نام استاد(ان) راهنما را وارد کنید
\firstsupervisor{دکتر بهادر بخشی}
%\secondsupervisor{استاد راهنمای دوم}
% نام استاد(دان) مشاور را وارد کنید. چنانچه استاد مشاور ندارید، دستور پایین را غیرفعال کنید.
% \firstadvisor{نام کامل استاد مشاور}
%\secondadvisor{استاد مشاور دوم}
% نام نویسنده را وارد کنید
\name{پرهام}
% نام خانوادگی نویسنده را وارد کنید
\surname{الوانی}
%%%%%%%%%%%%%%%%%%%%%%%%%%%%%%%%%%
\thesisdate{شهریور ۱۳۹۸}

% چکیده پایان‌نامه را وارد کنید
\fa-abstract{
    در روش سنتی استقرار سرویس شبکه، ترافیک کاربر باید از تعدادی کارکرد شبکه به ترتیب معینی عبور کند تا یک مسیر پردازش ترافیک ایجاد شود.
    در حال حاضر این کارکردها به صورت سخت‌افزاری به یکدیگر متصل هستند و ترافیک با استفاده از جداول مسیریابی به سمت آن‌ها هدایت می‌شود.
    چالش اصلی این روش در این است که استقرار و تغییر ترتیب کارکردها دشوار است.
    دو فناوری برای پاسخ گویی به این چالش ها مطرح شد:
    مجازی‌‍سازی کارکرد شبکه (\lr{NFV}) و زنجیره‌سازی کارکرد سرویس (\lr{SFC}).
    با استفاده از مجازی‌سازی کارکردهای شبکه و اجرای آن‌ها بر روی سرورهای استاندارد با توان بالا، امکان اجرای کارکردها بر روی سخت افزارهای عمومی را فراهم کرده است
    تا نیاز به تجهیزات سخت افزاری خاص منظوره کاهش یابد.
    از طرف دیگر \lr{SFC} امکان تعریف زنجیره کارکردها را ارائه می‌کند
    که ایجاد و انتخاب مسیرهای متفاوت برای پردازش ترافیک به صورت پویا و بدون ایجاد تغییر در زیرساخت فیزیکی را امکان‌پذیر می‌کند.
    با توجه به این فناوری‌ها، مسائل تحقیقاتی جدیدی مطرح شدند که از مهم‌ترین آن‌ها می توان تخصیص منابع بهینه به سرویس درخواستی کاربر را نام برد.
    یکی از چالش‌های مهم در زنجیره‌سازی کارکرد سرویس چگونگی جایگذاری کارکرد‌ها در شبکه زیرساخت می‌باشد که تا به حال پژوهش‌های زیادی در این حوزه انجام شده است.
    یکی دیگر مسائلی که در معماری \lr{NFV} مطرح است چگونگی مدیریت و مانیتورینگ کارکردهای مجازی می‌باشد. تا به حال این دو مساله در کنار یکدیگر مورد مطالعه قرار نگرفته‌اند
    و این در حالی است که برای ارائه سرویس‌هایی با کیفیت مناسب نیاز است که مدیریت و مانیتورینگ بر روی آن‌ها صورت بگیرد.
    در این رساله ما به بررسی همین مساله می‌پردازیم.
    در اولین گام مساله فوق به صورت خطی صحیح فرمول‌بندی شده و در چهارچوب \lr{CPLEX} پیاده‌سازی می‌شود.
    از آنجایی که این مساله \lr{NP-Hard} می‌باشد نیاز است برای حل آن در زمان مناسب از یک الگوریتم مکاشفه‌ای با پیچیدگی چند جمله‌ای استفاده شود.
    این رساله الگوریتمی با زمان چند جمله‌ای برای این مساله پییشنهاد می‌دهد و در نهایت آن را با مساله‌ی بهینه مقایسه می‌کند.  
}


% کلمات کلیدی پایان‌نامه را وارد کنید
\keywords{مجازی سازی کارکردهای شبکه، زنجیره‌سازی کارکردهای مجازی سرویس شبکه،بهینه‌سازی، بهینه‌سازی خطی صحیح}



\AUTtitle
%%%%%%%%%%%%%%%%%%%%%%%%%%%%%%%%%%
\vspace*{7cm}
\thispagestyle{empty}
\begin{center}
\includegraphics[height=5cm,width=12cm]{besm}
\end{center}
% تاییدیه دفاع
\newpage
\thispagestyle{empty}
%\fontsize{18pt}{19pt}\selectfont

\section*{صفحه فرم ارزیابی و تصویب پایان نامه- فرم تأیید اعضاء كميته دفاع}

\fontsize{12pt}{14pt}\selectfont
\renewcommand{\baselinestretch}{1.5}
\vspace*{1cm}
   در این صفحه فرم دفاع یا تایید و تصویب پایان نامه موسوم به فرم کمیته دفاع- موجود در پرونده آموزشی- را قرار دهید.
\vspace*{1cm}


\subsection*{نکات مهم:}
 
\begin{itemize}
\item
	نگارش پایان نامه/رساله باید به
	{\color{red}
		زبان فارسی
	}
	و بر اساس آخرین نسخه دستورالعمل و راهنمای تدوین پایان نامه های دانشگاه صنعتی امیرکبیر باشد.(دستورالعمل و راهنمای حاضر)
\item رنگ جلد پایان نامه/رساله چاپي كارشناسي، كارشناسي ارشد و دكترا  بايد به ترتيب مشكي، طوسي و سفيد رنگ باشد.  
\item چاپ و صحافی پایان نامه/رساله بصورت
{\color{red}
	پشت و رو(دورو)
}
بلامانع است و انجام آن توصيه مي شود. 
\end{itemize}
%%%%%%%%%%%%%%%%%%%%%%%%%%%%%%%%%%%%%%%%%%%%%%%%%%%%%%%%%%%%%%%%%%%%%%%%%%%%%%%%%%%%%%%%%%%%%%%%%%
%%%%%%%%%%%%%%%%%%%%%%%%%%%%%%%%%%%%%%%%%%%%%%%%%%%%%%%%%%%%%%%%%%%%%%%%%%%%%%%%%%%%%%%%%%%%%%%%%%
\newpage
\thispagestyle{empty}
\begin{picture}(50,50)
  \put(10,0){\includegraphics[scale=.4]{fa-logo}}
  \put(4.5,-13){\footnotesize{دانشگاه صنعتی امیرکبیر}}
  \put(10.5,-27){\footnotesize{(پلی‌تکنیک تهران)}}
  \put(170,30){\bf{به نام خدا}}
  \put(140,-5){\Large\bf{تعهدنامه اصالت اثر}}
  \put(300,0){تاریخ: \datethesis}
\end{picture}

\vspace*{2.5cm}

اينجانب {\bf{\fname\lname}} متعهد می‌شوم که مطالب مندرج در این پایان‌نامه حاصل کار پژوهشی اینجانب تحت نظارت و راهنمایی اساتید دانشگاه صنعتی امیرکبیر بوده و به دستاوردهای دیگران که در این پژوهش از آنها استفاده شده است مطابق مقررات و روال متعارف ارجاع و در فهرست منابع و مآخذ ذکر گردیده است. این پایان‌نامه قبلاً برای احراز هیچ مدرک هم‌سطح یا بالاتر ارائه نگردیده است.

در صورت اثبات تخلف در هر زمان، مدرک تحصیلی صادر شده توسط دانشگاه از درجه اعتبار ساقط بوده و دانشگاه حق پیگیری قانونی خواهد داشت.


کلیه نتایج و حقوق حاصل از این پایان‌نامه متعلق به دانشگاه صنعتی امیرکبیر می‌باشد. هرگونه استفاده از نتایج علمی و عملی، واگذاری اطلاعات به دیگران یا چاپ و تکثیر، نسخه‌برداری، ترجمه و اقتباس از این پایان نامه بدون موافقت کتبی دانشگاه صنعتی امیرکبیر ممنوع است. 
نقل مطالب با ذکر مآخذ بلامانع است.\\
\vspace{2.5cm}


{\centerline {\bf{\fname\lname}}}
\vspace*{.2cm}
{\centerline{امضا}}
%%%%%%%%%%%%%%%%%%%%%%%%%%%%%%%%%
% چنانچه مایل به چاپ صفحات «تقدیم»، «نیایش» و «سپاس‌گزاری» در خروجی نیستید، خط‌های زیر را با گذاشتن ٪  در ابتدای آنها غیرفعال کنید.
% پایان‌نامه خود را تقدیم کنید
% نیایش خود را در فایل زیر بنویسید.
\begin{acknowledgementpage}

\vspace{1.5cm}

{\nastaliq
{
    این رساله هرچند کوچک را تقدیم میکنم به:

    \begin{itemize}
        \item دکتر بخشی که ۶ سال شاگردی و همکاری با ایشان برای من مایه افتخار بود.
        \item دوستانم در دانشگاه امیرکبیر که امروز بسیاری‌شان از پیش ما رفته‌اند اما خاطرشان همواره در یادم می‌ماند.
        \item همکارانم در تیم چارلی که نور جدیدی به زندگی من دادند.
    \end{itemize}
}}\end{acknowledgementpage}
\newpage
%سپاسگزاری را در فایل زیر بنویسید.
%%%%%%%%%%%%%%%%%%%%%%%%%%%%%%%%%%%%
\newpage\thispagestyle{empty}
% سپاس‌گزاری
{\nastaliq
سپاس‌گزاری
}
\\[2cm]
در اینجا لازم می‌دانم از راهنمایی‌ها و مساعدت‌های استاد عزیز و گرانقدرم
جناب آقای دکتر بخشی صمیمانه قدردانی و سپاس‌گزاری نمایم.
در ادامه از دوست خوبم بهروز فرکیانی که همواره من را راهنمایی کرده‌
و
از پدر و مادرم که همواره من را حمایت کرده‌اند تشکر می‌کنم.


در نهایت جا دارد از دوست، همکار و مدیر خوبم سینا سعیدی، مدیریت فنی تیم منابع مشترک شرکت ایده گزین ارتباطات روماک تشکر کنم که بدون حمایت‌های ایشان نگارش این پایان‌نامه ممکن نبود.








% با استفاده از دستور زیر، امضای شما، به طور خودکار، درج می‌شود.
\signature








%%%%%%%%%%%%%%%%%%%%%%%%%%%%%%%%%%%%%%%%%
%%%%%%%%%%%%%%%%%%%%%%%%%%%%%%%%%کدهای زیر را تغییر ندهید.
\newpage\clearpage

\pagestyle{style2}

\vspace*{-1cm}
\section*{\centering چکیده}
%\addcontentsline{toc}{chapter}{چکیده}
\vspace*{.5cm}
\ffa-abstract
\vspace*{2cm}


{\noindent\large\textbf{واژه‌های کلیدی:}}\par
\vspace*{.5cm}
\fkeywords
%دستور زیر برای شماره گذاری صفحات قبل از فصل اول با حروف ابجد است.
\pagenumbering{alph}
%-----------------------------------------------------------------------------
%در فایل زیر دستورات مربوط به نمایش صفحات فهرست مطالب- فهرست اشکال و جداول است.
%{\pagestyle{style2}
%\tableofcontents}\newpage
%
%\listoffigures
\cleardoublepage
\pagestyle{style6}
\tableofcontents
\pagestyle{style6}
\cleardoublepage
%اگر لیست تصاویر و لیست جداول ندارید ، کدهای زیر را با گذاشتن % در ابتدای آنها، غیرفعال کنید.
\BeginNoToc
\addtocontents{lof}{\lofheading}% add heading to the first page in LoF
\pagestyle{style5}
\listoffigures
\thispagestyle{style5}
\cleardoublepage
\addtocontents{lot}{\lotheading}% add heading to the first page in LoT
\thispagestyle{style4}
\listoftables
\thispagestyle{style4}
%\cleardoublepage
%
\cleardoublepage
\setcounter{savepage}{\arabic{page}}
\mainmatter
\addtocontents{toc}{\tocheading}% add heading to the first page in ToC, after frontmatter entries
\EndNoToc
% در صورت تمایل می‌توانید با فعال کردن دستور بالا، لیست تصاویر را به  پایان‌نامه خود اضافه کنید.
%-------------------------------------------------------------------------symbols(فهرست نمادها)
% وجود لیست نمادها الزامیست.(لطفاً نمادهای خود را جایگذین نمادهای پیش‌فرض کنید.)
%%%%%%%%%%%%%

{\centering\LARGE\textbf{فهرست نمادها}\par}%

\pagenumbering{alph}
\setcounter{page}{\thesavepage}
%\setcounter{page}{6}
\vspace*{1cm}

\pagestyle{style3}
%\thispagestyle{empty}
%\addcontentsline{toc}{chapter}{فهرست نمادها}
\symb{\text{ نماد}}{مفهوم}
\\
%مقادیر بالا را تغییر ندهید
%%%%%%%%%%%%%%%%%%%%%%%%%%%%%%%%%%%%%%%%%%%%%%%%%%%%%%%%%
\symb{V^{SFC}_{i, F}}{
    \small
    مجموعه گره‌های گراف زنجیره‌ی \(i\)ام
}
\symb{E^{SFC}_i}{
    \small
    مجموعه یال‌های گراف زنجیره‌ی \(i\)ام
}
\symb{V_S^{PN}}{
    \small
    مجموعه گره‌های گراف زیرساخت
}
\symb{E_S^{PN}}{
    \small
    مجموعه یال‌های گراف زیرساخت
}
\symb{x_h}{
    \small
    متغیر باینری که نشان می‌دهد زنجیره‌ی \(h\)ام
    پذیرفته شده است یا خیر
}
\symb{y_{wk}}{
    \small
    تعداد نمونه‌هایی از نوع \(k\)
    که روی سرور فیزیکی \(w\) فعال شده‌اند
}
\symb{z^k_{vw}}{
    \small
    متغیر باینری که نشان می‌دهد نمونه‌ی \(v\)
    از نوع \(k\)
    روی سرور فیزیکی \(w\)
    جایگذاری شده است یا خیر
}
\symb{\bar{y}_w}{
    \small
    تعداد نمونه‌هایی از \lr{VNFM} که روی سرور \(w\) فعال شده‌اند
}
\symb{\bar{z}_{hw}}{
    \footnotesize
    متغیر باینری که نشان می‌دهد زنجیره‌ی \(h\) توسط \lr{VNFM}ای که روی سرور \(w\) قرار گرفته است مدیریت می‌شود یا خیر
}
\symb{\tau^{(u,v)}_{ij}}{
    \scriptsize
    متغیر باینری که نشان می‌دهد یال مجازی بین نمونه‌های \(u\) و \(v\) برای نگاشت از یال فیزیکی بین گره‌های \(i\) و \(j\)
    استفاده می‌کند یا خیر
}
\symb{\bar{\tau}^{v}_{ij}}{
    \scriptsize
    متغیر باینری که نشان می‌دهد برای نگاشت ارتباط مدیریتی نمونه‌ی \(v\) از یال فیزیکی بین گره‌های \(i\) و \(j\)
    استفاده شده است یا خیر
}

%%%%%%%%%%%%%%%%%%%%%%%%%%%%%%%%%%%%%%%

\thispagestyle{style3}
\newpage
%\pagestyle{style1}
%%%%%%%%%%%%%%%%%%%%%%%%%%%%%%%%%%%%


\pagenumbering{arabic}
\pagestyle{style1}
%--------------------------------------------------------------------------chapters(فصل ها)
\chapter{مقدمه}

پیشتر در ارائه سرویس‌های شبکه، از سخت‌افزارهای اختصاصی که توسط سازندگان اختصاصی ارائه می‌شد و به آن‌ها
\lr{middle box}
گفته می‌شد استفاده می‌گشت.
تنوع و تعداد رو به افزایش سرویس‌های جدیدی که توسط کاربران تقاضا می‌گردد
باعث هزینه‌های زیاد برای خرید و نگهداری
\lr{middle box}‌ها
توسط اپراتورها شده است.
به تازگی فراهم آورندگان شبکه
شروع به حرکت به سوی مجازی‌سازی و نرم‌افزاری کردن بسترهای شبکه کرده‌اند،
به این ترتیب آن‌ها قادر خواهند بود
سرویس‌های نوآورانه‌ای به کاربران ارائه بدهند.
این روند به سرویس دهندگان اجازه می‌دهد که ارائه سرویس‌های دلخواه‌شان وابسته به سخت‌افزارهای اختصاصی نباشد و 
هزینه‌های راه‌اندازی و نگهداری فراهم آوردندگان سرویس را کاهش می‌دهد.
با نرم‌افزاری سازی کارکردها، وابستگی آن‌ها به سخت افزار اختصاصی کاهش یافته و به سرعت می‌توان آن‌ها را افزایش/کاهش مقیاس داد.
مجازی‌سازی کارکردهای شبکه و زنجیره‌سازی کارکرد سرویس‌ راهکاری‌هایی هستند که برای همین منظور پیشنهاد شده‌اند.

ایده‌ی اصلی مجازی‌سازی توابع شبکه جداسازی تجهیزات فیزیکی شبکه از کارکردهایی می‌باشد که
بر روی آن‌ها اجرا می‌شوند.
به این معنی که یک کارکرد شبکه مانند دیوار آتش می‌تواند بر روی سرورهای
\lr{HVS}\footnote{\lr{High Volume Server}}
به عنوان یک نرم‌افزار ساده مستقر شود.
با این روش یک سرویس می‌تواند با استفاده از کارکردهای مجازی شبکه‌ای، که می‌توانند به صورت نرم‌افزاری پیاده‌سازی شده
و روی یک یا تعدادی سرور استاندارد فیزیکی اجرا شوند، استقرار یابد.
کارکردهای مجازی شبکه‌ای می‌توانند در مکان‌های مختلف بازمکان‌یابی یا نمونه‌سازی شوند بدون آنکه
نیاز به خریداری و نصب تجهیز جدیدی باشد.
\cite{Mijumbi2016}

در ادامه به معرفی معماری \lr{NFV} پرداخته
و به چالش‌هایی که در \lr{MANO} وجود دارد می‌پردازیم.
در فصل دوم کارهای مرتبط مرور می‌شوند و در فصل سوم مساله تعریف شده بیان می‌گردد. در فصل چهارم
در مورد راه‌حل پیشنهادی برای مساله بحث خواهد شد.

\section{معماری \lr{NFV}}
با توجه به استاندارد \lr{ETSI} معماری \lr{NFV}
از سه عنصر کلیدی تشکیل شده است.
زیرساخت مجازی‌سازی کارکردهای شبکه،
کارکردهای مجازی شبکه‌ای و
\lr{NFV MANO}.
این اجزا در شکل \cref{fig.1} نمایش داده شده‌اند.

\begin{figure}[!h]
\center\includegraphics[scale=.5]{images/nfv-arch}
\caption{معماری مجازی‌سازی کارکردهای شبکه
}\label{fig.1}
\end{figure}

\subsection{زیرساخت مجازی‌سازی کارکردهای شبکه}
زیرساخت مجازی‌سازی کارکردهای شبکه ترکیبی از منابع نرم‌افزاری و سخت‌افزاری است
که محیطی برای نصب
کارکردهای مجازی شبکه فراهم می‌آورد.
منابع سخت‌افزاری شامل منابع محاسباتی،
ذخیره‌سازها و شبکه
(شامل لینک‌ها و گره‌ها)
هستند
که پردازش، ذخیره‌سازی و ارتباط را
برای کارکردهای مجازی شبکه فراهم می‌آورند.
منابع مجازی انتزاعی از منابع شبکه‌ای، پردازشی و ذخیر‌ه‌سازی هستند.
به وسیله انتزاع از طریق لایه‌ی مجازی‌سازی (بر پایه‌ی \lr{hypervisor})
منابع سخت افزاری در اختیار کارکردهای مجازی
قرار می‌گیرند که این منابع شامل منابع محاسباتی، شبکه‌ای و ذخیره‌سازی می‌باشند.

در مراکز داده‌ای ممکن است منابع پردازشی و ذخیره‌سازی تحت عنوان یک یا چند
ماشین مجازی نمایش داده شوند در حالی که شبکه‌های مجازی از لینک‌ها و گره‌های مجازی تشکیل می‌شوند.
شبکه‌های مجازی پیش از بحث مجازی‌سازی کارکردهای شبکه مدنظر بوده‌اند و روی آن‌ها کار شده است.
در واقع از شبکه‌های مجازی در مراکز داده‌ای جهت فراهم آوردن شبکه‌های مختلف و مجزا که به کاربران مختلفی تعلق دارند
استفاده شده است. راه‌حل‌های مختلفی برای پیاده‌سازی این شبکه‌ها وجود دارد. در بحث مجازی‌سازی کارکردهای شبکه‌، زیرساخت ارتباطی
مورد نیاز 
برای کارکردهای مجازی از طریق همین شبکه‌های مجازی فراهم آورده می‌شود.
یعنی مسائلی که پیشتر در بحث جایگذاری شبکه‌های مجازی مطرح بود
امروز جزئی از مسائل جایگذاری زنجیره‌های کارکرد سرویس می‌باشند.

\subsection{کارکردهای مجازی شبکه}
یک کارکرد شبکه، یک بلوک عملیاتی در زیرساخت شبکه است که عملکرد رفتاری و رابط‌های ارتباط با خارج خوش تعریف دارد.
مثال‌هایی از کارکردهای شبکه می‌تواند شامل
\lr{DHCP}
یا
\lr{firewall}
و ... باشد.
با این توضیحات کارکرد مجازی شبکه، پیاده‌سازی یک کارکرد شبکه است
که می‌تواند روی منابع مجازی شده اجرا شود.
از هر کارکرد شبکه می‌توان نمونه‌سازی کرده و چند نمونه را در شبکه مستقر ساخت. 
این نمونه‌ها می‌توانند برای سرویس‌دهی به زنجیره‌های مختلف استفاده شوند. از آنجایی که 
هر نمونه توان پردازشی محدودی دارد با افزایش تعداد نمونه‌ها می‌توان توان پردازشی یک کارکرد را نیز افزایش داد.

\subsection{\lr{NFV MANO}}
بر اساس چهارچوب پیشنهادی \lr{ETSI}
وظیفه‌ی \lr{NFV MANO} فراهم آوردن کارکردهای لازم
برای تدارک و فرآیند‌های مشابه مانند تنظیم کردن و ... کارکردهای مجازی شبکه است.
\lr{NFV MANO} شامل هماهنگ کننده و مدیریت کننده چرخه‌ی زندگی
منابع سخت‌افزاری و نرم‌افزاری که مجازی‌سازی زیرساخت را پشتیبانی می‌کنند، است.
هر زنجیره نیاز دارد که حداقل توسط یک \lr{VNFM} مدیریت شود
تا مثلا خطاهای آن را تحت نظر قرار دهد و در صورت نیاز در قسمت دیگری از شبکه استقرار یابد.
مساله‌ی جایگذاری زنجیره‌ها بسیار مورد مطالعه قرار گرفته است، اما در این بین توجه لازم به نیاز این زنجیره‌ها به یک
\lr{VNFM}
صورت نپذیرفته است.
\chapter{مفاهیم پایه}

\section{مقدمه}

راه‌اندازی و استقرار سرویس در صنعت مخابرات به طور سنتی بر این اساس است که اپراتورهای شبکه سخت‌افزارهای اختصاصی فیزیکی و تجهیزات لازم برای هر کارکرد در سرویس را در زیرساخت خود مستقر کنند.
فراهم کردن نیازمندی‌هایی مانند پایداری و کیفیت بالا منجر به اتکای فراهم‌نندگان سرویس بر سخت‌افزارهای اختصاصی می‌شود. 
این در حالی است که نیازمندی کاربران به سرویس‌های متنوع و عموما با عمرکوتاه و نرخ بالای ترافیک افزایش یافته است.
بنابراین فراهم‌کنندگان سرویس‌ها باید مرتبا و به صورت پیوسته تجهیزات فیزیکی جدید را خریده، انبارداری کرده و مستقر کنند.
تمام این عملیات باعث افزایش هزینه‌های فراهم‌کنندگان سرویس می‌شود.
با افزایش تجهیزات، پیدا کردن فضای فیزیکی برای استقرار تجهیزات جدید به مرور دشوارتر می‌شود.
علاوه بر این باید افزایش هزینه و تاخیر ناشی از آموزش کارکنان برای کار با تجهیزات جدید را نیز در نظر گرفت.
بدتر این که هر چه نوآوری سرویس‌ها و فناوری شتاب بیشتری می‌گیرد، چرخه عمر سخت‌افزارها کوتاه‌تر می‌شود که مانع از ایجاد نوآوری در سرویس‌های شبکه می‌شود.

در روش سنتی استقرار سرویس شبکه، ترافیک کاربر باید از تعدادی کارکرد شبکه به ترتیب معینی عبور کند تا یک مسیر پردازش ترافیک ایجاد شود.
در حال حاضر این کارکردها به صورت سخت‌افزاری به یکدیگر متصل هستند و ترافیک با استفاده از جداول مسیریابی به سمت آن‌ها هدایت می‌شود.
چالش اصلی این روش در این است که استقرار و تغییر ترتیب کارکردها دشوار است.
به عنوان مثال، به مرور زمان با تغییر شرایط شبکه نیازمند تغییر همبندی و یا مکان کارکردها برای سرویس‌دهی بهتر به کاربران هستیم که نیاز به جا‌به‌جایی کارکردها و تغییر جداول مسیریابی دارد.
در روش سنتی این کار سخت و هزینه‌بر است که ممکن است خطاهای بسیاری در آن رخ دهد.
از جنبه‌ی دیگر، تغییر سریع سرویس‌های مورد نظر کاربران نیازمند تغییر سریع در ترتیب کارکردها است که در روش فعلی این تغییرات به سختی صورت گیرد.
بنابراین اپراتورهای شبکه نیاز به شبکه‌های قابل برنامه ریزی و ایجاد زنجیره سرویس کارکردها به صورت پویا پیدا کرده‌اند.

دو فناوری برای پاسخ‌گویی به این چالش‌ها مطرح شده است:

\begin{itemize}
    \item مجازی‌سازی کارکرد شبکه یا \lr{NFV}
    \item زنجیره‌سازی کارکردهای سرویس یا \lr{SFC}
\end{itemize}

با استفاده از مجازی‌سازی کارکردهای شبکه و اجرای آن‌ها بر روی سرورهای استاندارد با توان بالا،
امکان اجرای کارکردها بر روی سخت افزارهای عمومی فراهم شده است تا نیاز به تجهیزات سخت افزاری خاص منظوره کاهش یابد.
از طرف دیگر \lr{SFC} امکان تعریف زنجیره کارکردها را ارائه می‌کند که ایجاد
و انتخاب مسیرهای متفاوت برای پردازش ترافیک به صورت پویا و بدون ایجاد تغییر در زیرساخت فیزیکی را امکان‌پذیر می‌کند
با توجه به این فناوری‌ها، مسائل تحقیقاتی جدیدی مطرح شده‌اند که از مهم‌ترین آن‌ها می‌توان تخصیص منابع بهینه به سرویس درخواستی کاربر را نام برد.

از آنجایی که از مفاهیم این فناور‌ی‌ها برای طراحی و تعریف مساله در این رساله استفاده شده است، نیازمند آشنایی با مفاهیم ابتدایی و اصول اولیه آن‌ها خواهیم بود.

بنابراین در این فصل به صورت خلاصه اجزای این فناوری‌ها را مرور خواهیم کرد و کاربردها، چالش‌ها و مسائل تحقیقاتی که در هر یک از این معماری‌ها وجود دارد را مورد بررسی قرار خواهیم داد.

\section{مجازی‌سازی کارکرد شبکه}

مجازی‌سازی کارکرد شبکه اصل جداسازی کارکرد شبکه به وسیله انتزاع سخت‌افزاری مجازی از سخت افزاری است که بر روی آن اجرا می‌شود.
هدف مجازی‌سازی کارکرد شبکه تغییر روش اپراتورهای شبکه در طراحی شبکه
با تکامل مجازی سازی استاندارد فناوری اطلاعات به منظور تجمیع تجهیزات شبکه
در سرورهای استاندارد، سوییچ‌ها و ذخیره‌سازها با توان بالا است.
یک سرور استاندارد با توان بالا سروری است که توسط اجزای استاندارد شده \lr{IT}،
مانند معماری \lr{x86}، ساخته شده و
در تعداد بالایی، مانند میلیون،
فروخته می‌شود.
ویژگی اصلی این سرورها این است که اجزای آن‌ها به راحتی از فروشندگان مختلف قابل خریداری و
تعویض است.
این تجهیزات می‌توانند در مراکز داده، گره‌های شبکه، یا مکان کاربران انتهایی قرار بگیرند.
این روند در
شکل
\ref{fig.6}
نیز توصیف شده است.

\begin{figure}[!h]
\center\includegraphics[scale=.5]{images/nfv-concept}
\caption{رویکرد \lr{NFV}}\label{fig.6}
\end{figure}

با استفاده از \lr{NFV}، انواع کارکردهای شبکه مانند دیواره آتش و \lr{NAT}
را می‌توان به صورت یک برنامه نرم‌افزاری از فروشندگان مختلف تهیه کرد و
آن‌ها را بر روی سرورهای با توان بالا اجرا کرد که نیاز به نصب تجهیزات خاص منظوره و
جدید را برطرف می‌سازد.

مزایا و اهداف اساسی که \lr{NFV} برای تحقق و دست‍یابی به آن‍ها شکل گرفته است عبارتند از:

\begin{itemize}
    \item
    کاهش هزینه‌های تجهیزات و مصرف انرژی از طریق تجمیع کارکردها بر روی سرورها و در نتیجه کاهش تعداد تجهیزات
    \item
    کاهش نیاز به آموزش کارکنان، افزایش دسترسی پذیری به سخت افزار و کاهش زمان بازیابی از خرابی سخت افزار به علت استفاده از سخت افزارهای استاندارد و عمومی
    \item
    افزایش سرعت عرضه محصول به بازار با کوتاه‌کردن چرخه نوآوری و تولید. در واقع \lr{NFV} به اپراتورهای شبکه کمک می‍کند تا چرخه بلوغ محصول را به اندازه قابل توجهی کاهش دهند.
    \item
    امکان‌پذیر بودن تعریف سرویس مورد نظر بر اساس نوع مشتری یا محل جغرافیایی. مقیاس سرویس‌ها می‍تواند به سرعت، بر اساس نیاز، گسترش یا کاهش یابد.
    \item
    تشویق به ایجاد نوآوری و ارائه سرویس‌های جدید و دریافت جریان‌های درآمدی تازه با سرعت بالا و ریسک پایین.
    \item
    افزایش توانایی  مقابله با خرابی کارکردها، قابلیت به اشتراک گذاری منابع بین کارکردها و پشتیابی از چند مشتری
\end{itemize}

سازمان‌های استانداردگذاری متعددی در استانداردسازی فناوری \lr{NFV} دخیل هستند که شاخص‌ترین آن‌ها موسسه استانداردهای مخابراتي اروپا (\lr{ETSI}) است.
در اواخر سال ۲۰۱۲،
\lr{ETSI NFV ISG}
توسط هفت اپراتور جهانی شبکه به منظور ارتقا ایده مجازی‌سازی کارکرد شبکه تأسیس شد.
\lr{NFV ISG}
تبدیل به یک بستر صنعتی اصلی برای توسعه چارچوب معماری \lr{NFV} و نیازمندی‌های آن شده است و اکنون بیش از ۲۵۰ سازمان با آن همکاری می‌کنند.
اسناد معماری \lr{NFV} به صورت عمومی و رایگان توسط \lr{ETSI NFV ISG} منتشر می‌شود.
ما در این رساله برای توصیف معماری \lr{NFV} از اسناد ارائه شده این سازمان استفاده می‌کنیم.

\section{معماری \lr{NFV}}

در این بخش مؤلفه‌های تشکیل‌دهنده معماری \lr{NFV} شرح داده می‌شوند.
هر یک از اجزای معماری می‌توانند توسط تولیدکنندگان متفاوتی تأمین شوند و به وسیله واسط‌هایی که توسط معماری \lr{NFV}
توصیف شده‌اند با یکدیگر در ارتباط باشند.
بنابراین معماری \lr{NFV} توصیف شده توسط \lr{ETSI} راه‌حلی با قابلیت مشارکت و هماهنگی چندین تولیدکننده مختلف را دارد.
با توجه به استاندارد \lr{ETSI} معماری \lr{NFV}
از سه عنصر کلیدی تشکیل شده است.
زیرساخت مجازی‌سازی کارکردهای شبکه،
کارکردهای مجازی شبکه‌ای و
\lr{NFV MANO}.
این اجزا در شکل \ref{fig.1} نمایش داده شده‌اند.

\begin{figure}[!h]
\center\includegraphics[scale=.5]{images/nfv-arch}
\caption{معماری مجازی‌سازی کارکردهای شبکه
}\label{fig.1}
\end{figure}

\begin{itemize}
    \item
    \lr{NFVI}: شامل منابع سخت افزاری و نرم‌افزاری لازم برای اجرای \lr{VNF}‌ها
    \item
    \lr{Service}: شامل \lr{VNF}‌ها که کارکردهای شبکه را پیاده‌سازی کرده‌اند، \lr{EMS} برای مدیریت \lr{VNF}‌ها و \lr{OSS/BSS} برای ارتباط با سیستم‌های مدیریت سنتی
    \item
    \lr{MANO}: که وظیفه مدیریت و هماهنگی سرویس‌ها و تخصیص منابع را برعهده دارد و از سه بخش \lr{NFVO}، \lr{VIM} و \lr{VNFM} تشکیل شده است.
\end{itemize}

\subsection{زیرساخت مجازی‌سازی کارکردهای شبکه یا \lr{NFVI}}
زیرساخت مجازی‌سازی کارکردهای شبکه ترکیبی از منابع نرم‌افزاری و سخت‌افزاری است
که محیطی برای نصب
کارکردهای مجازی شبکه فراهم می‌آورد.
منابع سخت‌افزاری شامل منابع محاسباتی،
ذخیره‌سازها و شبکه
(شامل لینک‌ها و گره‌ها)
هستند
که پردازش، ذخیره‌سازی و ارتباط را
برای کارکردهای مجازی شبکه فراهم می‌آورند.
منابع مجازی انتزاعی از منابع شبکه‌ای، پردازشی و ذخیر‌ه‌سازی هستند.
به وسیله انتزاع از طریق لایه‌ی مجازی‌سازی (بر پایه‌ی \lr{hypervisor})
منابع سخت افزاری در اختیار کارکردهای مجازی
قرار می‌گیرند که این منابع شامل منابع محاسباتی، شبکه‌ای و ذخیره‌سازی می‌باشند.

در مراکز داده‌ای ممکن است منابع پردازشی و ذخیره‌سازی تحت عنوان یک یا چند
ماشین مجازی نمایش داده شوند در حالی که شبکه‌های مجازی از لینک‌ها و گره‌های مجازی تشکیل می‌شوند.
شبکه‌های مجازی پیش از بحث مجازی‌سازی کارکردهای شبکه مدنظر بوده‌اند و روی آن‌ها کار شده است.
در واقع از شبکه‌های مجازی در مراکز داده‌ای جهت فراهم آوردن شبکه‌های مختلف و مجزا که به کاربران مختلفی تعلق دارند
استفاده شده است. راه‌حل‌های مختلفی برای پیاده‌سازی این شبکه‌ها وجود دارد. در بحث مجازی‌سازی کارکردهای شبکه‌، زیرساخت ارتباطی
مورد نیاز 
برای کارکردهای مجازی از طریق همین شبکه‌های مجازی فراهم آورده می‌شود.
یعنی مسائلی که پیش‌تر در بحث جایگذاری شبکه‌های مجازی مطرح بود
امروز جزئی از مسائل جایگذاری زنجیره‌های کارکرد سرویس می‌باشند.

\subsection{کارکردهای مجازی شبکه}
یک کارکرد شبکه، یک بلوک عملیاتی در زیرساخت شبکه است که عملکرد رفتاری و رابط‌های ارتباط با خارج خوش تعریف دارد.
مثال‌هایی از کارکردهای شبکه می‌تواند شامل
\lr{DHCP}
یا
\lr{firewall}
و ... باشد.
با این توضیحات، کارکرد مجازی شبکه، پیاده‌سازی یک کارکرد شبکه است
که می‌تواند روی منابع مجازی شده اجرا شود.
از هر کارکرد شبکه می‌توان نمونه‌سازی کرده و چند نمونه را در شبکه مستقر ساخت. 
این نمونه‌ها می‌توانند برای سرویس‌دهی به زنجیره‌های مختلف استفاده شوند. از آنجایی که 
هر نمونه توان پردازشی محدودی دارد با افزایش تعداد نمونه‌ها می‌توان توان پردازشی یک کارکرد را نیز افزایش داد.

\subsection{\lr{EM}}
این مولفه کارکردهای \lr{FCAPS}\footnote{\lr{Fault,Config,Accounting,Performance,Security}} را برای \lr{VNF} ها انجام می‌دهد که شامل مدیریت خطا، پیکربندی، امنیت، حسابداری و کارایی برای کارکردی است که \lr{VNF} ارائه می‌دهد. این مولفه ممکن است آگاه از مجازی بودن کارکرد، باشد و با همکاری \lr{VNFM} عملکردهای خودش را انجام بدهد.

\subsection{\lr{OSS/BSS}}
این مولفه، ترکیبی از سایر بخش‌های عملکردهای اپراتور است که در چارچوب معماری \lr{NFV} ارائه شده از طرف \lr{ETSI} قرار نمی‌گیرند. به عنوان مثال می‌تواند شامل مدیریت سیستم‌های \lr{Legacy} باشد.

\subsection{\lr{NFV MANO}}
بر اساس چهارچوب پیشنهادی \lr{ETSI}
وظیفه‌ی \lr{NFV MANO} فراهم آوردن کارکردهای لازم
برای تدارک فرآیند‌های مشابه مانند تنظیم کردن و ... کارکردهای مجازی شبکه است.
\lr{NFV MANO} شامل هماهنگ‌کننده و مدیریت‌کننده چرخه‌ی زندگی
منابع سخت‌افزاری و نرم‌افزاری که مجازی‌سازی زیرساخت را پشتیبانی می‌کنند، است.
هر زنجیره نیاز دارد که حداقل توسط یک \lr{VNFM} مدیریت شود
تا مثلا خطاهای آن را تحت نظر قرار دهد و در صورت نیاز در قسمت دیگری از شبکه استقرار یابد.
مساله‌ی جایگذاری زنجیره‌ها بسیار مورد مطالعه قرار گرفته است، اما در این بین توجه لازم به نیاز این زنجیره‌ها به یک
\lr{VNFM}
صورت نپذیرفته است.

\lr{VNFO} بخشی از مولفه \lr{MANO} است که وظیفه تخصیص منابع به سرویس را برعهده دارد.
یکی از مهم ترین اجزای سرویس گراف \lr{VNF-FG} است که بیانگر \lr{VNF} های سرویس و ارتباطات بین آن‌ها است.
وظیفه اصلی مولفه \lr{NFVO} ایجاد نمونه از سرویس و مدیریت چرخه حیات آن است.
ایجاد نمونه از سرویس شامل ایجاد نمونه از \lr{VNF}‌های تشکیل‌دهنده آن و ایجاد ارتباط بین نمونه‌ها است.
سایر وظایف مولفه \lr{VNFO} به شرح زیر است:
\begin{itemize}
    \item مدیریت چرخه حیات سرویس شبکه
    \item مدیریت و هماهنگی منابع مورد نیاز \lr{NFVI} بین چندین \lr{VIM}
    \item مدیریت منابع و ایجاد نمونه از \lr{VNF}‌ها با هماهنگی \lr{VNFM}
    \item مدیریت منابع و نمونه‌سازی \lr{VNFM}
    \item مدیریت همبندی نمونه ساخته شده از سرویس شبکه مانند ایجاد، حذف و به روز رسانی \lr{VNF-FG}
    \item مدیریت قالب‌های استقرار سرویس شبکه و \lr{VNF‌}ها مانند اعتبارسنجی قالب‌ها
\end{itemize}
همچنین این مولفه مسئولیت مشخص کردن مکان فیزیکی نمونه های ایجاد شده از \lr{VNF}ها را برعهده دارد.

مولفه‌ی \lr{VNFM} مسئولیت مدیریت چرخه حیات نمونه‌های ایجاد شده از\lr{VNF}‌ها را برعهده دارد.
بنابراین فرض می‌شود هر نمونه ایجاد شده از هر \lr{VNF}، به یک \lr{VNFM} اختصاص یافته است.
مهم‌ترین وظایف این مولفه به شرح زیر است:
\begin{itemize}
    \item پیکربندی و نمونه‌سازی از \lr{VNF}‌ها
    \item گسترش و یا کاهش مقیاس‌پذیری افقی یا عمودی برای نمونه‌های ایجاد شده از \lr{VNF}ها
    \item مدیریت نمونه‌های ایجاد شده شامل  تغییرات، به روزرسانی برنامه‌ها و خاتمه دادن به نمونه‌ها
\end{itemize}
مولفه \lr{VNFM} با استفاده از \lr{VNFD}، از \lr{VNF} نمونه ایجاد می‌کند و
مدیریت چرخه حیات آن را انجام می‌دهد.
منابع پردازشی، محاسباتی و شبکه مطابق با توصیفات گفته شده در \lr{VNFD} به نمونه‌های آن اختصاص می‌یابند.

مولفه \lr{VIM} مسئولیت کنترل و مدیریت منابع محاسباتی، ذخیره‌سازی و شبکه‌ای،
معمولا در حوزه یک اپراتور، را برعهده دارد.
مهم ترین وظایف این مولفه عبارتند از:

\begin{itemize}
    \item
    هماهنگی تخصیص، ارتقا و آزادسازی منابع \lr{NFVI} شامل بهینه‌‌سازی استفاده از منابع و مدیریت انجمنی منابع مجازی و فیزیکی.
    بنابراین \lr{VIM} اطلاعات \lr{Inventory} تخصیص منابع مجازی به منابع فیزیکی را نگهداری می‌کند.
    \item
    پشتیبانی از مدیریت \lr{VNF-FG} به وسیله ایجاد و نگهداری لینک‌های مجازی،
    شبکه‌های مجازی زیرشبکه‌ها و پورت‌ها
    \item
    مدیریت اطلاعات \lr{Inventory} سخت افزارها و نرم افزارها و کشف قابلیت‌ها و ویژگی‌های آن‌ها
    \item
    مدیریت ظرفیت منابع مجازی مانند نسبت منابع مجازی به حقیقی
    \item
    مدیریت تصویرهای نرم‌افزاری، مانند تصاویر \lr{VNF}ها، که ممکن است توسط سایر مولفه‌های \lr{MANO} هم مورد استفاده قرار بگیرند.
    \item
    جمع‌آوری اطلاعات کارایی و خطا از منابع سخت‌افزاری و نرم‌افزاری
\end{itemize}

\section{\lr{VNFD}}

هر \lr{VNF} توسط توصیف‌گر مربوط به آن که نیازمندی‌های استقرار و رفتاری آن را مشخص می‌کند توصیف می‌شود.
مولفه \lr{VNFM} از \lr{VNFD} در فرایند نمونه‌سازی \lr{VNF}‌ها و مدیریت چرخه حیات آن‌ها استفاده می‌کند.
همچنین این اطلاعات توسط مولفه \lr{VNFO} برای ایجاد مدیریت و هماهنگی سرویس شبکه نیز استفاده می‌شود.
\lr{VNFD} شامل شاخص‌های کارایی است که می‌تواند توسط \lr{VNFM} نیز مورد استفاده قرار بگیرد.
در \lr{VNFD} ارتباطات داخلی و واسط‌ها نیز توصیف می‌شوند که
برای ایجاد لینک‌های مجازی بین مولفه‌های \lr{VNFC} و یا ارتباط بین \lr{VNF} با سایر \lr{VNF}‌ها مورد استفاده قرار می‌گیرد.
\lr{VNFD} همچنین شامل قالب‌های استقرار \lr{VNF} به همراه نیازمندی منابع برای هر قالب است.

\section{سرویس شبکه و اجزای آن}
یک سرویس شبکه را می‌توان به صورت یک گراف جلورانی از کارکردهای شبکه
\lr{(NF-FG)}
که به یکدیگر از طریق زیرساخت شبکه متصل هستند دید.
کارکردهای شبکه می‌تواند توسط یک یا چند اپراتور ارائه شده باشند.
نقاط انتهایی سرویس را می‌توان به صورت گره‌های گراف و
ارتباطات میان کارکردها را توسط لینک‌های گراف مدل سازی کرد
که لینک‌های گراف می‌توانند، یک طرفه یا دو طرفه، چند پخشی یا همه پخشی باشند.
مثالی از یک سرویس شبکه در شکل
\ref{fig.18}
نمایش داده شده است.
در این شکل، یک سرویس شبکه انتها به انتها از طریق نقاط انتهایی \lr{A} و \lr{B} ایجاد شده که شامل یک \lr{NF-FG} داخلی است.
این \lr{NF-FG} خود شامل سه کارکرد شبکه است که به یکدیگر متصل هستند.


\begin{figure}[h!]
\center\includegraphics[scale=.5]{images/network-service}
\caption{یک سرویس شبکه شامل یک گراف جلورانی}
\label{fig.18}
\end{figure}

در صورتی که در یک  \lr{NF-FG} حداقل یکی از این کارکردها \lr{VNF} باشد، به آن \lr{VNF-FG} گفته می شود.
در صورتی که فرض کنیم همه \lr{NF} های شکل
\ref{fig.18}
، \lr{VNF} هستند می‌توان آن را مطابق شکل
\ref{fig.19}
نمایش داد.
در این شکل \lr{NF2} خود توسط سه \lr{VNF} پیاده‌سازی شده است.

\begin{figure}[h!]
\center\includegraphics[scale=.5]{images/vnf-fg}
\caption{گراف \lr{VNF-FG} متناظر با شکل \ref{fig.18}}
\label{fig.19}
\end{figure}

مشخص است که گراف \lr{VNF-FG} صرفا ارتباطات بین \lr{VNF}‌ها را مشخص می‌کند ولی ترتیب عبور ترافیک از کارکردها را بیان نمی‌کند.
ترتیب عبور ترافیک از کارکردها توسط \lr{NFP} بیان می‌شود که یکی از اجزای \lr{VNF-FG} است و هر \lr{VNF-FG} باید حداقل یک \lr{NFP} داشته باشد.

\section{موارد کاربرد}
در این بخش موارد کاربرد مهم معماری \lr{NFV} را شرح می‌دهیم.
یکی از مهم‌ترین موارد کاربرد ذکر شده برای \lr{NFV}،  مجازی‌سازی تجهیزات \lr{CPE} است.
عموما تجهیزاتی در مکان کاربران برای اتصال به اینترنت نگهداری می‌شود که شامل دیواره آتش، \lr{NAT}، مسیریاب و سوییچ است.
در این حالت تنظیمات تجهیزات باید در مکان کاربر صورت بگیرد که هزینه بالایی دارد.
با استفاده از مجازی‌سازی این کارکردها و نگه‌داری آن در سمت \lr{ISP}، می‌توان هزینه تجهیزات و زمان نگهداری را کاهش داد.
این مورد کاربرد در شکل \ref{fig.20} نمایش داده شده است.

\begin{figure}[h!]
\center\includegraphics[scale=.5]{images/cpe}
\caption{مجازی‌سازی \lr{CPE}}
\label{fig.20}
\end{figure}

به عنوان مثالی دیگر از موارد کاربرد می‌توان مجازی‌سازی کارکردها در زیرساخت \lr{LTE} را در نظر گرفت.
در شکل \ref{fig.21} شرایط قبل و بعد از مجازی سازی
\lr{P-GW}، \lr{S-GW}، \lr{PCRF} و \lr{MMF}
نمایش داده شده است.
با مجازی‌سازی این کارکردها علاوه بر استفاده بهتر از منابع، می توان تعداد نمونه‌های آن ها را مطابق با تعداد کاربران بدون تغییر در زیرساخت افزایش و یا کاهش داد.

\begin{figure}[h!]
\center\includegraphics[scale=.5]{images/lte}
\caption{مجازی سازی زیرساخت \lr{LTE}}
\label{fig.21}
\end{figure}

\section{زنجیره‌سازی کارکرد سرویس}

زنجیره‌سازی کارکردها ایده جدیدی نیست.
در حال حاضر اپراتورها برای ارائه سرویس یک زنجیره از کارکردها ایجاد می‌کنند که ترافیک کاربر باید از کارکردها با یک ترتیب مشخص عبور کند.
اگرچه همانطور که بیان شد در صورت تغییر در ترتیب کارکردها و یا زنجیره‌ها و یا ایجاد سرویس‌های جدید، نیازمند تغییر مکان فیزیکی کارکردها خواهیم بود
که کاری سخت است و خالی از اشکال نیست.
معماری زنجیره‌سازی کارکردهای سرویس، اپراتورهای شبکه را قادر می‌سازد که سرویس‌های جدید را به صورت نرم افزاری و پویا و بدون اینکه در سطح سخت افزار تغییری ایجاد کنند، ارائه کنند.
در این راستا \lr{IETF} در اسناد متعددی به شرح معماری و اجزای آن پرداخته است.
در این بخش به شرح معماری زنجیره‌سازی کارکرد سرویس می‌پردازیم و بخش‌های اصلی آن را بیان می‌کنیم.

\subsection{اجزای معماری \lr{SFC}}

در این \lr{RFC} یک سرویس شبکه که توسط اپراتور ارائه می‌شود و
از طریق یک یا چند کارکرد سرویس تحویل می‌شود،
تعریف شده است.
یک کارکرد سرویس، رفتار خاصی (به غیر از جلورانی) با بسته را انجام می‌دهد
و می‌تواند در هر یک از لایه‌های مدل \lr{OSI} فعالیت کند.
به یک شبکه یا بخشی از آن که در آن \lr{SFC}  پیاده‌سازی شده است یک دامنه \lr{SFC} گفته می‌شود.
در یک دامنه \lr{SFC}، معماری \lr{SFC} مطابق با شکل \ref{fig.22} پیاده‌سازی می‌شود.
معماری \lr{SFC} توسط \lr{RFC 7665} تعریف شده است.

\begin{figure}[h!]
\center\includegraphics[scale=.5]{images/sfc}
\caption{معماری \lr{SFC}}
\label{fig.22}
\end{figure}

به صورت خلاصه اجزای اصلی این معماری عبارتند از:
\begin{itemize}
    \item زنجیره کارکرد سرویس: یا به صورت خلاصه زنجیره کارکرد یک مجموعه مرتب از کارکردهای سرویس انتزاعی و محدودیت‌های ترتیبی که باید به بسته‌ها، فریم‌ها و یا جریان‌های دریافتی به عنوان نتیجه دسته بندی اعمال شود.
    \item دسته‌بند: وظیفه دسته‌بندی و انتخاب زنجیره کارکرد برای ترافیک ورودی بر اساس قوانین از پیش تعیین شده را برعهده دارد.
    \item \lr{SFF}: وظیفه جلورانی و هدایت ترافیک در دامنه \lr{SFC} را برعهده دارد.
    \item کارکرد سرویس (\lr{SF}): یک کارکرد انتزاعی که مسئول رفتار خاصی با بسته به جز جلورانی است.
    \item \lr{SF Proxy}: وظیفه ارتباط با کارکردهای غیر آگاه از کپسول بندی \lr{SFC} را برعهده دارد.
    \item صفحه کنترل: وظیفه کنترل و نظارت بر زنجیره‌ها و ایجاد قوانین دسته‌بندی بر روی دسته‌بند را برعهده دارد.
\end{itemize}
همه این اعضا به صورت منطقی هستند و می‌توانند در شبکه به صورت فیزیکی و یا مجازی در یک یا چندین دستگاه فیزیکی به صورت مشترک با یکدیگر وجود داشته باشند.

\section{جمع‌بندی}
در این بخش معماری‌های \lr{NFV} و \lr{SFC} به صورت کامل شرح داده شد و اجزای آن‌ها بررسی شد.
همانگونه که بیان شد، معماری \lr{NFV} بر مجازی‌سازی کارکردها تمرکز دارد.
یک سرویس در معماری \lr{NFV} با استفاده از \lr{NSD} توصیف می‌شود که
شامل \lr{VNF-FG}، \lr{VNF}‌ها و لینک‌های توصیف کننده ارتباطات بین \lr{VNF}‌ها است.
معماری \lr{SFC} به ایجاد زنجیره پویا از کارکردها تمرکز دارد.
در معماری \lr{VNF} سرویس‌های مدیریتی و نظارتی مانند \lr{VNFM}، \lr{VNFO} و ... نیز تعریف می‌شوند که
وظیفه‌ی نظارت و مدیریت چرخه‌ی حیات در این معماری را دارا می‌باشد و تمرکز اصلی این پژوهش بر در نظر گرفتن اهمیت این سرویس‌ها می‌باشد.

یک زنجیره کارکرد توسط یک گراف \lr{SFC} توصیف می شود که
به صورت مجموعه مرتب از کارکردها که ترافیک باید با ترتیب مشخصی از آن‌ها عبور کند توصیف می‌‌شود.
این معماری تاکیدی بر مجازی‌سازی کارکردها ندارد.
همچنین برای مسیریابی ترافیک در این معماری نیز می توان از سرآیند \lr{NSH} استفاده کرد.
همانگونه که بیان شد در حقیقت این دو معماری مکمل یکدیگر هستند و
یک گراف \lr{SFC} را می‌توان توسط یک \lr{VNF-FG} معادل نمایش داد.
\chapter{کارهای مرتبط}
در \cite{Eramo2016}
نویسندگان قصد دارند با در نظر گرفتن محدودیت ظرفیت لینک‌ها و محدودیت پردازشی نودها
بیشترین تعداد زنجیره‌ی کاکرد را بپذیرند. برای این کار یک مساله‌ی \lr{ILP}
طراحی می‌کنند و ثابت می‌کنند که این مساله \lr{NP-Hard} می‌باشد.
در این مقاله وجود \lr{VNFM} برای زنجیره‌ها در نظر گرفته نشده است.

در \cite{AbuLebdeh2017}
نویسندگان استفاده از \lr{VNFM} را مدنظر قرار داده‌اند
. در این مقاله فرض شده است که جایگذاری \lr{SFC}ها صورت گرفته است
و می‌خواهیم \lr{VNFM}ها را به گونه‌ای استقرار دهیم
که با رعایت شدن نیازمندی‌های کارآیی، هزینه‌ی عملیاتی سیستم حداقل شود.
مساله مطرح شده به صورت \lr{ILP} مدلسازی می‌شود.
این مقاله هزینه‌ی عملیاتی سیستم را تحت چهار عنوان دسته‌بندی می‌کند:
هزینه‌ی مدیریت چرخه‌ی زندگی، هزینه‌ی منابع محاسباتی، هزینه‌ی مهاجرت و هزینه‌ی بازنگاشت.
در این مقاله فرض می‌شود که هر نمونه از \lr{VNFM}ها می‌تواند به تعداد مشخصی از نمونه‌های \lr{VNF}
سرویس‌دهی کند و این سرویس‌دهی به نوع نمونه وابسته نیست.
این مقاله محدودیت‌های پردازشی و ظرفیتی را مدنظر قرار می‌دهد.

در \cite{Ghaznavi2017}
نویسندگان سه مرحله برای عملیات جایگذاری زنجیره‌های کارکرد سرویس معرفی می‌کنند:
\begin{itemize}
    \item انتخاب
    \item جابگذاری
    \item مسیریابی
\end{itemize}
\chapter{راه‌حل پیشنهادی}

مساله‌ی بیان شده به صورت \lr{ILP}
مدل‌سازی می‌شود.
در \cite{Eramo2016}
مساله‌ی جایگذاری \lr{SFC}ها با هدف حداکثرسازی تعداد درخواست‌های پذیرفته شده
به صورت \lr{ILP} مدل‌سازی شده و اثبات شده است که مساله‌ی حاضر \lr{NP-Hard} می‌باشد.
مساله‌ای که در اینجا مدل‌سازی می‌شود از آن مساله پیچیده‌تر می‌باشد زیرا در نظر گرفتن \lr{VNFM}ها را نیز شامل می‌شود
بنابراین این مساله نیز \lr{NP-Hard} خواهد بود.
برای این مساله می‌توان
یک راه حل مکاشفه‌ای با زمان چند جمله‌ای
پیشنهاد داد. این راه حل بهینه نبوده و به همین علت کارآیی آن
در سناریوهایی با مدل‌سازی بهینه مقایسه می‌شود.

یکی از راه‌حل‌های ساده مرتب کردن تمام تقاضاها براساس منابع مصرفی (پهنای باند و منابع پردازشی)
و در ادامه جایگذاری آن‌ها از تقاضای با کمترین منابع مصرفی به تقاضای با بیشترین منابع مصرفی می‌باشد.
در ادامه از تقاضا با کمترین منابع مصرفی آغاز کرده و آن را روی سرورها قرار می‌دهیم، برای این امر یک تابع ارزش‌دهی پیشنهاد می‌شود
و این جایگذاری روی سرور با بیشترین ارزش صورت می‌پذیرد.
در نهایت نگاشت لینک‌ها صورت می‌پذیرد، برای این کار نگاشت با هدف توزیع‌بار و به صورت چند مسیره صورت می‌پذیرد.

\chapter{راه‌حل پیشنهادی}

مساله‌ی بیان شده به صورت \lr{ILP}
مدل‌سازی می‌شود.
در \cite{Eramo2016}
مساله‌ی جایگذاری \lr{SFC}ها با هدف حداکثرسازی تعداد درخواست‌های پذیرفته شده
به صورت \lr{ILP} مدل‌سازی شده و اثبات شده است که مساله‌ی حاضر \lr{NP-Hard} می‌باشد.
مساله‌ای که در اینجا مدل‌سازی می‌شود از آن مساله پیچیده‌تر می‌باشد زیرا در نظر گرفتن \lr{VNFM}ها را نیز شامل می‌شود.
برای این مساله می‌توان
یک راه حل مکاشفه‌ای با زمان چند جمله‌ای
پیشنهاد داد.

\section{الگوریتم مکاشفه‌ای}

مساله از دو قسمت تشکیل شده است. قسمت اول مساله‌ی جایگذاری لینک‌ها و نمونه‌ها می‌باشد
و قسمت دوم جایگذاری
\lr{VNFM}
برای زنجیره است.
برای قسمت اول راه‌حل‌های مکاشفه‌ای زیادی ارائه شده است که ما در اینجا
از راه‌حل \cite{Bari2015} استفاده می‌کنیم.
در این راه حل برای قرارگیری هر زنجیره یک گراف چند گامی شگل می‌گیرد.
هر گام این گراف نماینده یک نمونه از زنجیره است که می‌بایست قرار گیرد.
در نظر داشته باشید که در مساله‌ای اصلی نیازی نیست که حتما زنجیره‌ها به صورت خطی باشند اما در این راه‌حل این فرض وجود دارد
که البته فرضی نزدیک به واقعیت می‌باشد.
در هر گام از این گراف مجموعه‌ای از نودهای فیزیکی امکان پذیر شکل می‌گیرد.
با توجه به وضعیت مسیریابی این مجموعه با مجموعه بعدی نود فیزیکی برای نمونه مورد نظر از زنجیره انتخاب می‌شود.

\begin{figure}[h]
\center\includegraphics[scale=.45]{images/bari}
\caption{مدل‌سازی با گراف چندگامی \cite{Bari2015}}
\label{fig.5}
\end{figure}

منظور از وضعیت مسیریابی به شرح زیر است. برای هر یک از گام‌ها از الگوریتم جستجوی اول سطح یا
\lr{BFS}
استفاده می‌کنیم
و به این ترتیب مسیرهای فیزیکی که می‌توان از آن‌ها برای جایابی لینک مجازی استفاده کرد پیدا می‌کنیم.
از این بین گره‌ای که مسیرهای فیزیکی امکان‌پذیر بیشتری دارد انتخاب می‌گردد.
با این روش مجموعه امکان‌پذیر گام بعدی بزرگتر می‌شود و امکان حذف زنجیره به دلیل نبود مسیر فیزیکی
برای جایابی لینک مجازی کمتر می‌گردد.

در ادامه یک گام به این الگوریتم اضافه می‌کنیم که در آن برای هر زنجیره بعد از قرارگرفتن یک
\lr{VNFM}
تخصیص می‌دهیم. برای اینکار مجموعه‌ای امکان‌پذیر از نودهای فیزیکی را انتخاب می‌کنیم
و سعی می‌کنیم از بین آن‌ها انتخاب کنیم. در روند این انتخاب از اصول زیر پیروی می‌کنیم:

\begin{itemize}
    \item اولویت با نود فیزیکی است که روی آن \lr{VNFM} با ظرفیت خالی وجود دارد.
    \item از بین نودهایی که ظرفیت خالی دارند اولویت با نودی است که منابع پردازشی بیشتری دارد.
\end{itemize}

از آنجایی که مساله‌ی طرح شده به صورت آفلاین می‌باشد می‌توان با بررسی ورودی‌های الگوریتم کارآیی آن را بهبود داد.
برای این منظور زنجیره‌های ورودی را برحسب اندازه‌ی آن‌ها مرتب می‌کنیم.
در این مرتب‌سازی تلاش می‌شود که زنجیره‌های بزرگتر که سود بیشتری دارند زودتر جایابی شوند.
به این ترتیب برای زنجیره‌هایی که سود بیشتری دارند منابع بیشتری در اختیار الگوریتم قرار دارد.


%--------------------------------------------------------------------------appendix( مراجع و پیوست ها)
\chapterfont{\vspace*{-2em}\centering\LARGE}%

\appendix
\bibliographystyle{plain-fa}
\bibliography{references}
\include{appendix1}
%--------------------------------------------------------------------------dictionary(واژه نامه ها)
%اگر مایل به داشتن صفحه واژه‌نامه نیستید، خط زیر را غیر فعال کنید.
\parindent=0pt
%
\chapter*{واژه‌نامه‌ی فارسی به انگلیسی}
\pagestyle{style9}

\addcontentsline{toc}{chapter}{واژه‌نامه‌ی فارسی به انگلیسی}
%%%%%%
\begin{multicols*}{2}

{\bf آ}
%%\vspace*{3mm}

\vspace*{3mm}
{\bf ب}
%%\vspace*{3mm}

\vspace*{3mm}
{\bf پ}
%%\vspace*{3mm}

\vspace*{3mm}
{\bf ت}
%%\vspace*{3mm}

\vspace*{3mm}
{\bf ث}
%%\vspace*{3mm}

\vspace*{3mm}
{\bf ج}
%%\vspace*{3mm}

\vspace*{3mm}
{\bf چ}
%%\vspace*{3mm}

\vspace*{3mm}
{\bf ح}
%%\vspace*{3mm}

\vspace*{3mm}
{\bf خ}
%%\vspace*{3mm}

\vspace*{3mm}
{\bf د}
%%\vspace*{3mm}

\vspace*{3mm}
{\bf ر}
%%\vspace*{3mm}

\vspace*{3mm}
{\bf ز}
%%\vspace*{3mm}

\vspace*{3mm}
{\bf س}
%%\vspace*{3mm}

\vspace*{3mm}
{\bf ص}
%%\vspace*{3mm}

\vspace*{3mm}
{\bf ض}
%%\vspace*{3mm}

\vspace*{3mm}
{\bf ط}
%%\vspace*{3mm}

\vspace*{3mm}
{\bf ظ}
%%\vspace*{3mm}

\vspace*{3mm}
{\bf ع}
%%\vspace*{3mm}

\vspace*{3mm}
{\bf ف}
%%\vspace*{3mm}

\farsiTOenglish{فراهم‌آورنده‌ی شبکه}{Network Provider}



\vspace*{3mm}
{\bf ک}
%%\vspace*{3mm}

\farsiTOenglish{کارکردهای مجازی شبکه‌ای}{Virtual Network Function}

\vspace*{3mm}
{\bf گ}
%%\vspace*{3mm}

\vspace*{3mm}
{\bf م}
%%\vspace*{3mm}

\farsiTOenglish{مجازی‌سازی کارکردهای شبکه}{Network Function Virtualization}


\vspace*{3mm}
{\bf ن}
%%\vspace*{3mm}

\vspace*{3mm}
{\bf و}
%%\vspace*{3mm}

{\bf ه}
%%\vspace*{3mm}

{\bf ی}
%%\vspace*{3mm}


\end{multicols*}%
%%%%%%
\chapter*{واژه‌نامه}
\pagestyle{style9}
\lhead{\thepage}\rhead{واژه‌نامه}
\addcontentsline{toc}{chapter}{واژه‌نامه}

\englishTOfarsi{Interface}{رابط}

\englishTOfarsi{Application}{کاربرد}

\englishTOfarsi{Optimality Gap}{شکاف بهینه}

\englishTOfarsi{Edge}{یال}

\englishTOfarsi{Node}{گره}

\englishTOfarsi{Virtualization}{مجازی‌سازی}

\englishTOfarsi{Constraint}{محدودیت}

\englishTOfarsi{Heuristic}{مکاشفه‌ای}

\englishTOfarsi{Offline}{برون‌خط}

\englishTOfarsi{Online}{برخط}

\englishTOfarsi{Image}{تصویر}

\englishTOfarsi{Classifier}{دسته‌بند}

\englishTOfarsi{Forwarding}{جلورانی}

\englishTOfarsi{Function}{کارکرد}

\englishTOfarsi{Pod}{غلاف}

\englishTOfarsi{Aggregation}{تجمعی}

\englishTOfarsi{Edge}{لبه}

\englishTOfarsi{Instance}{نمونه}

\englishTOfarsi{Feasible Set}{مجموعه امکان‌پذیر}
%--------------------------------------------------------------------------index(نمایه)
%اگر مایل به داشتن صفحه نمایه نیستید، خط زیر را غیر فعال کنید.
\pagestyle{style7}
\printindex
\pagestyle{style7}
%کلمات کلیدی انگلیسی
% \latinkeywords{NFV, SFC, Optimization, ILP}
%چکیده انگلیسی

\en-abstract{
In the old times, Network providers use hardware network functions to create their service chains, but a change in this manner is difficult and may cause many service distribution. SFC and NFV is the solution to this difficulty. By using SFC and NFV, providers can provision chains dynamically and then change them in runtime. One of the main requirements is management and monitoring for the chains.
In this research, we consider the chain acceptance problem subject to management resources. In the first step, we formulate
problem with ILP and then implement it in CPLEX framework. As we know, ILP problems are NP-Hard, so we need an NP solution to the problem. In this research, we create a heuristic algorithm and compare its result with the optimal solution. In the end, the heuristic solution produces near-optimal results in the polynomial time.
}
%%%%%%%%%%%%%%%%%%%%% کدهای زیر را تغییر ندهید.

\newpage
\thispagestyle{empty}
\begin{latin}
\section*{\LARGE\centering Abstract}

\een-abstract

\vspace*{.5cm}
{\large\textbf{Key Words:}}\par
\vspace*{.5cm}
\elatinkeywords
\end{latin}

% در این فایل، عنوان پایان‌نامه، مشخصات خود و چکیده پایان‌نامه را به انگلیسی، وارد کنید.
%%%%%%%%%%%%%%%%%%%%%%%%%%%%%%%%%%%%
\baselineskip=.6cm
\begin{latin}

\latinfaculty{Department of Computer Engineering \& Information Technology}


\latintitle{Virtualized Network Service Function Chaining Subject to Management Resource Constraint}


\firstlatinsupervisor{Prof.\ Bahador Bakhshi}

%\secondlatinsupervisor{Second Supervisor}

% \firstlatinadvisor{Dr. }

%\secondlatinadvisor{Second Advisor}

\latinname{Parham}

\latinsurname{Alvani}

\latinthesisdate{September 2019}

\latinvtitle{}
\end{latin}
\end{document}