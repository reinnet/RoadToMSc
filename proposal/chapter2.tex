\chapter{تعریف مساله}
پذیرفتن بیشترین تقاضای زنجیره‌ کارکرد سرویس با در نظر گرفتن نیاز هر نمونه کارکرد مجازی شبکه به یک \lr{VNFM}.
همانطور که در مستند \cite{} نیز آمده است، نیاز است که هر یک نمونه‌های کارکردهای مجازی شبکه
توسط حداقل یک \lr{NNFM} مدیریت شوند.
در این مساله قصد داریم مساله پذیریش تقاضاهای زنجیره‌های کارکرد سرویس را با نظر گرفتن این نیازمندی در کنار
نیازمندی‌های پردازشی و پهنای‌باند هر یک از تقاضاها حل کنیم.
در ادامه به صورت تیتروار شرایط مساله را بررسی می‌کنیم:

\begin{itemize}
    \item توپولوژی زیرساخت شامل پنهای باند لینک‌ها و ظرفیت \lr{NFVI-PoP}ها موجود است.
    \item \lr{n} تقاضای زنجیره‌ کارکرد سرویس به صورت کامل و از پیش مشخص شده داریم.
    \item هر تقاضا شامل نوع و تعداد نمونه‌های مجازی و پنهای باند لینک‌های مجازی می‌باشد.
    \item \lr{F} نوع کارکرد مجازی شبکه تعریف شده است که هر یک مقدار مشخصی از حافظه را مصرف می‌کنند.
    \item تعداد پردازنده‌هایی که به هر نمونه تخصیص می‌یابد با توجه به ترافیک ورودی نمونه مشخص می‌شود.
    \item نمونه‌ها بین زنجیره‌ها به اشتراک گذاشته نمی‌شوند.
    \item محدودیت ظرفیت لینک‌ها
    \item محدودیت توان پردازش سرورهای فیزیکی با توجه به میزان حافظه و تعداد پردازنده‌ها
    \item برای سادگی مساله برای هر زنجیره یک \lr{VNFM} تخصیص می‌دهیم.
    \item \lr{VNFM}ها می‌توانند بین زنجیره به اشتراک گذاشته شوند.
    \item هر نمونه از \lr{VNFM}ها می‌تواند تعداد مشخصی از نمونه‌های کارکرد مجازی شبکه را سرویس دهد. 
    \item برای ارتباط میان هر نمونه از \lr{VNFM}ها و \lr{VNF}ها پهنای باند مشخصی رزرو می‌گردد.
    \item بر روی هر \lr{NFVI-PoP} حداکثر یک نمونه \lr{VNFM} مستقر می‌گردد.
\end{itemize}

استفاده از \lr{VNFM} در این ادبیات موضوعی بسیار جدید می‌باشد
و برای اولین بار می‌باشد که جایگذاری \lr{VNFM} در کنار جایگذاری
زنجیره‌های کارکرد سرویس مدنظر قرار داده می‌شود.

اگر جایگذاری \lr{VNFM}ها به صورت غیر برنامه‌ریزی شده صورت بپذیرد
ممکن است به تاخیرهای غیرقابل تحمل منجر شده و به این ترتیب تاثیر منفی بر روی کارآیی سیستم
داشته باشد.