\chapter{کارهای مرتبط}
در \cite{Eramo2016}
نویسندگان قصد دارند با در نظر گرفتن محدودیت ظرفیت لینک‌ها و محدودیت پردازشی نودها
بیشترین تعداد زنجیره‌ی کاکرد را بپذیرند. برای این کار یک مساله‌ی \lr{ILP}
طراحی می‌کنند و ثابت می‌کنند که این مساله \lr{NP-Hard} می‌باشد.
در این مقاله وجود \lr{VNFM} برای زنجیره‌ها در نظر گرفته نشده است.

در \cite{AbuLebdeh2017}
نویسندگان استفاده از \lr{VNFM} را مدنظر قرار داده‌اند
. در این مقاله فرض شده است که جایگذاری \lr{SFC}ها صورت گرفته است
و می‌خواهیم \lr{VNFM}ها را به گونه‌ای استقرار دهیم
که با رعایت شدن نیازمندی‌های کارآیی، هزینه‌ی عملیاتی سیستم حداقل شود.
مساله مطرح شده به صورت \lr{ILP} مدلسازی می‌شود.
این مقاله هزینه‌ی عملیاتی سیستم را تحت چهار عنوان دسته‌بندی می‌کند:
هزینه‌ی مدیریت چرخه‌ی زندگی، هزینه‌ی منابع محاسباتی، هزینه‌ی مهاجرت و هزینه‌ی بازنگاشت.
در این مقاله فرض می‌شود که هر نمونه از \lr{VNFM}ها می‌تواند به تعداد مشخصی از نمونه‌های \lr{VNF}
سرویس‌دهی کند و این سرویس‌دهی به نوع نمونه وابسته نیست.
این مقاله محدودیت‌های پردازشی و ظرفیتی را مدنظر قرار می‌دهد.

در \cite{Ghaznavi2017}
نویسندگان سه مرحله برای عملیات جایگذاری زنجیره‌های کارکرد سرویس معرفی می‌کنند:
\begin{itemize}
    \item انتخاب
    \item جابگذاری
    \item مسیریابی
\end{itemize}