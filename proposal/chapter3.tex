\chapter{نگارش صحيح}
%\thispagestyle{empty}

\section{مقدمه}

فصل مقدمه یک پایان نامه، با بیان نیاز موضوع، تعريف مسئله و اهمیت آن در یک یا چند بند (پاراگراف) آغاز مي‌شود\footnote{شروع مقدمه نبايد چنان طولاني باشد كه هدف اصلي را تحت‌ تاثير قرار دهد.}  و با مرور پيشينه موضوع (سابقه کارهای انجام‌شده پیشین که ارتباط مستقیمی با مسئله مورد بررسی دارند) ادامه مي‌يابد. سپس در یک یا دو بند توضیح داده مي‌شود كه در این پایان نامه، چه ديدگاه يا راهكار جدیدي نسبت به مسئله (موضوع) مورد بررسي وجود دارد. به‌عبارت دیگر نوآوری‌ها به‌صورت کاملاً شفاف و صریح بیان می‌شود. در ادامه ممکن است به نتايج بدست‌آمده نیز به‌طور مختصر و کلی اشاره ‌شود. در آخرین بند از مقدمه به محتواي فصل‌هاي بعدي پایان نامه به‌اختصار اشاره مي‌شود.\\
برای مشاهده دستورالعمل کامل دانشگاه صنعتی امیرکبیر(پلی تکنیک تهران) به \cite{zakeri} یا به سایت
%
\href{http://library.aut.ac.ir/Thesis%20Guide}{کتابخانه دانشگاه صنعتی امیرکبیر(پلی تکنیک تهران)}%
مراجعه نمایید.

نگارش صحيح يک پایان نامه در فهم آسان آن بسيار موثر است. در اين فصل مهمترین قواعد نگارشی که باید مورد توجه جدی نگارنده قرار گیرد، به اختصار بیان می‌شود. اين قواعد را مي‌توان در محورهای اصلی زير دسته‌بندی کرد:
\begin{itemize}
\item
فارسی نویسی
\item
رعایت املای صحيح 
\item
رعایت قواعد نشانه‌گذاری
\end{itemize}
\section{فارسی نویسی}
در حد امکان سعی کنيد به جاي کلمات غير‌فارسی از معادل فارسی آنها استفاده کنيد، به‌ويژه در مواردی که معادل فارسی مصطلح و رايج است‌.‌ به‌طور مثال استفاده از کلمه «لذا» به‌جای «برای همين» يا «به‌همين دليل» توجيهی ندارد‌. همچنين کلمه «پردازش» زيباتر از «پروسس» و معادل فارسی «ريز‌پردازنده» مناسب‌تر از «ميکروپروسسور» است‌.‌ در اين‌گونه موارد چنانچه احتمال عدم آشنايی خواننده با معادل فارسی وجود دارد، يا اصطلاح غير‌فارسی معمول‌تر است، در اولين ظهور کلمه فارسی، اصل غير‌فارسی آن به‌صورت پاورقي آورده شود‌.‌ اگر به‌ناچار بايد کلمات انگليسی در لابه‌لای جملات گنجانده شوند، از هر طرف يک فاصله بين آنها و کلمات فارسی پیش و پس از آنها در‌نظر گرفته شود‌.‌ چنانچه در پایان نامه از مختصر‌نويسی استفاده شود، لازم است در اولين استفاده، تفصيل آن در پاورقي آورده شود‌.‌ 

\section{رعایت املای صحيح }
رعايت املاي صحيح فارسي به مطالعه و درک راحت‌تر کمک مي‌کند. همچنين در نوشته‌هاي فارسي بايد در حد امکان از همزه « ء، أ، ؤ، ة، إ، ئ» استفاده نشود‌.‌ به‌عنوان مثال «اجزاء هواپیما» و «آئين نگارش» ناصحیح، اما «اجزاي هواپیما» و «آيين نگارش» صحيح هستند.‌
\section{رعایت قواعد نشانه‌گذاری}
منظور از نشانه‌گذاري به‌کار‌بردن علامت‌ها و نشانه‌هايي است که خواندن و فهم درست یک جمله را ممکن و آسان مي‌کند. در ادامه نشانه‌هاي معمول و متداول در زبان فارسي و موارد کاربرد آنها به اختصار معرفی می‌شوند.

\subsection{ويرگول}
ويرگول نشانه ضرورت یک مکث کوتاه است و در موارد زير به‌کار مي‌رود:
\begin{itemize}
\item
در ميان دو کلمه که احتمال داده شود خواننده آنها را با کسره اضافه بخواند، يا نبودن ويرگول موجب بروز اشتباه در خواندن جمله شود.
\item
در موردي که کلمه يا عبارتي به‌‌‌‌عنوان توضيح، در ضمن یک جمله آورده شود. مثلاً برای کنترل وضعیت فضاپیماها، به‌دلیل آن‌که در خارج از جو هستند، نمی‌توان از بالک‌های آیرودینامیکی استفاده کرد.
\item
جدا‌کردن بخش‌هاي مختلف يک نشاني يا یک مرجع
\item
موارد دیگر از این قبیل
\end{itemize}
پیش از ويرگول نبايد فاصله گذاشته شود و پس از آن يک فاصله لازم است و بيشتر از آن صحیح نیست.
\subsection{نقطه}
نقطه نشانه پایان یک جمله است. پیش از نقطه نبايد فاصله گذاشته شود و پس از آن يک فاصله لازم است و بيشتر از آن صحیح نیست.
\subsection{دونقطه}
موارد کاربرد دونقطه عبارتند از:
\begin{itemize}
\item
پيش از نقل قول مستقيم
\item
پيش از بيان تفصيل مطلبي که به اجمال به آن اشاره شده‌است.
\item
پس از واژه‌اي که معني آن در برابرش آورده و نوشته مي‌شود.
\item
پس از کلمات تفسير‌کننده از قبيل «يعني» و ...
\end{itemize}
پیش از دونقطه نبايد فاصله گذاشته شود و پس از آن يک فاصله لازم است و بيشتر از آن صحیح نیست.
\subsection{گیومه}
موارد کاربرد گیومه عبارتند از:
\begin{itemize}
\item
وقتي که عين گفته يا نوشته کسي را در ضمن نوشته و مطلب خود مي‌آوريم. 
\item
در آغاز و پايان کلمات و اصطلاحات علمي و يا هر کلمه و عبارتي که بايد به‌صورت ممتاز از قسمت‌هاي ديگر نشان داده شود.
\item
در ذکر عنوان مقاله‌ها، رساله‌ها، اشعار، روزنامه‌ها و ...
\end{itemize}
\subsection{نشانه پرسشی}
پیش از «؟» نبايد فاصله گذاشته شود و پس از آن يک فاصله لازم است و بيشتر از آن صحیح نیست.
\subsection{خط تیره}
موارد کاربرد خط تیره عبارتند از:
\begin{itemize}
\item
جدا‌کردن عبارت‌هاي توضيحي، بدل، عطف بيان و ...
\item
به‌جاي حرف اضافه «تا» و «به» بين تاريخ‌ها، اعداد و کلمات
\end{itemize}
\subsection{پرانتز}
موارد کاربرد پرانتز عبارتند از:
\begin{itemize}
\item
به‌معني «يا» و «يعني» و وقتي که یک کلمه يا عبارت را براي توضيح بيشتر کلام بياورند.
\item
وقتي که نويسنده بخواهد آگاهي‌هاي بيشتر (اطلاعات تکميلي) به خواننده عرضه کند.
\item
براي ذکر مرجع در پايان مثال‌ها و شواهد.
\end{itemize}
نکته: بین کلمه یا عبارت داخل پرانتز و پرانتز باز و بسته نباید فاصله وجود داشته باشد.
\section{جدا یا سرهم نوشتن برخی کلمات}
تقريباً تمامي کلمات مرکب در زبان فارسي بايد از هم جدا نوشته شوند؛ به استثناي صفات فاعلي مانند «عملگر»، «باغبان» و يا «دانشمند» و کلماتي نظير «اينکه»، «آنها». در ادامه به نمونه‌هايي از مواردي که بايد اجزاي يک کلمه جدا، اما بدون فاصله نوشته شوند، اشاره مي‌شود‌:
\begin{enumerate}
\item
در افعال مضارع و ماضی استمراری که با «می» شروع می‌شوند، لازم است که در عين جدا نوشتن، «می» از بخش بعدي فعل جدا نيافتد‌.‌ برای اين منظور بايد از «فاصله متصل» استفاده و «می» در اول فعل با \lr{SS}\LTRfootnote{Shift+Ctrl+@} از آن جدا شود.‌ به‌طور مثال «می‌شود» به‌جاي «می شود». 
\item
	«ها»ی جمع بايد از کلمه جمع بسته‌شده جدا نوشته شود؛ مگر در برخی کلمات مانند «آنها». اين امر در مورد کلمات غير‌فارسي که وارد زبان فارسي شده‌اند و با حرف «ها» جمع بسته می‌شوند، مانند «کانال‌ها» يا «فرمول‌ها» مورد تاکيد است.
\item
	حروف اضافه مانند «به» وقتي به‌صورت ترکيب ثابت همراه کلمه پس از خود آورده می‌شوند، بهتر است با \lr{SS} از آن جدا شوند‌.‌ مانند «به‌صورت»، «به‌عنوان» و «به‌‌‌لحاظ»‌.‌ لازم به ذکر است هنگامی که حرف اضافه «به» با کلمه پس از خود معناي قيدي داشته باشد، مثل «بشدت» يا «بسادگي»، بهتر است که به‌صورت چسبيده نوشته شود‌.
\item
	کلمات فارسی نبايد با قواعد عربی جمع بسته شوند؛ پس «پيشنهادها» صحيح و «پيشنهادات» اشتباه است‌.‌
\item
	اسم‌ها و صفت‌هاي دو‌قسمتي مثل «خط‌چين» و «نوشته‌شده» با \lr{SS} از هم جدا می‌شود‌.‌
\item
	شناسه‌ها با \lr{SS} از کلمه اصلي جدا می‌شود‌.‌ مثل «شده‌اند»‌ و «شده‌است». 
\item
	‌ «است» هنگامی که نقش شناسه را داشته باشد توسط \lr{SS} از قسمت اصلي جدا می‌شود‌.‌ مانند «گفته‌است»‌.
\item
	بند پیشین نبايد باعث افراط در استفاده از فاصله متصل شود. مثلاً عبارت «نوشته می‌شود‌« صحيح و عبارت «نوشته‌می‌شود» ناصحیح است. 
\item
	فعل‌هاي دو‌کلمه‌اي که معناي اجزاي آنها کاملاً با معناي کل متفاوت است، بهتر است که با \lr{SS} از هم جدا ‌شوند‌.‌
\item
	کلمات مرکب مثل کلمه «دوکلمه‌اي» در عبارت «فعل‌هاي دوکلمه‌اي» و «يادداشت‌برداري».
\item
	مصدرهاي دو قسمتي با \lr{SS} از هم جدا می‌شوند‌.‌ مثل «ذوب‌کردن» و «واردکردن»‌.
\item
	 صفات تفضيلي مثل « آسان‌تر».
\end{enumerate}

