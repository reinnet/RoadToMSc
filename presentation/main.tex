% !TeX TS-program = xelatex

\documentclass{beamer}
%Set the slide theme
%Change to meet your taste
% Madrid, Copenhagen, Berlin, ... works
\usetheme{metropolis}

\usepackage{multicol}

\usepackage{ragged2e} % who justifies the text
\usepackage{xecolor}
\usepackage{amsmath}
%\usefonttheme[onlymath]{serif} %Change the math font

\usepackage{multirow}
\usepackage{tabularx}
\usepackage{booktabs}
\usepackage[sorting=none,style=ieee]{biblatex}
\addbibresource{references.bib}
\usepackage{xepersian}
\settextfont[Path=fonts/]{Parastoo}

%---------------------------------------------------------------------------------
% Seetings to force Beamer works with Xepersian and RTL typesetting
%-------------------------------------------------------------------------------
%\raggedleft

% For right to left lists (itemize and enumerate)
\makeatletter
\newcommand{\RTList}{\raggedleft\rightskip\@totalleftmargin}
\makeatother
% Correct the bullet for RTL texts
\setbeamertemplate{itemize item}{\scriptsize\raise1.25pt%
 \hbox{\donotcoloroutermaths$\blacktriangleleft$}} 

% To force beamer use numbering in captions
\setbeamertemplate{caption}[numbered]{}% Number float-like environments

\setbeamertemplate{footline}[frame number]

\setbeamertemplate{headline}
{
    \begin{beamercolorbox}{section in head/foot}
        \vskip2pt\insertnavigation{\paperwidth}\vskip2pt
    \end{beamercolorbox}
}


%---------------------------------------------------------------------------------
\title{
    زنجیره‌سازی کارکردهای مجازی سرویس شبکه با در نظر گرفتن محدودیت منابع مدیریتی
}
\subtitle{مهندسی فناوری اطلاعات - شبکه‌های کامپیوتری}
\author{پرهام الوانی}
\institute{
    دانشکده مهندسی کامپیوتر و فناوری اطلاعات\\دکتر بهادر بخشی
    \\
    \\
    \includegraphics[width=1cm]{images/logo}
}
\date{شهریور ۱۳۹۸}

\begin{document}
\begin{persian}

%------------------------------------------
% Title frame (0)
%------------------------------------------
\begin{frame}
    \titlepage{}
\end{frame}

% To adjust the paragraphs in RTL
\everypar{\rightskip\rightmargin}
%-------------------------------------------------------------------------------
\begin{frame}{فهرست}
    \tableofcontents
\end{frame}
%-------------------------------------------------------------------------------
\begin{frame}{}
    \section{مقدمه}
\end{frame}
%-------------------------------------------------------------------------------
\begin{frame}{مقدمه}
    \begin{itemize}\RTList{}
        \justifying
        \item عدم انعطاف‌پذیری معماری فعلی شبکه
        \item در مجازی‌سازی کارکرد شبکه با استفاده از مجازی‌سازی منابع، می‌توان کارکردها را بر روی سرورهای استاندارد اجرا کرد
        و بهره‌وری منابع را فزایش داده و هزینه‌های انرژی را کاهش داد.
        \item زنجیره سازی کارکرد سرویس نیز امکان ایجاد زنجیره‌ای از کارکردها را به صورت پویا فراهم می‌کند.
    \end{itemize}
\end{frame}
%-------------------------------------------------------------------------------
\begin{frame}{مقدمه}
    \begin{itemize}\RTList{}
        \justifying
        \item با توجه به جداسازی زیرساخت از نرم‌افزار کارکردهای شبکه، نیاز به هماهنگی میان آن‌ها ایجاد شده است.
        \item به صورت کلی تفاوت‌هایی که با توجه به فرآیند مجازی‌سازی کارکردهای شبکه ایجاد شده‌اند را  می‌توان
        به ترتیب زیر دسته‌بندی نمود:
        \begin{itemize}\RTList{}
            \item زیرساخت مجازی‌سازی شده
            \item کارکردهای شبکه‌ای مجازی‌سازی شده
            \item سرویس‌های شبکه‌ای
        \end{itemize}
    \end{itemize}
\end{frame}
%-------------------------------------------------------------------------------
\begin{frame}{مقدمه}
    \begin{center}\begin{figure}
        \includegraphics[scale=0.5]{images/nfv-arch.png}
        \caption{معماری سطح بالای مجازی‌سازی کارکردهای شبکه}
    \end{figure}\end{center}
\end{frame}
%-------------------------------------------------------------------------------
\begin{frame}{مقدمه}
    \begin{itemize}\RTList{}
        \item \lr{NFVO} وظیفه‌ی استقرار زنجیره‌های کارکرد سرویس را برعهده دارد.
        \item \lr{VNFM} مسئول چرخه‌ی زندگی کارکردهای مجازی شبکه می‌باشد.
        \item چرخه‌ی زندگی هر کارکرد مجازی شامل عملیات‌هایی همچون نمونه‌سازی، مقیاس‌کردن، به‌روزرسانی و پایان دادن می‌باشد.
        \item هر نمونه از کارکردهای مجازی شبکه نیاز دارد تحت مدیریت یکی از \lr{VNFM}های موجود در شبکه باشد.
    \end{itemize}
\end{frame}
%-------------------------------------------------------------------------------
\begin{frame}{چالش‌ها}
    \begin{columns}
        \begin{column}{0.6\textwidth}
            \includegraphics[scale=0.35]{images/embedding.png}
        \end{column}
        \begin{column}{0.4\textwidth}
            \begin{itemize}\RTList{}
                \item مدیریت و هماهنگی
                \item مصرف بهینه‌ی انرژی
                \item تخصیص منابع به کارکردهای مجازی
                \item مسیریابی زنجیره‌های کارکرد سرویس
                \item پذیرش زنجیره‌های کارکرد سرویس
                \item به روزرسانی و مقیاس کردن کارکردهای مجازی سرویس
            \end{itemize}
        \end{column}
    \end{columns}
\end{frame}
%-------------------------------------------------------------------------------
\begin{frame}{}
    \section{سابقه‌ی کارها}
\end{frame}
%-------------------------------------------------------------------------------
\begin{frame}{سابقه‌ی کارها}
    \fontsize{6pt}{7.2}\selectfont
    \begin{table}[h]
        \caption{مقایسه مقالات پذیرش زنجیره‌های کارکرد سرویس}
        \vspace{0.5cm}
        \begin{tabularx}{\textwidth}{XXXXXXXXXXXXXXXXX}
            \toprule
            منبع &
            \multicolumn{4}{X}{منابع تخصیص یافته} &
            \multicolumn{2}{X}{محدودیت ظرفیت پردازشی نمونه} &
            \multicolumn{2}{X}{برخط یا برون خط} &
            \multicolumn{2}{X}{نگاشت کارکرد و لینک} &
            \multicolumn{2}{X}{انتساب کارکرد} &
            \multicolumn{2}{X}{اشتراک نمونه} &
            \multicolumn{2}{X}{تخصیص \lr{VNFM}} \\
            \midrule
            \lr{\#} &
            \lr{other} &
            \lr{MEM} &
            \lr{BW} &
            \lr{CPU} &
            دارد &
            ندارد &
            برخط &
            برون خط &
            کارکرد &
            لینک &
            یک نمونه &
            چند نمونه &
            دارد &
            ندارد &
            دارد &
            ندارد \\
            \midrule
            \cite{Eramo2016} &
            \lr{---} &
            \lr{---} &
            \checkmark&
            \checkmark&
            \lr{---}&
            \checkmark&
            \lr{---} &
            \checkmark&
            \checkmark&
            \checkmark&
            \checkmark&
            \lr{---} &
            \lr{---} &
            \checkmark&
            \lr{---} &
            \checkmark\\
            \midrule
            \cite{Ghaznavi2017} &
            \lr{---} &
            \lr{---} &
            \checkmark&
            \checkmark&
            \checkmark&
            \lr{---} &
            \lr{---} &
            \checkmark&
            \checkmark&
            \checkmark&
            \lr{---} &
            \checkmark&
            \lr{---} &
            \checkmark&
            \lr{---} &
            \checkmark\\
            \midrule
            \cite{Huang2017} &
            \lr{---} &
            \lr{---} &
            \checkmark&
            \checkmark&
            \checkmark&
            \lr{---} &
            \lr{---} &
            \checkmark&
            \checkmark&
            \checkmark&
            \lr{---} &
            \checkmark&
            \lr{---} &
            \checkmark&
            \lr{---} &
            \checkmark\\
            \midrule
            \cite{AbuLebdeh2017} &
            \lr{VNFM capacity} &
            \lr{---} &
            \lr{---} &
            \lr{---} &
            \lr{---} &
            \checkmark&
            \checkmark&
            \lr{---} &
            \lr{---} &
            \checkmark &
            \lr{---} &
            \lr{---} &
            \lr{---} &
            \lr{---} &
            \checkmark&
            \lr{---} \\
            \midrule
            پژوهش حاضر &
            \lr{---} &
            \checkmark&
            \checkmark&
            \checkmark&
            \checkmark&
            \lr{---} &
            \lr{---} &
            \checkmark&
            \checkmark&
            \checkmark&
            \checkmark&
            \lr{---}&
            \lr{---}&
            \checkmark&
            \checkmark&
            \lr{---}\\
            \bottomrule
        \end{tabularx}
    \end{table}
\end{frame}
%-------------------------------------------------------------------------------
\begin{frame}{سابقه‌ی کارها}
    \begin{itemize}\RTList{}
        \item این مفاله مساله‌ی جایگذاری \lr{VNFM}ها را مطرح می‌کند.
        \item این مقاله فرض می‌کند زنجیره‌های جایگذاری شده‌اند و هر در بازه‌ی زمانی می‌توانند بازنگاشت شوند.
        \item این مساله قصد دارد با در نظر گرفتن هزینه‌های عملیاتی مساله‌ی بازنشگات \lr{VNFM}ها در بازه‌های زمانی را حل کند.
    \end{itemize}
    \begin{latin}
    \fullcite{AbuLebdeh2017}
    \end{latin}
\end{frame}
%-------------------------------------------------------------------------------
\begin{frame}{}
    \section{تعریف مساله}
\end{frame}
%-------------------------------------------------------------------------------
\begin{frame}{تعریف مساله}
    \par
    بیشینه‌سازی سود حاصل از پذیرش زنجیره‌های کارکرد سرویس با در نظر گرفتن کارکرد سرویس با در نظر گرفتن نیاز برخی از نمونه‌های کارکرد
    مجازی شبکه به \lr{VNFM}.
\end{frame}‌
%-------------------------------------------------------------------------------
\begin{frame}{تعریف مساله}
    \begin{itemize}\RTList{}
        \justifying
        \item توپولوژی زیرساخت شامل پنهای باند لینک‌ها و ظرفیت \lr{NFVI-PoP}ها، موجود است.
        \item n تقاضای زنجیره‌ کارکرد سرویس به صورت کامل و از پیش مشخص شده داریم.
        \item هر تقاضا شامل نوع و تعداد نمونه‌های مجازی، پنهای باند لینک‌های مجازی و توپولوژی
        نمونه‌های مجازی می‌باشد.
    \end{itemize}
\end{frame}
%-------------------------------------------------------------------------------
\begin{frame}{تعریف مساله}
    \justifying
    \begin{itemize}\RTList{}
        \item نمونه‌ها بین زنجیره‌ها به اشتراک گذاشته نمی‌شوند.
        \item محدودیت ظرفیت لینک‌ها
        \item محدودیت توان پردازش سرورهای فیزیکی با توجه به میزان حافظه و تعداد پردازنده‌ها
        \item برخی از سرور‌های فیزیکی نمی‌توانند سرور‌های فیزیکی مشخصی را مدیریت کنند.
        \item برخی از سرورهای فیزیکی توانایی پشتیبانی از کارکردهای مجازی را ندارد.
        \item تنها برخی از نمونه‌های کارکردهای مجازی نیاز به مدیریت دارند.
    \end{itemize}
\end{frame}
%-------------------------------------------------------------------------------
\begin{frame}{تعریف مساله}
    \justifying
    \begin{itemize}\RTList{}
        \item برای مدیریت یکدست و آسان‌تر زنجیره‌ها و در عین حال جمع‌آوری راحتر خطاها، برای هر زنجیره یک \lr{VNFM} تخصیص می‌دهیم.
        \item \lr{VNFM}ها می‌توانند بین زنجیره به اشتراک گذاشته شوند.
        \item هر نمونه از \lr{VNFM}ها می‌تواند تعداد مشخصی از نمونه‌های کارکرد مجازی شبکه را سرویس دهد. 
        \item برای ارتباط میان هر نمونه از \lr{VNFM}ها و \lr{VNF}ها پهنای باند مشخصی رزرو می‌گردد.
        \item در صورتی که \lr{NFVI-PoP} بتواند از \lr{VNFM} پشتیبانی نماید،
        می‌توان به هر تعداد که ظرفیت آن اجازه می‌دهد بر روی آن \lr{VNFM} نصب نمود.
    \end{itemize}
\end{frame}
%-------------------------------------------------------------------------------
\begin{frame}{چالش‌ها و نوآوری‌های مساله}
    \begin{itemize}\RTList{}
        \item در نظر گرفتن نیازمندی نمونه‌های کارکرد مجازی به یک \lr{VNFM}
        \item در نظر گرفتن نیازمندی تاخیر برای لینک‌های مدیریتی
        \item تخصیص منابع مدیریتی به زنجیره‌ها و مسیریابی ارتباط مدیریتی
        \item جایگذاری و مسیریابی توامان زنجیره‌های کارکرد سرویس
    \end{itemize}
\end{frame}
%-------------------------------------------------------------------------------
\begin{frame}{معیار و نحوه‌ی ارزیابی}
    \begin{itemize}\RTList{}
        \item مدل‌سازی مساله
        \item حل مساله‌ی بهینه در ابعاد کوچک
        \item پیاده‌سازی راه‌حل مکاشفه‌ای
        \item معیار مقایسه این راه حل سود حاصل از پذیرش تقاضاهای زنجیره‌های کارکرد سرویس می‌باشد.
        \item مقایسه‌ی نتایج راه‌حل مکاشفه‌ای با جواب بهینه
    \end{itemize}
\end{frame}
%-------------------------------------------------------------------------------
\begin{frame}{}
    \section{فرمول‌بندی و مدل‌سازی ریاضی مساله}
\end{frame}
%-------------------------------------------------------------------------------
\begin{frame}{فرمول‌بندی}
    \par پارامترهای مساله
    \begin{center}\begin{latin}\begin{tabular}{|c|p{5cm}|}
        \hline
        \(memory(k)\) & required RAM of VNF instance with type \(k\) in GB \\
        \hline
        \(core(k)\) & required CPU cores of VNF instance with type \(k\) \\
        \hline
        \(\hat{memory}\) & required RAM of VNFM in GB \\
        \hline
        \(\hat{core}\) & required CPU cores of VNFM \\
        \hline
        \(capacity\) & maximum number of VNF instances that VNFM can handle \\
        \hline
        \(len(h)\) & number of VNF instances in \(h\)th SFC request \\
        \hline
    \end{tabular}\end{latin}\end{center}
\end{frame}
%-------------------------------------------------------------------------------
\begin{frame}{فرمول‌بندی}
    \par پارامترهای مساله
    \begin{center}\begin{latin}\begin{tabular}{|c|p{5cm}|}
        \hline
        \(type(v, k)\) & assuming the value 1 if the VNF instance \(v\) has type \(k\)  \\
        \hline
        \(bandwidth(u, v)\) & required bandwidth in link from VNF instance \(u\) to \(v\) \\
        \hline
        \(\hat{bandwidth}\) & required bandwidth in managmeent link \\
        \hline
        \(radius\) & maximum neighborhood distance for instance management \\
        \hline
    \end{tabular}\end{latin}\end{center}
\end{frame}
%-------------------------------------------------------------------------------
\begin{frame}{فرمول‌بندی}
    \par پارامترهای مساله
    \begin{center}\begin{latin}\begin{tabular}{|c|p{5cm}|}
        \hline
        \(licenseFee\) & VNFM license fee that must pay for each VNFM \\
        \hline
        \(vnfSupport(w)\) & assuming the value 1 if the physical server \(w\) can support VNF instances \\
        \hline
        \(isManageable(k)\) & assuming the value 1 if the type \(k\) needs a manager \\
        \hline
        \(notManagableBy(w1, w2)\) & assuming the value 1 if the physical server \(w1\) cannot manage by physical server \(w2\) \\
        \hline
    \end{tabular}\end{latin}\end{center}
\end{frame}
%-------------------------------------------------------------------------------
\begin{frame}{فرمول‌بندی}
    \par
    متغیرهای تصمیم‌گیری
    \begin{latin}\begin{tabular}{c p{10cm}}
        $x_h$ & binary variable assuming the value 1 if the $h$th SFC request is accepted; otherwise its value is zero \\
        $y_{wk}$ & the number of VNF instances of type $k$ that are used in server $w \in V_s^{PN}$ \\
        $z^k_{vw}$ & binary variable assuming the value 1 if the VNF node $v \in \cup_{i=1}^{T} V_{i, F}^{SFC}$ is served by the VNF instance of type k in the server $w \in V_s^{PN}$ \\
    \end{tabular}\end{latin}
\end{frame}
%-------------------------------------------------------------------------------
\begin{frame}{فرمول‌بندی}
    \par
    متغیرهای تصمیم‌گیری
    \begin{latin}\begin{tabular}{c p{10cm}}
        $\bar{y}_w$ & the number of VNFMs that are used in server $w \in V_s^{PN}$\\
        $\bar{z}_{hw}$ & binary variable assuming the value 1 if $h$th SFC is assigned to VNFM on server $w \in V_s^{PN}$\\
    \end{tabular}\end{latin}
\end{frame}
%-------------------------------------------------------------------------------
\begin{frame}{فرمول‌بندی}
    \par
    هدف اصلی مساله پذیریش بیشترین تعداد تقاضا می‌باشد.
    در اینجا فرض می‌کنیم پذیرش هر تقاضا سودی منحصر به فرد خواهد داشت.
    بنابراین تابع هدف به شکل زیر می‌باشد:
    \begin{latin}\begin{align}
    \max \sum_{h=1}^T x_h
    \end{align}\end{latin}
\end{frame}
%-------------------------------------------------------------------------------
\begin{frame}{فرمول‌بندی}
    \par
    محدودیت حافظه نودها
    \begin{latin}\begin{align}
    \sum_{k=1}^F y_{wk} memory(k) + \bar{y_w} \bar{memory} \le N_{ram}^{PN}(w)
    \quad
    \forall w \in V_s^{PN}
    \end{align}\end{latin}
    \par
    محدودیت تعداد پردازنده‌های نودها
    \begin{latin}\begin{align}
    \sum_{k=1}^F y_{wk} core(k) + \bar{y_w} \bar{core} \le N_{core}^{PN}(w)
    \quad
    \forall w \in V_s^{PN}
    \end{align}\end{latin}
\end{frame}
%-------------------------------------------------------------------------------
\begin{frame}{فرمول‌بندی}
    \par
    اگر \lr{VNF}, \lr{v}
    توسط \lr{VNF instance} نوع \lr{k}
    روی سرور \lr{w} سرویس شود می‌بایست
    \lr{VNF instance} نوع \lr{k}
    روی سرور \lr{w} فعال شود.
    توجه شود که
    اشتراک گذاری \lr{VNF}ها پشتیبانی نمی‌گردد.
    \begin{latin}\begin{align}
    \sum_{v \in \cup_{i=1}^T V_{i, F}^{SFC}} z_{vw}^k \le y_{wk}
    \quad
    \forall w \in V_s^{PN}, \forall k \in [1,\ldots, F]
    \end{align}\end{latin}
\end{frame}
%-------------------------------------------------------------------------------
\begin{frame}{فرمول‌بندی}
    \par
    اگر تقاضای \lr{h}ام پذیرفته شده باشد
    می‌بایست تمام \lr{VNF node}های آن‌
    سرویس شده باشند.
    یک \lr{VNF} حداکثر یکبار سرویس داده شود.
    \begin{latin}\begin{align}
        x_h = \sum_{k=1}^{F} \sum_{w \in V_{s}^{PN}} z_{vw}^{k}
        \quad
        \forall v \in V_{h,F}^{SFC}, \forall h \in [1,\ldots, T]
    \end{align}\end{latin}
\end{frame}
%-------------------------------------------------------------------------------
\begin{frame}{فرمول‌بندی}
    \par
    اگر تقاضای \lr{h}ام پذیرفته شده باشد
    می‌بایست توسط یک \lr{VNFM} سرویس شده باشد.
    توجه شود که این محدودیت اجازه‌ی تخصیص بیش از یک \lr{VNFM}
    به زنجیره نمی‌دهد.
    \begin{latin}\begin{align}
        x_h = \sum_{w \in V_{s}^{PN}} \bar{z}_{hw}
        \quad
        \forall h \in [1,\ldots, T]
    \end{align}\end{latin}
\end{frame}
%-------------------------------------------------------------------------------
\begin{frame}{فرمول‌بندی}
    \par
    اگر \lr{SFC}، \lr{i}
    توسط \lr{VNFM} روی سرور \lr{w}
    سرویس شود می‌بایست یک \lr{VNFM} سرور \lr{w}
    برای آن فعال شود.
    \begin{latin}\begin{align}
        \sum_{h=1}^T
        \bar{z}_{hw} \le \bar{y}_w
        \quad
        \forall w \in V_{s}^{PN}
    \end{align}\end{latin}
    \par
    محدودیت ظرفیت سرویس‌دهی \lr{VNFM}
    \begin{latin}\begin{align}
        \sum_{i=1}^{T} \bar{z}_{iw} * len(i) \le capacity
        \quad
        \forall w \in V_{s}^{PN}
    \end{align}\end{latin}
\end{frame}
%-------------------------------------------------------------------------------
\begin{frame}{فرمول‌بندی}
    \par
    اگر \lr{VNF}، \lr{v}
    توسط نمونه‌ای نوع \lr{k}
    روی سرور \lr{w} سرویس می‌شود می‌بایست
    این \lr{VNF} از نوع \lr{k}ام باشد.
    \begin{latin}\begin{align}
        z_{vw}^{k} \le type(v, k)
        \quad
        \forall w \in V_{s}^{PN},
        \forall k \in [1,\ldots, F],
        \forall v \in \cup_{i=1}^T V_{i, F}^{SFC}
    \end{align}\end{latin}
\end{frame}
%-------------------------------------------------------------------------------
\begin{frame}{فرمول‌بندی}
    \par
    متغیرهای تصمیم‌گیری
    \begin{latin}\begin{tabular}{c p{10cm}}
        $\tau^{(u,v)}_{ij}$ & binary variable assuming the value 1 if the virual link $(u,v)$ is routed on the physical network link $(i,j)$\\
        $\bar{\tau}^{v}_{ij}$ & binary variable assuming the value 1 if the management traffic of VNF node $v$ is routed on the physical network link $(i,j)$\\
    \end{tabular}\end{latin}
\end{frame}
%-------------------------------------------------------------------------------
\begin{frame}{فرمول‌بندی}
    \par
    \lr{Flow Conservation}
    \begin{latin}\begin{align}
        \sum_{(i,j) \in E^{PN}} \tau_{ij}^{(u,v)} - \sum_{(j,i) \in E^{PN}} \tau_{ji}^{(u,v)} = \sum_{k=1}^{F} z_{ui}^{k} - \sum_{k=1}^{F} z_{vi}^{k} \nonumber \\
        \forall i \in V_{S}^{PN}, (u,v) \in E_{h}^{SFC}, h \in [1,\ldots, T]
    \end{align}\end{latin}
\end{frame}
%-------------------------------------------------------------------------------
\begin{frame}{فرمول‌بندی}
    \par
    \lr{Flow Conservation}
    \begin{latin}\begin{align}
        \sum_{(i,j) \in E^{PN}} \bar{\tau}_{ij}^{v} - \sum_{(j,i) \in E^{PN}} \bar{\tau}_{ji}^{v} = \sum_{k=1}^{F} z_{vi}^{k} - \bar{z}_{hi} \nonumber \\
        \forall i \in V_{S}^{PN}, v \in V_{h, F}^{SFC}, h \in [1,\ldots, T]
    \end{align}\end{latin}
\end{frame}
%-------------------------------------------------------------------------------
\begin{frame}{فرمول‌بندی}
    \par
    محدودیت ظرفیت لینک‌ها
    \begin{latin}\begin{align}
        \sum_{v \in \cup_{i=1}^{T} V_{i,F}^{SFC}} \bar{\tau}_{ij}^{v} * \bar{bandwidth} + \sum_{(u,v) \in \cup_{i=1}^{T} E_{i}^{SFC}} \tau_{ij}^{(u,v)} * bandwidth(u,v) \le C_{ij} \nonumber \\
        \forall (i, j) \in E^{PN}
    \end{align}\end{latin}
\end{frame}
%-------------------------------------------------------------------------------
\begin{frame}{فرمول‌بندی}
    \par
    شعاع همسایگی تضمین می‌کند که زمان سرویس‌دهی توسط
    \lr{VNFM}ها
    در یک بازه مشخص (از نظر تعداد گام)
    خواهد بود.
    \begin{latin}\begin{align}
        \sum_{(i, j) \in E^{PN}} \bar{\tau}_{ij}^{v} \le radius
        \quad
        \forall v \in \cup_{i=1}^T V_{i, F}^{SFC}
    \end{align}\end{latin}
\end{frame}
%-------------------------------------------------------------------------------
\begin{frame}{مساله‌ی نمونه}
    \par
    زنجیره‌های زیر را به عنوان تقاضاها در نظر می‌گیریم.

    \begin{center}
        \includegraphics[scale=0.4]{../diagrams/chains.pdf}
    \end{center}
\end{frame}
%-------------------------------------------------------------------------------
\begin{frame}{مساله‌ی نمونه}
    \par
    فرض می‌کنیم مرکز داده‌ای دارای توپولوژی زیر می‌باشد.

    \begin{center}
        \includegraphics[scale=0.4]{../diagrams/topology.pdf}
    \end{center}
\end{frame}
%-------------------------------------------------------------------------------
\begin{frame}{}
    \section{راه‌حل پیشنهادی}
\end{frame}
%-------------------------------------------------------------------------------
\begin{frame}{راه‌حل پیشنهادی}
    \begin{itemize}\RTList{}
        \item مساله‌ی اصلی یک مساله‌ی \lr{NP-Hard} می‌باشد.
        \item برای حل مساله در زمان معقول برای ابعاد بزرگ نیاز به یک الگوریتم مکاشفه‌ای می‌باشد.
        \item از ایده‌ی الگوریتم \lr{\cite{Bari2015}} برای جایگذاری زنجیره‌ها شروع می‌کنیم.
    \end{itemize}
\end{frame}
%-------------------------------------------------------------------------------
\begin{frame}{\lr{JSD-MP}}
    \begin{itemize}\RTList{}
        \item \lr{Joint Service Deployment - Manager Placement}
        \item زنجیره‌ها را با استفاده از الگوریتم \lr{\cite{Bari2015}} جایگذاری می‌کنیم.
        \item در زمان انتخاب مجموعه‌ی امکان‌پذیر محدودیت‌های مساله را اعمال می‌کنیم.
        \item بعد از جایگذاری هر زنجیره \lr{VNFM} آن را انتخاب می‌کنیم.
        \item برای انتخاب \lr{VNFM} اولویت با نمونه‌هایی است که ظرفیت آن‌ها کامل استفاده نشده است.
        \item در بین \lr{VNFM}هایی که ظرفیت خالی دارند اولویت با نمونه‌هایی است که منابع پردازشی بیشتری دارند.
    \end{itemize}
\end{frame}
%-------------------------------------------------------------------------------
\begin{frame}{\lr{eJSD-MP}}
    \begin{itemize}\RTList{}
        \item الگوریتم پیشنهادی \lr{JSD-MP} زمان اجرای زیادی دارد که می‌توان آن را کاهش داد.
        \item الگوریتم پیشنهادی \lr{eJSD-MP} از برون‌خط بودن مساله استفاده نمی‌کند.
        \item برای استفاده از ویژگی برون‌خط بودن مساله زنجیره‌ها را بر اساس قیمت‌شان مرتب می‌کنیم.
        \item برای کاهش زمان اجرای الگوریتم نسب مشخصی از زنجیره‌ها را با الگوریتم \lr{first-fit} جایگذاری می‌کنیم.
    \end{itemize}
\end{frame}
%-------------------------------------------------------------------------------
\begin{frame}{}
    \begin{center}
        ارزیابی
    \end{center}
\end{frame}
%-------------------------------------------------------------------------------
\begin{frame}{ارزیابی}
    \par
    فرمول‌بندی ارائه شده بر روی نرم‌افزار
    \lr{cplex}
    که محصول شرکت \lr{IBM} بوده و برای حل مسائل برنامه‌ریزی خطی و ...
    استفاده می‌شود،
    به زبان جاوا پیاده‌سازی شده
    و تست گشت.

    \begin{center}
        \includegraphics[scale=0.4]{images/ibm-cplex.png}
    \end{center}
\end{frame}
%-------------------------------------------------------------------------------
\begin{frame}[shrink=25]{مراجع}
    \begin{latin}
    \printbibliography[heading=none]
    \end{latin}
\end{frame}
\end{persian}
\end{document}
