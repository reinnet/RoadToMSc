% !TeX TS-program = xelatex

\documentclass{beamer}
%Set the slide theme
%Change to meet your taste
% Madrid, Copenhagen, Berlin, ... works
\usetheme{Madrid}
%\usetheme{metropolis}


\usepackage{xecolor}
\usepackage{amsmath}
%\usefonttheme[onlymath]{serif} %Change the math font

\usepackage{xepersian}
\settextfont[Path=fonts/]{Parastoo}

%---------------------------------------------------------------------------------
% Seetings to force Beamer works with Xepersian and RTL typesetting
%-------------------------------------------------------------------------------
%\raggedleft

% For right to left lists (itemize and enumerate)
\makeatletter
\newcommand{\RTList}{\raggedleft\rightskip\@totalleftmargin}
\makeatother
% Correct the bullet for RTL texts
\setbeamertemplate{itemize item}{\scriptsize\raise1.25pt%
 \hbox{\donotcoloroutermaths$\blacktriangleleft$}} 

% To force beamer use numbering in captions
\setbeamertemplate{caption}[numbered]{}% Number float-like environments



%---------------------------------------------------------------------------------
\title{
	عنوان پروژه
}
\subtitle{}
\author{پرهام الوانی}
\institute{دانشکده مهندسی کامپیوتر و فناوری اطلاعات}
\date{پاییز ۱۳۹۶}

\begin{document}
\begin{persian}
%------------------------------------------
% Title page
%------------------------------------------
\begin{frame}
\maketitle
\end{frame}

% To adjust the paragraphs in RTL
\everypar{\rightskip\rightmargin}
%-------------------------------------------------------------------------------
\begin{frame}{طرح مساله}
	\begin{itemize}\RTList
		\item سمت کاربر
		\item سمت دیتاسنتر
	\end{itemize}
\end{frame}
\begin{frame}{طرح مساله}
\begin{latin}\begin{thebibliography}{9}

\bibitem{eramo1}
Eramo, V., Miucci, E., Ammar, M., Lavacca, F. G. (2017).
\emph{An approach for service function chain routing and virtual function network instance migration in network function virtualization architectures.}
\relax IEEE/ACM Transactions on Networking, 25(4), 2008–2025.

\end{thebibliography}\end{latin}
\end{frame}
\begin{frame}{طرح مساله}
\par
دیتاسنترها مصرف انرژی زیادی دارند، یکی از مسائل مطرح کاهش انرژی دیتاسنترها می‌باشد.
این کاهش مصرف می‌بایست در تکنولوژی‌های جدید نیز صورت گیرد. یکی از این تکنولوژی‌های جدید NFV
است. 
\end{frame}

\end{persian}
\end{document}