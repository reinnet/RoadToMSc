% !TeX TS-program = xelatex

\documentclass{beamer}
%Set the slide theme
%Change to meet your taste
% Madrid, Copenhagen, Berlin, ... works
\usetheme{metropolis}

\usepackage{multicol}

\usepackage{xecolor}
\usepackage{amsmath}
%\usefonttheme[onlymath]{serif} %Change the math font

\usepackage{multirow}
\usepackage{tabularx}
\usepackage{booktabs}

\usepackage{xepersian}
\settextfont[Path=fonts/]{Parastoo}

%---------------------------------------------------------------------------------
% Seetings to force Beamer works with Xepersian and RTL typesetting
%-------------------------------------------------------------------------------
%\raggedleft

% For right to left lists (itemize and enumerate)
\makeatletter
\newcommand{\RTList}{\raggedleft\rightskip\@totalleftmargin}
\makeatother
% Correct the bullet for RTL texts
\setbeamertemplate{itemize item}{\scriptsize\raise1.25pt%
 \hbox{\donotcoloroutermaths$\blacktriangleleft$}} 

% To force beamer use numbering in captions
\setbeamertemplate{caption}[numbered]{}% Number float-like environments



%---------------------------------------------------------------------------------
\title{
    زنجیره‌سازی کارکردهای مجازی سرویس شبکه با لحاظ محدودیت منابع مدیریتی
}
\subtitle{}
\author{پرهام الوانی}
\institute{دانشکده مهندسی کامپیوتر و فناوری اطلاعات}
\date{بهار ۱۳۹۷}

\begin{document}
\begin{persian}
%------------------------------------------
% Title page
%------------------------------------------
\begin{frame}
\maketitle
\end{frame}

% To adjust the paragraphs in RTL
\everypar{\rightskip\rightmargin}
%-------------------------------------------------------------------------------
\begin{frame}{فهرست}
    \begin{itemize}\RTList{}
        \item مقدمه
        \item چالش‌ها
        \item مرورکارهای مرتبط
        \item تعریف مساله
    \end{itemize}
\end{frame}
\begin{frame}{مقدمه}
    \begin{itemize}\RTList{}
        \item عدم انعطاف‌پذیری معماری فعلی شبکه
        \item در مجازی‌سازی کارکرد شبکه با استفاده از مجازی‌سازی منابع، می‌توان کارکردها را بر روی سرورهای استاندارد اجرا کرد و بهره‌وری منابع و هزینه‌های انرژی را کاهش داد.
        \item زنجیره سازی کارکرد سرویس نیز امکان ایجاد زنجیره‌ای از کارکردها را به صورت پویا فراهم می‌کند. در مجازی‌سازی کارکرد شبکه با استفاده از مجازی‌سازی منابع، می‌توان کارکردها را بر روی سرورهای استاندارد اجرا کرد و بهره‌وری منابع و هزینه‌های انرژی را کاهش داد.
    \end{itemize}
\end{frame}
\begin{frame}{مقدمه}
    \begin{center}\begin{figure}
        \includegraphics[scale=0.4]{images/nfv-arch.jpg}
        \caption{معماری سطح بالای مجازی‌سازی کارکردهای شبکه}
    \end{figure}\end{center}
\end{frame}
\begin{frame}{مقدمه}
    \par
    \lr{NFVO} وظیفه‌ی استقرار زنجیره‌های کارکرد سرویس را برعهده دارد.
    همانگونه که در مستند \lr{ETSI} نیز آمده است هر نمونه از کارکردهای مجازی شبکه نیاز دارد
    تحت مدیریت یکی از \lr{VNFM}های موجود در شبکه باشد.
\end{frame}
\begin{frame}{سابقه‌ی کارها}
    \fontsize{6pt}{7.2}\selectfont
    \begin{table}[h]
        \caption{مقایسه مقالات پذیرش زنجیره‌های کارکرد سرویس}
        \vspace{0.5cm}
        \begin{tabularx}{\textwidth}{XXXXXXXXXXXXXXXXX}
            \toprule
            منبع &
            \multicolumn{4}{X}{منابع تخصیص یافته} &
            \multicolumn{2}{X}{محدودیت ظرفیت پردازشی نمونه} &
            \multicolumn{2}{X}{برخط یا برون خط} &
            \multicolumn{2}{X}{نگاشت کارکرد و لینک} &
            \multicolumn{2}{X}{انتساب کارکرد} &
            \multicolumn{2}{X}{اشتراک نمونه} &
            \multicolumn{2}{X}{تخصیص \lr{VNFM}} \\
            \midrule
            \lr{\#} &
            \lr{other} &
            \lr{MEM} &
            \lr{BW} &
            \lr{CPU} &
            دارد &
            ندارد &
            برخط &
            برون خط &
            کارکرد &
            لینک &
            یک نمونه &
            چند نمونه &
            دارد &
            ندارد &
            دارد &
            ندارد \\
            \midrule
            \cite{Eramo2016} &
            \lr{---} &
            \lr{---} &
            \checkmark&
            \checkmark&
            \lr{---}&
            \checkmark&
            \lr{---} &
            \checkmark&
            \checkmark&
            \checkmark&
            \checkmark&
            \lr{---} &
            \lr{---} &
            \checkmark&
            \lr{---} &
            \checkmark\\
            \midrule
            \cite{Ghaznavi2017} &
            \lr{---} &
            \lr{---} &
            \checkmark&
            \checkmark&
            \checkmark&
            \lr{---} &
            \lr{---} &
            \checkmark&
            \checkmark&
            \checkmark&
            \lr{---} &
            \checkmark&
            \lr{---} &
            \checkmark&
            \lr{---} &
            \checkmark\\
            \midrule
            \cite{Huang2017} &
            \lr{---} &
            \lr{---} &
            \checkmark&
            \checkmark&
            \checkmark&
            \lr{---} &
            \lr{---} &
            \checkmark&
            \checkmark&
            \checkmark&
            \lr{---} &
            \checkmark&
            \lr{---} &
            \checkmark&
            \lr{---} &
            \checkmark\\
            \midrule
            پژوهش حاضر &
            \lr{---} &
            \checkmark&
            \checkmark&
            \checkmark&
            \checkmark&
            \lr{---} &
            \lr{---} &
            \checkmark&
            \checkmark&
            \checkmark&
            \checkmark&
            \lr{---}&
            \lr{---}&
            \checkmark&
            \lr{---} &
            \checkmark\\
            \bottomrule
        \end{tabularx}
    \end{table}
\end{frame}
\begin{frame}{سابقه‌ی کارها}
    \par
    \cite{AbuLebdeh2017}
    هدف کاهش هزینه‌ی عملیاتی در حالی که تاخیرهای ارتباطی و محدودیت‌های ظرفیت رعایت می‌شوند.
    در این مقاله فرض می‌شود جایگذاری کارکردهای مجازی شبکه صورت پذیرفته است.
\end{frame}
\begin{frame}{تعریف مساله}
    \par
    پذیرفتن بیشترین تقاضای زنجیره‌ کارکرد سرویس با در نظر گرفتن نیاز هر نمونه کارکرد مجازی شبکه به یک \lr{VNFM}.
\end{frame}
\begin{frame}{تعریف مساله}
    \begin{itemize}\RTList{}
        \item توپولوژی زیرساخت شامل پنهای باند لینک‌ها و ظرفیت \lr{NFVI-PoP}ها موجود است.
        \item n تقاضای زنجیره‌ کارکرد سرویس به صورت کامل و از پیش مشخص شده داریم.
        \item هر تقاضا شامل نوع و تعداد نمونه‌های مجازی و پنهای باند لینک‌های مجازی می‌باشد.
        \item F نوع کارکرد مجازی شبکه تعریف شده است که هر یک مقدار مشخصی از حافظه را مصرف می‌کنند.
        \item تعداد پردازنده‌هایی که به هر نمونه تخصیص می‌یابد با توجه به ترافیک ورودی نمونه مشخص می‌شود.
        \item نمونه‌ها بین زنجیره‌ها به اشتراک گذاشته نمی‌شوند.
    \end{itemize}
\end{frame}
\begin{frame}{تعریف مساله}
    \begin{itemize}\RTList{}
        \item محدودیت ظرفیت لینک‌ها
        \item محدودیت توان پردازش سرورهای فیزیکی با توجه به میزان حافظه و تعداد پردازنده‌ها
    \end{itemize}
\end{frame}
\begin{frame}{تعریف مساله}
    \begin{itemize}\RTList{}
        \item برای مدیریت یکدست و آسان‌تر زنجیره‌ها و در عین حال جمع‌آوری راحتر خطاها، برای هر زنجیره یک \lr{VNFM} تخصیص می‌دهیم.
        \item \lr{VNFM}ها می‌توانند بین زنجیره به اشتراک گذاشته شوند.
        \item هر نمونه از \lr{VNFM}ها می‌تواند تعداد مشخصی از نمونه‌های کارکرد مجازی شبکه را سرویس دهد. 
        \item برای ارتباط میان هر نمونه از \lr{VNFM}ها و \lr{VNF}ها پهنای باند مشخصی رزرو می‌گردد.
        \item بر روی هر \lr{NFVI-PoP} حداکثر یک نمونه \lr{VNFM} مستقر می‌گردد.
    \end{itemize}
\end{frame}
\begin{frame}{چالش‌های مساله}
    \begin{itemize}\RTList{}
        \item جایگذاری و مسیریابی توامان زنجیره‌های کارکرد سرویس
        \item تخصیص منابع مدیریتی به زنجیره‌ها و مسیریابی ارتباط مدیریتی
    \end{itemize}
\end{frame}
\begin{frame}{نوآوری‌های مساله}
    \par
    یکی از وظایف \lr{VNFM} مانیتور کردن وضعیت و خطاهای نمونه‌ها می‌باشد
    این امر باعث افزایش بار پردازشی \lr{VNFM} می‌گردد
    و از سوی دیگر تحلیل این اطلاعات می‌بایست با تاخیر معقولی صورت پذیرد که این امر
    نیاز به یک بستر ارتباطی مطمئن دارد.
\end{frame}
\begin{frame}{معیار و نحوه‌ی ارزیابی}
    \begin{itemize}\RTList{}
        \item مدل‌سازی مساله
        \item حل مساله‌ی بهینه در ابعاد کوچک
        \item پیاده‌سازی راه‌حل مکاشفه‌ای
        \item مقایسه‌ی نتایج راه‌حل مکاشفه‌ای با جواب بهینه
        \item مقایسه با کارهای مرتبط که نیازمندی‌های مدیریتی را مدنظر قرار نداده‌اند
    \end{itemize}
\end{frame}
\begin{frame}{مراجع}
    \begin{latin}
    \bibliographystyle{IEEEtran}
    \bibliography{references}
    \end{latin}
\end{frame}
\begin{frame}
    \begin{multicols}{3}
        \begin{center}	
            فرمول‌بندی
        \end{center}
    \end{multicols}
    \begin{center}	
        پذیرش زنجیره‌های کارکرد سرویس
        \\
        و مدیریت آن‌ها با استفاده از \lr{VNFM}
    \end{center}
\end{frame}

\begin{frame}{فرمول‌بندی}
    \par
    متغیرهای تصمیم‌گیری
    \begin{latin}\begin{tabular}{c p{10cm}}
        $x_h$ & binary variable assuming the value 1 if the $h$th SFC request is accepted; otherwise its value is zero \\
        $y_{wk}$ & the number of VNF instances of type $k$ that are used in server $w \in V_s^{PN}$ \\
        $z^k_{vw}$ & binary variable assuming the value 1 if the VNF node $v \in \cup_{i=1}^{T} V_{i, F}^{SFC}$ is served by the VNF instance of type k in the server $w \in V_s^{PN}$ \\
    \end{tabular}\end{latin}
\end{frame}
\begin{frame}{فرمول‌بندی}
    \par
    متغیرهای تصمیم‌گیری
    \begin{latin}\begin{tabular}{c p{10cm}}
        $\bar{y}_w$ & binary varibale assuming the value 1 if VNFM on server $w \in V_s^{PN}$ is used; otherwise its value is zero\\
        $\bar{z}_{hw}$ & binary variable assuming the value 1 if $h$th SFC is assigned to VNFM on server $w \in V_s^{PN}$\\
    \end{tabular}\end{latin}
\end{frame}
\begin{frame}{فرمول‌بندی}
    \begin{latin}\begin{align}
    \max \sum_{h=1}^T x_h
    \end{align}\end{latin}
\end{frame}
\begin{frame}{فرمول‌بندی}
    \par
    محدودیت حافظه نودها
    \begin{latin}\begin{align}
    \sum_{k=1}^F y_{wk} memory(k) + \bar{y_w} \bar{memory} \le N_{ram}^{PN}(w)
    \quad
    \forall w \in V_s^{PN}
    \end{align}\end{latin}
    \par
    محدودیت تعداد پردازنده‌های نودها
    \begin{latin}\begin{align}
    \sum_{k=1}^F y_{wk} core(k) + \bar{y_w} \bar{core} \le N_{core}^{PN}(w)
    \quad
    \forall w \in V_s^{PN}
    \end{align}\end{latin}
\end{frame}
\begin{frame}{فرمول‌بندی}
    \par
    اگر \lr{VNF}, \lr{v}
    توسط \lr{VNF instance} نوع \lr{k}
    روی سرور \lr{w} سرویس شود می‌بایست
    \lr{VNF instance} نوع \lr{k}
    روی سرور \lr{w} فعال شود.
    \par
    اشتراک گذاری \lr{VNF}ها پشتیبانی نمی‌گردد.
    \begin{latin}\begin{align}
    \sum_{v \in \cup_{i=1}^T V_{i, F}^{SFC}} z_{vw}^k \le y_{wk}
    \quad
    \forall w \in V_s^{PN}, \forall k \in [1,\ldots, F]
    \end{align}\end{latin}
\end{frame}
\begin{frame}{فرمول‌بندی}
    \par
    اگر تقاضای \lr{h}ام پذیرفته شده باشد
    می‌بایست تمام \lr{VNF node}های آن‌
    سرویس شده باشند.
    یک \lr{VNF} حداکثر یکبار سرویس داده شود.
    \begin{latin}\begin{align}
        x_h = \sum_{k=1}^{F} \sum_{w \in V_{s}^{PN}} z_{vw}^{k}
        \quad
        \forall v \in V_{h,F}^{SFC}, \forall h \in [1,\ldots, T]
    \end{align}\end{latin}
\end{frame}
\begin{frame}{فرمول‌بندی}
    \par
    اگر تقاضای \lr{h}ام پذیرفته شده باشد
    می‌بایست توسط یک \lr{VNFM} سرویس شده باشد.
    \begin{latin}\begin{align}
        x_h = \sum_{w \in V_{s}^{PN}} \bar{z}_{hw}
        \quad
        \forall h \in [1,\ldots, T]
    \end{align}\end{latin}
\end{frame}
\begin{frame}{فرمول‌بندی}
    \par
    اگر \lr{SFC}، \lr{i}
    توسط \lr{VNFM} روی سرور \lr{w}
    سرویس شود می‌بایست \lr{VNFM} سرور \lr{w}
    فعال شود.
    \begin{latin}\begin{align}
        \bar{z}_{hw} \le \bar{y}_w
        \quad
        \forall w \in V_{s}^{PN}, \forall h \in [1,\ldots, T]
    \end{align}\end{latin}
    \par
    محدودیت ظرفیت سرویس‌دهی \lr{VNFM}
    \begin{latin}\begin{align}
        \sum_{i=1}^{T} z_{iw} \le capacity
        \quad
        \forall w \in V_{s}^{PN}
    \end{align}\end{latin}
\end{frame}
\begin{frame}{فرمول‌بندی}
    \par
    متغیرهای تصمیم‌گیری
    \begin{latin}\begin{tabular}{c p{10cm}}
        $\tau^{(u,v)}_{ij}$ & binary variable assuming the value 1 if the virual link $(u,v)$ is routed on the physical network link $(i,j)$\\
        $\bar{\tau}^{v}_{ij}$ & binary variable assuming the value 1 if the managemnt of VNF node $v$ is routed on the physical network link $(i,j)$\\
    \end{tabular}\end{latin}
\end{frame}
\begin{frame}{فرمول‌بندی}
    \par
    \lr{Flow Conservation}
    \begin{latin}\begin{align}
        \sum_{(i,j) \in E^{PN}} \tau_{ij}^{(u,v)} - \sum_{(j,i) \in E^{PN}} \tau_{ji}^{(u,v)} = \sum_{k=1}^{F} z_{ui}^{k} - \sum_{k=1}^{F} z_{vi}^{k} \nonumber \\
        \forall i \in V_{S}^{PN}, (u,v) \in E_{h}^{SFC}, h \in [1,\ldots, T]
    \end{align}\end{latin}
\end{frame}
\begin{frame}{فرمول‌بندی}
    \par
    \lr{Flow Conservation}
    \begin{latin}\begin{align}
        \sum_{(i,j) \in E^{PN}} \bar{\tau}_{ij}^{v} - \sum_{(j,i) \in E^{PN}} \bar{\tau}_{ji}^{v} = \sum_{k=1}^{F} z_{vi}^{k} - \bar{z}_{hi} \nonumber \\
        \forall i \in V_{S}^{PN}, v \in V_{h, F}^{SFC}, h \in [1,\ldots, T]
    \end{align}\end{latin}
\end{frame}
\begin{frame}{فرمول‌بندی}
    \par
    محدودیت ظرفیت لینک‌ها
    \begin{latin}\begin{align}
        \sum_{v \in \cup_{i=1}^{T} V_{i,F}^{SFC}} \bar{\tau}_{ij}^{v} * \bar{bandwidth} + \sum_{(u,v) \in \cup_{i=1}^{T} E_{i}^{SFC}} \tau_{ij}^{(u,v)} * bandwidth(u,v) \le C_{ij}
    \end{align}\end{latin}
\end{frame}
\end{persian}
\end{document}
