% !TeX TS-program = xelatex

\documentclass{beamer}
%Set the slide theme
%Change to meet your taste
% Madrid, Copenhagen, Berlin, ... works
\usetheme{Madrid}
%\usetheme{metropolis}


\usepackage{xecolor}
\usepackage{amsmath}
%\usefonttheme[onlymath]{serif} %Change the math font

\usepackage{xepersian}
\settextfont[Path=fonts/]{Parastoo}

%---------------------------------------------------------------------------------
% Seetings to force Beamer works with Xepersian and RTL typesetting
%-------------------------------------------------------------------------------
%\raggedleft

% For right to left lists (itemize and enumerate)
\makeatletter
\newcommand{ \RTList}{\raggedleft\rightskip\@totalleftmargin}
\makeatother
% Correct the bullet for RTL texts
\setbeamertemplate{itemize item}{\scriptsize\raise1.25pt%
 \hbox{\donotcoloroutermaths$\blacktriangleleft$}} 

% To force beamer use numbering in captions
\setbeamertemplate{caption}[numbered]{}% Number float-like environments



%---------------------------------------------------------------------------------
\title{
	عنوان پروژه
}
\subtitle{}
\author{پرهام الوانی}
\institute{دانشکده مهندسی کامپیوتر و فناوری اطلاعات}
\date{پاییز ۱۳۹۶}

\begin{document}
\begin{persian}
%------------------------------------------
% Title page
%------------------------------------------
\begin{frame}
\maketitle
\end{frame}

% To adjust the paragraphs in RTL
\everypar{\rightskip\rightmargin}
%-------------------------------------------------------------------------------
\begin{frame}{سرآغاز}
\section{مبانی}
\subsection{متن ساده}
این یک نمونه بسیار ساده از اسلاید است که با بیمر و زی‌پرشین ساخته شده‌است.

از فونت آزاد Roya XB برای این اسلاید استفاده شده‌است. این فونت در اینترنت موجود است و باید روی کامپیوتر شما نصب شده باشد یا در فولدر قابل دسترس برای زی‌لاتک باشد

تنظیمات اندکی در بخش آغازین برای اصلاح لیست‌ها و عنوان اسلایدها اضافه شده است. همچنین نحوه شماره‌گذاری تصاویر و جدول‌ها تنظیم شده‌است.

بسیاری از تم‌های استاندارد بیمر با این الگو قابل استفاده است.

استفاده از پانویس توصیه نمی شود. سفارش می‌شود که جدول فهرست مطالب به شکل دستی باشد. این اسلاید با بسیاری از تم‌های بیمر کار میکند اگرچه ممکن است اشکالاتی وجود داشته باشد.
\end{frame}

\end{persian}
\end{document}